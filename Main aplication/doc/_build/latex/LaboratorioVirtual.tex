%% Generated by Sphinx.
\def\sphinxdocclass{report}
\documentclass[letterpaper,10pt,spanish]{sphinxmanual}
\ifdefined\pdfpxdimen
   \let\sphinxpxdimen\pdfpxdimen\else\newdimen\sphinxpxdimen
\fi \sphinxpxdimen=.75bp\relax

\PassOptionsToPackage{warn}{textcomp}
\usepackage[utf8]{inputenc}
\ifdefined\DeclareUnicodeCharacter
% support both utf8 and utf8x syntaxes
  \ifdefined\DeclareUnicodeCharacterAsOptional
    \def\sphinxDUC#1{\DeclareUnicodeCharacter{"#1}}
  \else
    \let\sphinxDUC\DeclareUnicodeCharacter
  \fi
  \sphinxDUC{00A0}{\nobreakspace}
  \sphinxDUC{2500}{\sphinxunichar{2500}}
  \sphinxDUC{2502}{\sphinxunichar{2502}}
  \sphinxDUC{2514}{\sphinxunichar{2514}}
  \sphinxDUC{251C}{\sphinxunichar{251C}}
  \sphinxDUC{2572}{\textbackslash}
\fi
\usepackage{cmap}
\usepackage[T1]{fontenc}
\usepackage{amsmath,amssymb,amstext}
\usepackage{babel}



\usepackage{times}
\expandafter\ifx\csname T@LGR\endcsname\relax
\else
% LGR was declared as font encoding
  \substitutefont{LGR}{\rmdefault}{cmr}
  \substitutefont{LGR}{\sfdefault}{cmss}
  \substitutefont{LGR}{\ttdefault}{cmtt}
\fi
\expandafter\ifx\csname T@X2\endcsname\relax
  \expandafter\ifx\csname T@T2A\endcsname\relax
  \else
  % T2A was declared as font encoding
    \substitutefont{T2A}{\rmdefault}{cmr}
    \substitutefont{T2A}{\sfdefault}{cmss}
    \substitutefont{T2A}{\ttdefault}{cmtt}
  \fi
\else
% X2 was declared as font encoding
  \substitutefont{X2}{\rmdefault}{cmr}
  \substitutefont{X2}{\sfdefault}{cmss}
  \substitutefont{X2}{\ttdefault}{cmtt}
\fi


\usepackage[Sonny]{fncychap}
\ChNameVar{\Large\normalfont\sffamily}
\ChTitleVar{\Large\normalfont\sffamily}
\usepackage{sphinx}

\fvset{fontsize=\small}
\usepackage{geometry}


% Include hyperref last.
\usepackage{hyperref}
% Fix anchor placement for figures with captions.
\usepackage{hypcap}% it must be loaded after hyperref.
% Set up styles of URL: it should be placed after hyperref.
\urlstyle{same}
\addto\captionsspanish{\renewcommand{\contentsname}{Contenido:}}

\usepackage{sphinxmessages}
\setcounter{tocdepth}{3}
\setcounter{secnumdepth}{3}


\title{Laboratorio Virtual de sistemas de control clasicos y difusos utilizando software libre}
\date{09 de marzo de 2020}
\release{1.0}
\author{Kleiver J. Carrasco M.}
\newcommand{\sphinxlogo}{\vbox{}}
\renewcommand{\releasename}{Versión}
\makeindex
\begin{document}

\ifdefined\shorthandoff
  \ifnum\catcode`\=\string=\active\shorthandoff{=}\fi
  \ifnum\catcode`\"=\active\shorthandoff{"}\fi
\fi

\pagestyle{empty}
\sphinxmaketitle
\pagestyle{plain}
\sphinxtableofcontents
\pagestyle{normal}
\phantomsection\label{\detokenize{index::doc}}



\chapter{Promociones de Widgets}
\label{\detokenize{codigos/Promociones:promociones-de-widgets}}\label{\detokenize{codigos/Promociones::doc}}

\section{focusLineEdit}
\label{\detokenize{codigos/Promociones:module-focusLineEdit}}\label{\detokenize{codigos/Promociones:focuslineedit}}\index{focusLineEdit (módulo)@\spxentry{focusLineEdit}\spxextra{módulo}}
Archivo para definir la clase FocusLineEdit, esta clases permite promocionar un lineEdit y agregar el evento focus con el fin de generar una señal al dar clic en un lineEdit
\index{FocusLineEdit (clase en focusLineEdit)@\spxentry{FocusLineEdit}\spxextra{clase en focusLineEdit}}

\begin{fulllineitems}
\phantomsection\label{\detokenize{codigos/Promociones:focusLineEdit.FocusLineEdit}}\pysigline{\sphinxbfcode{\sphinxupquote{class }}\sphinxcode{\sphinxupquote{focusLineEdit.}}\sphinxbfcode{\sphinxupquote{FocusLineEdit}}}
Clase basica para promocionar el lineEdit
\begin{quote}\begin{description}
\item[{Parámetros}] \leavevmode
\sphinxstyleliteralstrong{\sphinxupquote{QtWidgets}} (\sphinxstyleliteralemphasis{\sphinxupquote{objectType}}) \textendash{} Clase base de los Widgets

\end{description}\end{quote}
\index{focusInEvent() (método de focusLineEdit.FocusLineEdit)@\spxentry{focusInEvent()}\spxextra{método de focusLineEdit.FocusLineEdit}}

\begin{fulllineitems}
\phantomsection\label{\detokenize{codigos/Promociones:focusLineEdit.FocusLineEdit.focusInEvent}}\pysiglinewithargsret{\sphinxbfcode{\sphinxupquote{focusInEvent}}}{\emph{event}}{}
Evento de focus para el lineEdit
\begin{quote}\begin{description}
\item[{Parámetros}] \leavevmode
\sphinxstyleliteralstrong{\sphinxupquote{event}} (\sphinxstyleliteralemphasis{\sphinxupquote{tuple}}) \textendash{} Evento generado

\end{description}\end{quote}

\end{fulllineitems}


\end{fulllineitems}



\section{mlpwidget}
\label{\detokenize{codigos/Promociones:module-mlpwidget}}\label{\detokenize{codigos/Promociones:mlpwidget}}\index{mlpwidget (módulo)@\spxentry{mlpwidget}\spxextra{módulo}}
Archivo para definir las clases MlpWidget, MlpWidgetNoToolbar, MlpWidgetSubplot y MlpWidget3D, estas clases son utilizadas por qtdesigner para promocionar un QGraphicsView a las clases aca definidas en orden de mostrar las graficas en un QGraphicsView
\index{MlpWidget (clase en mlpwidget)@\spxentry{MlpWidget}\spxextra{clase en mlpwidget}}

\begin{fulllineitems}
\phantomsection\label{\detokenize{codigos/Promociones:mlpwidget.MlpWidget}}\pysiglinewithargsret{\sphinxbfcode{\sphinxupquote{class }}\sphinxcode{\sphinxupquote{mlpwidget.}}\sphinxbfcode{\sphinxupquote{MlpWidget}}}{\emph{parent=None}}{}
Clase basica para mostrar graficas utilizando Matplotlib
\begin{quote}\begin{description}
\item[{Parámetros}] \leavevmode
\sphinxstyleliteralstrong{\sphinxupquote{QGraphicsView}} (\sphinxstyleliteralemphasis{\sphinxupquote{objectType}}) \textendash{} Clase base del QGraphicsView

\end{description}\end{quote}
\index{\_\_init\_\_() (método de mlpwidget.MlpWidget)@\spxentry{\_\_init\_\_()}\spxextra{método de mlpwidget.MlpWidget}}

\begin{fulllineitems}
\phantomsection\label{\detokenize{codigos/Promociones:mlpwidget.MlpWidget.__init__}}\pysiglinewithargsret{\sphinxbfcode{\sphinxupquote{\_\_init\_\_}}}{\emph{parent=None}}{}
Initialize self.  See help(type(self)) for accurate signature.

\end{fulllineitems}


\end{fulllineitems}

\index{MlpWidget3D (clase en mlpwidget)@\spxentry{MlpWidget3D}\spxextra{clase en mlpwidget}}

\begin{fulllineitems}
\phantomsection\label{\detokenize{codigos/Promociones:mlpwidget.MlpWidget3D}}\pysiglinewithargsret{\sphinxbfcode{\sphinxupquote{class }}\sphinxcode{\sphinxupquote{mlpwidget.}}\sphinxbfcode{\sphinxupquote{MlpWidget3D}}}{\emph{parent=None}}{}
Clase basica para mostrar graficas en 3D utilizando Matplotlib
\begin{quote}\begin{description}
\item[{Parámetros}] \leavevmode
\sphinxstyleliteralstrong{\sphinxupquote{QGraphicsView}} (\sphinxstyleliteralemphasis{\sphinxupquote{objectType}}) \textendash{} Clase base del QGraphicsView

\end{description}\end{quote}
\index{\_\_init\_\_() (método de mlpwidget.MlpWidget3D)@\spxentry{\_\_init\_\_()}\spxextra{método de mlpwidget.MlpWidget3D}}

\begin{fulllineitems}
\phantomsection\label{\detokenize{codigos/Promociones:mlpwidget.MlpWidget3D.__init__}}\pysiglinewithargsret{\sphinxbfcode{\sphinxupquote{\_\_init\_\_}}}{\emph{parent=None}}{}
Initialize self.  See help(type(self)) for accurate signature.

\end{fulllineitems}


\end{fulllineitems}

\index{MlpWidgetNoToolbar (clase en mlpwidget)@\spxentry{MlpWidgetNoToolbar}\spxextra{clase en mlpwidget}}

\begin{fulllineitems}
\phantomsection\label{\detokenize{codigos/Promociones:mlpwidget.MlpWidgetNoToolbar}}\pysiglinewithargsret{\sphinxbfcode{\sphinxupquote{class }}\sphinxcode{\sphinxupquote{mlpwidget.}}\sphinxbfcode{\sphinxupquote{MlpWidgetNoToolbar}}}{\emph{parent=None}}{}
Clase para mostrar graficas utilizando Matplotlib sin el toolbar
\begin{quote}\begin{description}
\item[{Parámetros}] \leavevmode
\sphinxstyleliteralstrong{\sphinxupquote{QGraphicsView}} (\sphinxstyleliteralemphasis{\sphinxupquote{objectType}}) \textendash{} Clase base del QGraphicsView

\end{description}\end{quote}
\index{\_\_init\_\_() (método de mlpwidget.MlpWidgetNoToolbar)@\spxentry{\_\_init\_\_()}\spxextra{método de mlpwidget.MlpWidgetNoToolbar}}

\begin{fulllineitems}
\phantomsection\label{\detokenize{codigos/Promociones:mlpwidget.MlpWidgetNoToolbar.__init__}}\pysiglinewithargsret{\sphinxbfcode{\sphinxupquote{\_\_init\_\_}}}{\emph{parent=None}}{}
Initialize self.  See help(type(self)) for accurate signature.

\end{fulllineitems}


\end{fulllineitems}

\index{MlpWidgetSubplot (clase en mlpwidget)@\spxentry{MlpWidgetSubplot}\spxextra{clase en mlpwidget}}

\begin{fulllineitems}
\phantomsection\label{\detokenize{codigos/Promociones:mlpwidget.MlpWidgetSubplot}}\pysiglinewithargsret{\sphinxbfcode{\sphinxupquote{class }}\sphinxcode{\sphinxupquote{mlpwidget.}}\sphinxbfcode{\sphinxupquote{MlpWidgetSubplot}}}{\emph{parent=None}}{}
Clase para mostrar graficas en subplots utilizando Matplotlib
\begin{quote}\begin{description}
\item[{Parámetros}] \leavevmode
\sphinxstyleliteralstrong{\sphinxupquote{QGraphicsView}} (\sphinxstyleliteralemphasis{\sphinxupquote{objectType}}) \textendash{} Clase base del QGraphicsView

\end{description}\end{quote}
\index{\_\_init\_\_() (método de mlpwidget.MlpWidgetSubplot)@\spxentry{\_\_init\_\_()}\spxextra{método de mlpwidget.MlpWidgetSubplot}}

\begin{fulllineitems}
\phantomsection\label{\detokenize{codigos/Promociones:mlpwidget.MlpWidgetSubplot.__init__}}\pysiglinewithargsret{\sphinxbfcode{\sphinxupquote{\_\_init\_\_}}}{\emph{parent=None}}{}
Initialize self.  See help(type(self)) for accurate signature.

\end{fulllineitems}


\end{fulllineitems}



\section{pyqtgraphWidget}
\label{\detokenize{codigos/Promociones:module-pyqtgraphWidget}}\label{\detokenize{codigos/Promociones:pyqtgraphwidget}}\index{pyqtgraphWidget (módulo)@\spxentry{pyqtgraphWidget}\spxextra{módulo}}
Archivo para definir las clases PgraphWidget y PgraphWidgetpid, estas clases son utilizadas por qtdesigner para promocionar un QGraphicsView a las clases aca definidas en orden de mostrar las graficas en un QGraphicsView
\index{PgraphWidget (clase en pyqtgraphWidget)@\spxentry{PgraphWidget}\spxextra{clase en pyqtgraphWidget}}

\begin{fulllineitems}
\phantomsection\label{\detokenize{codigos/Promociones:pyqtgraphWidget.PgraphWidget}}\pysiglinewithargsret{\sphinxbfcode{\sphinxupquote{class }}\sphinxcode{\sphinxupquote{pyqtgraphWidget.}}\sphinxbfcode{\sphinxupquote{PgraphWidget}}}{\emph{parent=None}}{}
Clase para las graficas utilizadas en la prueba de los controladores difusos, PyQtGraph es acto para realizar graficas en tiempo real
\begin{quote}\begin{description}
\item[{Parámetros}] \leavevmode
\sphinxstyleliteralstrong{\sphinxupquote{QGraphicsView}} (\sphinxstyleliteralemphasis{\sphinxupquote{objectType}}) \textendash{} Clase base del QGraphicsView

\end{description}\end{quote}
\index{\_\_init\_\_() (método de pyqtgraphWidget.PgraphWidget)@\spxentry{\_\_init\_\_()}\spxextra{método de pyqtgraphWidget.PgraphWidget}}

\begin{fulllineitems}
\phantomsection\label{\detokenize{codigos/Promociones:pyqtgraphWidget.PgraphWidget.__init__}}\pysiglinewithargsret{\sphinxbfcode{\sphinxupquote{\_\_init\_\_}}}{\emph{parent=None}}{}
Initialize self.  See help(type(self)) for accurate signature.

\end{fulllineitems}


\end{fulllineitems}

\index{PgraphWidgetpid (clase en pyqtgraphWidget)@\spxentry{PgraphWidgetpid}\spxextra{clase en pyqtgraphWidget}}

\begin{fulllineitems}
\phantomsection\label{\detokenize{codigos/Promociones:pyqtgraphWidget.PgraphWidgetpid}}\pysiglinewithargsret{\sphinxbfcode{\sphinxupquote{class }}\sphinxcode{\sphinxupquote{pyqtgraphWidget.}}\sphinxbfcode{\sphinxupquote{PgraphWidgetpid}}}{\emph{parent=None}}{}
Clase para las graficas utilizadas en el tunning de controladores PID, PyQtGraph es acto para realizar graficas en tiempo real
\begin{quote}\begin{description}
\item[{Parámetros}] \leavevmode
\sphinxstyleliteralstrong{\sphinxupquote{QGraphicsView}} (\sphinxstyleliteralemphasis{\sphinxupquote{objectType}}) \textendash{} Clase base del QGraphicsView

\end{description}\end{quote}
\index{\_\_init\_\_() (método de pyqtgraphWidget.PgraphWidgetpid)@\spxentry{\_\_init\_\_()}\spxextra{método de pyqtgraphWidget.PgraphWidgetpid}}

\begin{fulllineitems}
\phantomsection\label{\detokenize{codigos/Promociones:pyqtgraphWidget.PgraphWidgetpid.__init__}}\pysiglinewithargsret{\sphinxbfcode{\sphinxupquote{\_\_init\_\_}}}{\emph{parent=None}}{}
Initialize self.  See help(type(self)) for accurate signature.

\end{fulllineitems}


\end{fulllineitems}



\chapter{Archivo principal (main)}
\label{\detokenize{codigos/Principal:archivo-principal-main}}\label{\detokenize{codigos/Principal::doc}}

\section{Archivo Handler para la función de análisis}
\label{\detokenize{codigos/analisisHandler:archivo-handler-para-la-funcion-de-analisis}}\label{\detokenize{codigos/analisisHandler::doc}}

\subsection{Archivo de rutinas para la función de análisis}
\label{\detokenize{codigos/rutinas_analisis:module-rutinas_analisis}}\label{\detokenize{codigos/rutinas_analisis:archivo-de-rutinas-para-la-funcion-de-analisis}}\label{\detokenize{codigos/rutinas_analisis::doc}}\index{rutinas\_analisis (módulo)@\spxentry{rutinas\_analisis}\spxextra{módulo}}
Archivo que contiene todas las rutinas necesarias para la funcionalidad de analisis de sistemas de control
\index{margenes\_ganancias() (en el módulo rutinas\_analisis)@\spxentry{margenes\_ganancias()}\spxextra{en el módulo rutinas\_analisis}}

\begin{fulllineitems}
\phantomsection\label{\detokenize{codigos/rutinas_analisis:rutinas_analisis.margenes_ganancias}}\pysiglinewithargsret{\sphinxcode{\sphinxupquote{rutinas\_analisis.}}\sphinxbfcode{\sphinxupquote{margenes\_ganancias}}}{\emph{self}, \emph{system}, \emph{mag}, \emph{phase}, \emph{omega}}{}
Función para obtener el margen de ganancia y el margen de fase
\begin{quote}\begin{description}
\item[{Parámetros}] \leavevmode\begin{itemize}
\item {} 
\sphinxstyleliteralstrong{\sphinxupquote{system}} (\sphinxstyleliteralemphasis{\sphinxupquote{LTI}}) \textendash{} Representación del sistema

\item {} 
\sphinxstyleliteralstrong{\sphinxupquote{mag}} (\sphinxstyleliteralemphasis{\sphinxupquote{numpyArray}}) \textendash{} Magnitud de la respuesta en frecuencia

\item {} 
\sphinxstyleliteralstrong{\sphinxupquote{phase}} (\sphinxstyleliteralemphasis{\sphinxupquote{numpyArray}}) \textendash{} Fase de la respuesta en frecuencia

\item {} 
\sphinxstyleliteralstrong{\sphinxupquote{omega}} (\sphinxstyleliteralemphasis{\sphinxupquote{numpyArray}}) \textendash{} Frecuencias utilizadas para la respuesta en frecuencia

\end{itemize}

\end{description}\end{quote}

\end{fulllineitems}

\index{rutina\_bode\_plot() (en el módulo rutinas\_analisis)@\spxentry{rutina\_bode\_plot()}\spxextra{en el módulo rutinas\_analisis}}

\begin{fulllineitems}
\phantomsection\label{\detokenize{codigos/rutinas_analisis:rutinas_analisis.rutina_bode_plot}}\pysiglinewithargsret{\sphinxcode{\sphinxupquote{rutinas\_analisis.}}\sphinxbfcode{\sphinxupquote{rutina\_bode\_plot}}}{\emph{self}, \emph{system}}{}
Funcion para obtener la respuesta en frecuencia del sistema y su respectiva graficacion en diagrama de bode
\begin{quote}\begin{description}
\item[{Parámetros}] \leavevmode
\sphinxstyleliteralstrong{\sphinxupquote{system}} (\sphinxstyleliteralemphasis{\sphinxupquote{LTI}}) \textendash{} Representacion del sistema

\end{description}\end{quote}

\end{fulllineitems}

\index{rutina\_impulse\_plot() (en el módulo rutinas\_analisis)@\spxentry{rutina\_impulse\_plot()}\spxextra{en el módulo rutinas\_analisis}}

\begin{fulllineitems}
\phantomsection\label{\detokenize{codigos/rutinas_analisis:rutinas_analisis.rutina_impulse_plot}}\pysiglinewithargsret{\sphinxcode{\sphinxupquote{rutinas\_analisis.}}\sphinxbfcode{\sphinxupquote{rutina\_impulse\_plot}}}{\emph{self}, \emph{system}, \emph{T}}{}
Funcion para obtener la respuesta impulso del sistema y su respectiva graficacion
\begin{quote}\begin{description}
\item[{Parámetros}] \leavevmode\begin{itemize}
\item {} 
\sphinxstyleliteralstrong{\sphinxupquote{system}} (\sphinxstyleliteralemphasis{\sphinxupquote{LTI}}) \textendash{} Representacion del sistema

\item {} 
\sphinxstyleliteralstrong{\sphinxupquote{T}} (\sphinxstyleliteralemphasis{\sphinxupquote{numpyArray}}) \textendash{} Vector de tiempo

\end{itemize}

\end{description}\end{quote}

\end{fulllineitems}

\index{rutina\_nichols\_plot() (en el módulo rutinas\_analisis)@\spxentry{rutina\_nichols\_plot()}\spxextra{en el módulo rutinas\_analisis}}

\begin{fulllineitems}
\phantomsection\label{\detokenize{codigos/rutinas_analisis:rutinas_analisis.rutina_nichols_plot}}\pysiglinewithargsret{\sphinxcode{\sphinxupquote{rutinas\_analisis.}}\sphinxbfcode{\sphinxupquote{rutina\_nichols\_plot}}}{\emph{self}, \emph{system}}{}
Funcion para obtener el diagram de nichols del sistema y su respectiva graficacion, la graficacion se realizo de forma interna en la libreria de control, para esto se moodifico la funcion nichols\_plot para poder enviar el axis y la figura, adicionalmente se realizaron algunas modificaciones para una mejor presentacion de la grafica
\begin{quote}\begin{description}
\item[{Parámetros}] \leavevmode
\sphinxstyleliteralstrong{\sphinxupquote{system}} (\sphinxstyleliteralemphasis{\sphinxupquote{LTI}}) \textendash{} Representacion del sistema

\end{description}\end{quote}

\end{fulllineitems}

\index{rutina\_nyquist\_plot() (en el módulo rutinas\_analisis)@\spxentry{rutina\_nyquist\_plot()}\spxextra{en el módulo rutinas\_analisis}}

\begin{fulllineitems}
\phantomsection\label{\detokenize{codigos/rutinas_analisis:rutinas_analisis.rutina_nyquist_plot}}\pysiglinewithargsret{\sphinxcode{\sphinxupquote{rutinas\_analisis.}}\sphinxbfcode{\sphinxupquote{rutina\_nyquist\_plot}}}{\emph{self}, \emph{system}}{}
Funcion para obtener la respuesta en frecuencia del sistema y su respectiva graficacion en diagrama de Nyquist
\begin{quote}\begin{description}
\item[{Parámetros}] \leavevmode
\sphinxstyleliteralstrong{\sphinxupquote{system}} (\sphinxstyleliteralemphasis{\sphinxupquote{LTI}}) \textendash{} Representacion del sistema

\end{description}\end{quote}

\end{fulllineitems}

\index{rutina\_root\_locus\_plot() (en el módulo rutinas\_analisis)@\spxentry{rutina\_root\_locus\_plot()}\spxextra{en el módulo rutinas\_analisis}}

\begin{fulllineitems}
\phantomsection\label{\detokenize{codigos/rutinas_analisis:rutinas_analisis.rutina_root_locus_plot}}\pysiglinewithargsret{\sphinxcode{\sphinxupquote{rutinas\_analisis.}}\sphinxbfcode{\sphinxupquote{rutina\_root\_locus\_plot}}}{\emph{self}, \emph{system}}{}
Funcion para obtener el lugar de la raices del sistema y su respectiva graficacion, la graficacion se realizo de forma interna en la libreria de control, para esto se moodifico la funcion root\_locus para poder enviar el axis y la figura
\begin{quote}\begin{description}
\item[{Parámetros}] \leavevmode
\sphinxstyleliteralstrong{\sphinxupquote{system}} (\sphinxstyleliteralemphasis{\sphinxupquote{LTI}}) \textendash{} Representacion del sistema

\end{description}\end{quote}

\end{fulllineitems}

\index{rutina\_step\_plot() (en el módulo rutinas\_analisis)@\spxentry{rutina\_step\_plot()}\spxextra{en el módulo rutinas\_analisis}}

\begin{fulllineitems}
\phantomsection\label{\detokenize{codigos/rutinas_analisis:rutinas_analisis.rutina_step_plot}}\pysiglinewithargsret{\sphinxcode{\sphinxupquote{rutinas\_analisis.}}\sphinxbfcode{\sphinxupquote{rutina\_step\_plot}}}{\emph{self}, \emph{system}, \emph{T}}{}
Funcion para obtener la respuesta escalon del sistema y su respectiva graficacion
\begin{quote}\begin{description}
\item[{Parámetros}] \leavevmode\begin{itemize}
\item {} 
\sphinxstyleliteralstrong{\sphinxupquote{system}} (\sphinxstyleliteralemphasis{\sphinxupquote{LTI}}) \textendash{} Representacion del sistema

\item {} 
\sphinxstyleliteralstrong{\sphinxupquote{T}} (\sphinxstyleliteralemphasis{\sphinxupquote{numpyArray}}) \textendash{} Vector de tiempo

\end{itemize}

\end{description}\end{quote}

\end{fulllineitems}

\index{rutina\_system\_info() (en el módulo rutinas\_analisis)@\spxentry{rutina\_system\_info()}\spxextra{en el módulo rutinas\_analisis}}

\begin{fulllineitems}
\phantomsection\label{\detokenize{codigos/rutinas_analisis:rutinas_analisis.rutina_system_info}}\pysiglinewithargsret{\sphinxcode{\sphinxupquote{rutinas\_analisis.}}\sphinxbfcode{\sphinxupquote{rutina\_system\_info}}}{\emph{self}, \emph{system}, \emph{T}, \emph{mag}, \emph{phase}, \emph{omega}}{}
Funcion para mostrar los resultados obtenidos de los calculos en un TextEdit
\begin{quote}\begin{description}
\item[{Parámetros}] \leavevmode\begin{itemize}
\item {} 
\sphinxstyleliteralstrong{\sphinxupquote{system}} (\sphinxstyleliteralemphasis{\sphinxupquote{LTI}}) \textendash{} Representacion del sistema

\item {} 
\sphinxstyleliteralstrong{\sphinxupquote{T}} (\sphinxstyleliteralemphasis{\sphinxupquote{numpyArray}}) \textendash{} Vector de tiempo

\item {} 
\sphinxstyleliteralstrong{\sphinxupquote{mag}} (\sphinxstyleliteralemphasis{\sphinxupquote{numpyArray}}) \textendash{} Magnitud de la respuesta en frecuencia

\item {} 
\sphinxstyleliteralstrong{\sphinxupquote{phase}} (\sphinxstyleliteralemphasis{\sphinxupquote{numpyArray}}) \textendash{} Fase de la respuesta en frecuencia

\item {} 
\sphinxstyleliteralstrong{\sphinxupquote{omega}} (\sphinxstyleliteralemphasis{\sphinxupquote{numpyArray}}) \textendash{} Frecuencias utilizadas para la respuesta en frecuencia

\end{itemize}

\end{description}\end{quote}

\end{fulllineitems}

\index{system\_creator\_ss() (en el módulo rutinas\_analisis)@\spxentry{system\_creator\_ss()}\spxextra{en el módulo rutinas\_analisis}}

\begin{fulllineitems}
\phantomsection\label{\detokenize{codigos/rutinas_analisis:rutinas_analisis.system_creator_ss}}\pysiglinewithargsret{\sphinxcode{\sphinxupquote{rutinas\_analisis.}}\sphinxbfcode{\sphinxupquote{system\_creator\_ss}}}{\emph{self}, \emph{A}, \emph{B}, \emph{C}, \emph{D}}{}
Funcion para la creacion del sistema a partir de la matriz de estado, matriz de entrada, matriz de salida y la matriz de transmision directa la ecuacion de espacio de estados
\begin{quote}\begin{description}
\item[{Parámetros}] \leavevmode\begin{itemize}
\item {} 
\sphinxstyleliteralstrong{\sphinxupquote{A}} (\sphinxstyleliteralemphasis{\sphinxupquote{list}}) \textendash{} Matriz de estados

\item {} 
\sphinxstyleliteralstrong{\sphinxupquote{B}} (\sphinxstyleliteralemphasis{\sphinxupquote{list}}) \textendash{} Matriz de entrada

\item {} 
\sphinxstyleliteralstrong{\sphinxupquote{C}} (\sphinxstyleliteralemphasis{\sphinxupquote{list}}) \textendash{} Matriz de salida

\item {} 
\sphinxstyleliteralstrong{\sphinxupquote{D}} (\sphinxstyleliteralemphasis{\sphinxupquote{list}}) \textendash{} Matriz de transmision directa

\end{itemize}

\end{description}\end{quote}

\end{fulllineitems}

\index{system\_creator\_tf() (en el módulo rutinas\_analisis)@\spxentry{system\_creator\_tf()}\spxextra{en el módulo rutinas\_analisis}}

\begin{fulllineitems}
\phantomsection\label{\detokenize{codigos/rutinas_analisis:rutinas_analisis.system_creator_tf}}\pysiglinewithargsret{\sphinxcode{\sphinxupquote{rutinas\_analisis.}}\sphinxbfcode{\sphinxupquote{system\_creator\_tf}}}{\emph{self}, \emph{numerador}, \emph{denominador}}{}
Funcion para la creacion del sistema a partir de los coeficientes del numerador y del denominador de la funcion de transferencia
\begin{quote}\begin{description}
\item[{Parámetros}] \leavevmode\begin{itemize}
\item {} 
\sphinxstyleliteralstrong{\sphinxupquote{numerador}} (\sphinxstyleliteralemphasis{\sphinxupquote{list}}) \textendash{} Coeficientes del numerador

\item {} 
\sphinxstyleliteralstrong{\sphinxupquote{denominador}} (\sphinxstyleliteralemphasis{\sphinxupquote{list}}) \textendash{} Coeficientes del denominador

\end{itemize}

\end{description}\end{quote}

\end{fulllineitems}

\phantomsection\label{\detokenize{codigos/analisisHandler:module-analisisHandler}}\index{analisisHandler (módulo)@\spxentry{analisisHandler}\spxextra{módulo}}
Archivo para el manejo de la funcion de analisis de sistemas de control, sirve de intermediario entre la interfaz grafica y las rutinas de analisis
\index{AnalisisHandler() (en el módulo analisisHandler)@\spxentry{AnalisisHandler()}\spxextra{en el módulo analisisHandler}}

\begin{fulllineitems}
\phantomsection\label{\detokenize{codigos/analisisHandler:analisisHandler.AnalisisHandler}}\pysiglinewithargsret{\sphinxcode{\sphinxupquote{analisisHandler.}}\sphinxbfcode{\sphinxupquote{AnalisisHandler}}}{\emph{self}}{}
Funcion principal para el manejo de la funcionalida de analisis de sistemas de control, se crean las señales a ejecutar cuando se interactua con los widgets incluyendo las validaciones de entradas

\end{fulllineitems}

\index{analisis\_bool\_discreto() (en el módulo analisisHandler)@\spxentry{analisis\_bool\_discreto()}\spxextra{en el módulo analisisHandler}}

\begin{fulllineitems}
\phantomsection\label{\detokenize{codigos/analisisHandler:analisisHandler.analisis_bool_discreto}}\pysiglinewithargsret{\sphinxcode{\sphinxupquote{analisisHandler.}}\sphinxbfcode{\sphinxupquote{analisis\_bool\_discreto}}}{\emph{self}}{}
Funcion para habilitar y deshabilitar el periodo de muestreo

\end{fulllineitems}

\index{analisis\_stacked\_to\_ss() (en el módulo analisisHandler)@\spxentry{analisis\_stacked\_to\_ss()}\spxextra{en el módulo analisisHandler}}

\begin{fulllineitems}
\phantomsection\label{\detokenize{codigos/analisisHandler:analisisHandler.analisis_stacked_to_ss}}\pysiglinewithargsret{\sphinxcode{\sphinxupquote{analisisHandler.}}\sphinxbfcode{\sphinxupquote{analisis\_stacked\_to\_ss}}}{\emph{self}}{}
Funcion para cambiar de funcion de transferencia a ecuacion de espacio de estados

\end{fulllineitems}

\index{analisis\_stacked\_to\_tf() (en el módulo analisisHandler)@\spxentry{analisis\_stacked\_to\_tf()}\spxextra{en el módulo analisisHandler}}

\begin{fulllineitems}
\phantomsection\label{\detokenize{codigos/analisisHandler:analisisHandler.analisis_stacked_to_tf}}\pysiglinewithargsret{\sphinxcode{\sphinxupquote{analisisHandler.}}\sphinxbfcode{\sphinxupquote{analisis\_stacked\_to\_tf}}}{\emph{self}}{}
Funcion para cambiar de ecuacion de espacio de estados a funcion de transferencia

\end{fulllineitems}

\index{calcular\_analisis() (en el módulo analisisHandler)@\spxentry{calcular\_analisis()}\spxextra{en el módulo analisisHandler}}

\begin{fulllineitems}
\phantomsection\label{\detokenize{codigos/analisisHandler:analisisHandler.calcular_analisis}}\pysiglinewithargsret{\sphinxcode{\sphinxupquote{analisisHandler.}}\sphinxbfcode{\sphinxupquote{calcular\_analisis}}}{\emph{self}}{}
Funcion para realizar el los calculos necesarios para la funcionalidad de analisis de sistemas de control, el llamado a esta funcion se realiza por medio del boton calcular

\end{fulllineitems}

\index{ssA\_validator() (en el módulo analisisHandler)@\spxentry{ssA\_validator()}\spxextra{en el módulo analisisHandler}}

\begin{fulllineitems}
\phantomsection\label{\detokenize{codigos/analisisHandler:analisisHandler.ssA_validator}}\pysiglinewithargsret{\sphinxcode{\sphinxupquote{analisisHandler.}}\sphinxbfcode{\sphinxupquote{ssA\_validator}}}{\emph{self}}{}
Validacion de la matriz de estados de la ecuacion de espacio de estados

\end{fulllineitems}

\index{ssB\_validator() (en el módulo analisisHandler)@\spxentry{ssB\_validator()}\spxextra{en el módulo analisisHandler}}

\begin{fulllineitems}
\phantomsection\label{\detokenize{codigos/analisisHandler:analisisHandler.ssB_validator}}\pysiglinewithargsret{\sphinxcode{\sphinxupquote{analisisHandler.}}\sphinxbfcode{\sphinxupquote{ssB\_validator}}}{\emph{self}}{}
Validacion de la matriz de entrada de la ecuacion de espacio de estados

\end{fulllineitems}

\index{ssC\_validator() (en el módulo analisisHandler)@\spxentry{ssC\_validator()}\spxextra{en el módulo analisisHandler}}

\begin{fulllineitems}
\phantomsection\label{\detokenize{codigos/analisisHandler:analisisHandler.ssC_validator}}\pysiglinewithargsret{\sphinxcode{\sphinxupquote{analisisHandler.}}\sphinxbfcode{\sphinxupquote{ssC\_validator}}}{\emph{self}}{}
Validacion de la matriz de salida de la ecuacion de espacio de estados

\end{fulllineitems}

\index{ssD\_validator() (en el módulo analisisHandler)@\spxentry{ssD\_validator()}\spxextra{en el módulo analisisHandler}}

\begin{fulllineitems}
\phantomsection\label{\detokenize{codigos/analisisHandler:analisisHandler.ssD_validator}}\pysiglinewithargsret{\sphinxcode{\sphinxupquote{analisisHandler.}}\sphinxbfcode{\sphinxupquote{ssD\_validator}}}{\emph{self}}{}
Validacion de la matriz de transmision directa de la ecuacion de espacio de estados

\end{fulllineitems}

\index{ssdelay\_validator() (en el módulo analisisHandler)@\spxentry{ssdelay\_validator()}\spxextra{en el módulo analisisHandler}}

\begin{fulllineitems}
\phantomsection\label{\detokenize{codigos/analisisHandler:analisisHandler.ssdelay_validator}}\pysiglinewithargsret{\sphinxcode{\sphinxupquote{analisisHandler.}}\sphinxbfcode{\sphinxupquote{ssdelay\_validator}}}{\emph{self}}{}
Validacion del delay de la ecuacion de espacio de estados

\end{fulllineitems}

\index{ssperiodo\_validator() (en el módulo analisisHandler)@\spxentry{ssperiodo\_validator()}\spxextra{en el módulo analisisHandler}}

\begin{fulllineitems}
\phantomsection\label{\detokenize{codigos/analisisHandler:analisisHandler.ssperiodo_validator}}\pysiglinewithargsret{\sphinxcode{\sphinxupquote{analisisHandler.}}\sphinxbfcode{\sphinxupquote{ssperiodo\_validator}}}{\emph{self}}{}
Validacion del periodo de muestreo de la ecuacion de espacio de estados

\end{fulllineitems}

\index{tfdelay\_validator() (en el módulo analisisHandler)@\spxentry{tfdelay\_validator()}\spxextra{en el módulo analisisHandler}}

\begin{fulllineitems}
\phantomsection\label{\detokenize{codigos/analisisHandler:analisisHandler.tfdelay_validator}}\pysiglinewithargsret{\sphinxcode{\sphinxupquote{analisisHandler.}}\sphinxbfcode{\sphinxupquote{tfdelay\_validator}}}{\emph{self}}{}
Validacion del delay de la funcion de transferencia

\end{fulllineitems}

\index{tfdem\_validator() (en el módulo analisisHandler)@\spxentry{tfdem\_validator()}\spxextra{en el módulo analisisHandler}}

\begin{fulllineitems}
\phantomsection\label{\detokenize{codigos/analisisHandler:analisisHandler.tfdem_validator}}\pysiglinewithargsret{\sphinxcode{\sphinxupquote{analisisHandler.}}\sphinxbfcode{\sphinxupquote{tfdem\_validator}}}{\emph{self}}{}
Validacion del denominador de la funcion de transferencia

\end{fulllineitems}

\index{tfnum\_validator() (en el módulo analisisHandler)@\spxentry{tfnum\_validator()}\spxextra{en el módulo analisisHandler}}

\begin{fulllineitems}
\phantomsection\label{\detokenize{codigos/analisisHandler:analisisHandler.tfnum_validator}}\pysiglinewithargsret{\sphinxcode{\sphinxupquote{analisisHandler.}}\sphinxbfcode{\sphinxupquote{tfnum\_validator}}}{\emph{self}}{}
Validacion del numerador de la funcion de transferencia

\end{fulllineitems}

\index{tfperiodo\_validator() (en el módulo analisisHandler)@\spxentry{tfperiodo\_validator()}\spxextra{en el módulo analisisHandler}}

\begin{fulllineitems}
\phantomsection\label{\detokenize{codigos/analisisHandler:analisisHandler.tfperiodo_validator}}\pysiglinewithargsret{\sphinxcode{\sphinxupquote{analisisHandler.}}\sphinxbfcode{\sphinxupquote{tfperiodo\_validator}}}{\emph{self}}{}
Validacion del periodo de muestreo de la funcion de transferencia

\end{fulllineitems}



\section{Archivo Handler para la función de entonación de controladores PID}
\label{\detokenize{codigos/TuningHandler:archivo-handler-para-la-funcion-de-entonacion-de-controladores-pid}}\label{\detokenize{codigos/TuningHandler::doc}}

\subsection{Archivo de rutinas para la función de entonación de PID}
\label{\detokenize{codigos/rutinas_PID:archivo-de-rutinas-para-la-funcion-de-entonacion-de-pid}}\label{\detokenize{codigos/rutinas_PID::doc}}\phantomsection\label{\detokenize{codigos/rutinas_PID:module-rutinas_PID}}\index{rutinas\_PID (módulo)@\spxentry{rutinas\_PID}\spxextra{módulo}}
Archivo que contiene todas las rutinas necesarias para la funcionalidad de tunning de PID
\index{auto\_tuning\_method() (en el módulo rutinas\_PID)@\spxentry{auto\_tuning\_method()}\spxextra{en el módulo rutinas\_PID}}

\begin{fulllineitems}
\phantomsection\label{\detokenize{codigos/rutinas_PID:rutinas_PID.auto_tuning_method}}\pysiglinewithargsret{\sphinxcode{\sphinxupquote{rutinas\_PID.}}\sphinxbfcode{\sphinxupquote{auto\_tuning\_method}}}{\emph{self}, \emph{k\_proceso}, \emph{tau}, \emph{alpha}, \emph{metodo}}{}
Funcion para obtener las ganancias del controlador PID a partir de los parametros del modelo de primer orden obtenidos de una respuesta escalon, las formulas son las dadas por Ziegler\sphinxhyphen{}Nichols y Cohen\sphinxhyphen{}Coon para una respuesta escalon en lazo abierto
\begin{quote}\begin{description}
\item[{Parámetros}] \leavevmode\begin{itemize}
\item {} 
\sphinxstyleliteralstrong{\sphinxupquote{k\_proceso}} (\sphinxstyleliteralemphasis{\sphinxupquote{float}}) \textendash{} Ganancia del proceso

\item {} 
\sphinxstyleliteralstrong{\sphinxupquote{tau}} (\sphinxstyleliteralemphasis{\sphinxupquote{float}}) \textendash{} Constante de tiempo del proceso

\item {} 
\sphinxstyleliteralstrong{\sphinxupquote{alpha}} (\sphinxstyleliteralemphasis{\sphinxupquote{float}}) \textendash{} Tiempo muerto o delay del proceso

\item {} 
\sphinxstyleliteralstrong{\sphinxupquote{metodo}} (\sphinxstyleliteralemphasis{\sphinxupquote{str}}) \textendash{} Metodo a utilizar

\end{itemize}

\end{description}\end{quote}

\end{fulllineitems}

\index{model\_method() (en el módulo rutinas\_PID)@\spxentry{model\_method()}\spxextra{en el módulo rutinas\_PID}}

\begin{fulllineitems}
\phantomsection\label{\detokenize{codigos/rutinas_PID:rutinas_PID.model_method}}\pysiglinewithargsret{\sphinxcode{\sphinxupquote{rutinas\_PID.}}\sphinxbfcode{\sphinxupquote{model\_method}}}{\emph{self}, \emph{t}, \emph{y}, \emph{dc\_gain}}{}
Funcion para obtener los parametros del modelo de primer orden de un sistema a partir de su respuesta escalon
\begin{quote}\begin{description}
\item[{Parámetros}] \leavevmode\begin{itemize}
\item {} 
\sphinxstyleliteralstrong{\sphinxupquote{t}} (\sphinxstyleliteralemphasis{\sphinxupquote{numpyArray}}) \textendash{} Vector de tiempo

\item {} 
\sphinxstyleliteralstrong{\sphinxupquote{y}} (\sphinxstyleliteralemphasis{\sphinxupquote{numpyArray}}) \textendash{} Vector de respuesta

\item {} 
\sphinxstyleliteralstrong{\sphinxupquote{dc\_gain}} (\sphinxstyleliteralemphasis{\sphinxupquote{float}}) \textendash{} Ganancia DC del sistema

\end{itemize}

\end{description}\end{quote}

\end{fulllineitems}

\index{rutina\_step\_plot() (en el módulo rutinas\_PID)@\spxentry{rutina\_step\_plot()}\spxextra{en el módulo rutinas\_PID}}

\begin{fulllineitems}
\phantomsection\label{\detokenize{codigos/rutinas_PID:rutinas_PID.rutina_step_plot}}\pysiglinewithargsret{\sphinxcode{\sphinxupquote{rutinas\_PID.}}\sphinxbfcode{\sphinxupquote{rutina\_step\_plot}}}{\emph{self}, \emph{system}, \emph{T}, \emph{kp}, \emph{ki}, \emph{kd}}{}
Funcion para obtener la respuesta escalon del sistema en lazo cerrado en combinacion con un controlador PID y su respectiva graficacion
\begin{quote}\begin{description}
\item[{Parámetros}] \leavevmode\begin{itemize}
\item {} 
\sphinxstyleliteralstrong{\sphinxupquote{system}} (\sphinxstyleliteralemphasis{\sphinxupquote{LTI}}) \textendash{} Representacion del sistema

\item {} 
\sphinxstyleliteralstrong{\sphinxupquote{T}} (\sphinxstyleliteralemphasis{\sphinxupquote{numpyArray}}) \textendash{} Vector de tiempo

\item {} 
\sphinxstyleliteralstrong{\sphinxupquote{kp}} (\sphinxstyleliteralemphasis{\sphinxupquote{float}}) \textendash{} Ganancia proporcional

\item {} 
\sphinxstyleliteralstrong{\sphinxupquote{ki}} (\sphinxstyleliteralemphasis{\sphinxupquote{float}}) \textendash{} Ganancia integral

\item {} 
\sphinxstyleliteralstrong{\sphinxupquote{kd}} (\sphinxstyleliteralemphasis{\sphinxupquote{float}}) \textendash{} Ganancia derivativa

\end{itemize}

\end{description}\end{quote}

\end{fulllineitems}

\index{rutina\_system\_info() (en el módulo rutinas\_PID)@\spxentry{rutina\_system\_info()}\spxextra{en el módulo rutinas\_PID}}

\begin{fulllineitems}
\phantomsection\label{\detokenize{codigos/rutinas_PID:rutinas_PID.rutina_system_info}}\pysiglinewithargsret{\sphinxcode{\sphinxupquote{rutinas\_PID.}}\sphinxbfcode{\sphinxupquote{rutina\_system\_info}}}{\emph{self}, \emph{system}, \emph{T}, \emph{y}, \emph{kp=0}, \emph{ki=0}, \emph{kd=0}, \emph{autotuning=False}}{}
Funcion para mostrar los resultados obtenidos de los calculos en un TextEdit
\begin{quote}\begin{description}
\item[{Parámetros}] \leavevmode\begin{itemize}
\item {} 
\sphinxstyleliteralstrong{\sphinxupquote{system}} (\sphinxstyleliteralemphasis{\sphinxupquote{LTI}}) \textendash{} Representacion del sistema

\item {} 
\sphinxstyleliteralstrong{\sphinxupquote{T}} (\sphinxstyleliteralemphasis{\sphinxupquote{numpyArray}}) \textendash{} Vector de tiempo

\item {} 
\sphinxstyleliteralstrong{\sphinxupquote{y}} (\sphinxstyleliteralemphasis{\sphinxupquote{numpyArray}}) \textendash{} Vector de respuesta

\item {} 
\sphinxstyleliteralstrong{\sphinxupquote{kp}} (\sphinxstyleliteralemphasis{\sphinxupquote{float}}\sphinxstyleliteralemphasis{\sphinxupquote{, }}\sphinxstyleliteralemphasis{\sphinxupquote{optional}}) \textendash{} Ganancia proporcional, defaults to 0

\item {} 
\sphinxstyleliteralstrong{\sphinxupquote{ki}} (\sphinxstyleliteralemphasis{\sphinxupquote{float}}\sphinxstyleliteralemphasis{\sphinxupquote{, }}\sphinxstyleliteralemphasis{\sphinxupquote{optional}}) \textendash{} Ganancia integral, defaults to 0

\item {} 
\sphinxstyleliteralstrong{\sphinxupquote{kd}} (\sphinxstyleliteralemphasis{\sphinxupquote{float}}\sphinxstyleliteralemphasis{\sphinxupquote{, }}\sphinxstyleliteralemphasis{\sphinxupquote{optional}}) \textendash{} Ganancia derivativa, defaults to 0

\item {} 
\sphinxstyleliteralstrong{\sphinxupquote{autotuning}} (\sphinxstyleliteralemphasis{\sphinxupquote{bool}}\sphinxstyleliteralemphasis{\sphinxupquote{, }}\sphinxstyleliteralemphasis{\sphinxupquote{optional}}) \textendash{} Bandera para señar si es o no una operacion con auto tunning, defaults to False

\end{itemize}

\end{description}\end{quote}

\end{fulllineitems}

\index{system\_creator\_ss() (en el módulo rutinas\_PID)@\spxentry{system\_creator\_ss()}\spxextra{en el módulo rutinas\_PID}}

\begin{fulllineitems}
\phantomsection\label{\detokenize{codigos/rutinas_PID:rutinas_PID.system_creator_ss}}\pysiglinewithargsret{\sphinxcode{\sphinxupquote{rutinas\_PID.}}\sphinxbfcode{\sphinxupquote{system\_creator\_ss}}}{\emph{self}, \emph{A}, \emph{B}, \emph{C}, \emph{D}}{}~\begin{description}
\item[{Funcion para la creacion del sistema a partir de la matriz de estado, matriz de entrada, matriz de salida y}] \leavevmode
la matriz de transmision directa la ecuacion de espacio de estados

\end{description}
\begin{quote}\begin{description}
\item[{Parámetros}] \leavevmode\begin{itemize}
\item {} 
\sphinxstyleliteralstrong{\sphinxupquote{A}} (\sphinxstyleliteralemphasis{\sphinxupquote{list}}) \textendash{} Matriz de estados

\item {} 
\sphinxstyleliteralstrong{\sphinxupquote{B}} (\sphinxstyleliteralemphasis{\sphinxupquote{list}}) \textendash{} Matriz de entrada

\item {} 
\sphinxstyleliteralstrong{\sphinxupquote{C}} (\sphinxstyleliteralemphasis{\sphinxupquote{list}}) \textendash{} Matriz de salida

\item {} 
\sphinxstyleliteralstrong{\sphinxupquote{D}} (\sphinxstyleliteralemphasis{\sphinxupquote{list}}) \textendash{} Matriz de transmision directa

\end{itemize}

\end{description}\end{quote}

\end{fulllineitems}

\index{system\_creator\_ss\_tuning() (en el módulo rutinas\_PID)@\spxentry{system\_creator\_ss\_tuning()}\spxextra{en el módulo rutinas\_PID}}

\begin{fulllineitems}
\phantomsection\label{\detokenize{codigos/rutinas_PID:rutinas_PID.system_creator_ss_tuning}}\pysiglinewithargsret{\sphinxcode{\sphinxupquote{rutinas\_PID.}}\sphinxbfcode{\sphinxupquote{system\_creator\_ss\_tuning}}}{\emph{self}, \emph{A}, \emph{B}, \emph{C}, \emph{D}}{}
Funcion para la creacion del sistema a partir de la matriz de estado, matriz de entrada, matriz de salida y la matriz de transmision directa la ecuacion de espacio de estados, adicionalmente se realiza el auto tuning utilizando el metodo escojido por le usuario
\begin{quote}\begin{description}
\item[{Parámetros}] \leavevmode\begin{itemize}
\item {} 
\sphinxstyleliteralstrong{\sphinxupquote{A}} (\sphinxstyleliteralemphasis{\sphinxupquote{list}}) \textendash{} Matriz de estados

\item {} 
\sphinxstyleliteralstrong{\sphinxupquote{B}} (\sphinxstyleliteralemphasis{\sphinxupquote{list}}) \textendash{} Matriz de entrada

\item {} 
\sphinxstyleliteralstrong{\sphinxupquote{C}} (\sphinxstyleliteralemphasis{\sphinxupquote{list}}) \textendash{} Matriz de salida

\item {} 
\sphinxstyleliteralstrong{\sphinxupquote{D}} (\sphinxstyleliteralemphasis{\sphinxupquote{list}}) \textendash{} Matriz de transmision directa

\end{itemize}

\end{description}\end{quote}

\end{fulllineitems}

\index{system\_creator\_tf() (en el módulo rutinas\_PID)@\spxentry{system\_creator\_tf()}\spxextra{en el módulo rutinas\_PID}}

\begin{fulllineitems}
\phantomsection\label{\detokenize{codigos/rutinas_PID:rutinas_PID.system_creator_tf}}\pysiglinewithargsret{\sphinxcode{\sphinxupquote{rutinas\_PID.}}\sphinxbfcode{\sphinxupquote{system\_creator\_tf}}}{\emph{self}, \emph{numerador}, \emph{denominador}}{}
Funcion para la creacion del sistema a partir de los coeficientes del numerador y del denominador de la funcion de transferencia
\begin{quote}\begin{description}
\item[{Parámetros}] \leavevmode\begin{itemize}
\item {} 
\sphinxstyleliteralstrong{\sphinxupquote{numerador}} (\sphinxstyleliteralemphasis{\sphinxupquote{list}}) \textendash{} Coeficientes del numerador

\item {} 
\sphinxstyleliteralstrong{\sphinxupquote{denominador}} (\sphinxstyleliteralemphasis{\sphinxupquote{list}}) \textendash{} Coeficientes del denominador

\end{itemize}

\end{description}\end{quote}

\end{fulllineitems}

\index{system\_creator\_tf\_tuning() (en el módulo rutinas\_PID)@\spxentry{system\_creator\_tf\_tuning()}\spxextra{en el módulo rutinas\_PID}}

\begin{fulllineitems}
\phantomsection\label{\detokenize{codigos/rutinas_PID:rutinas_PID.system_creator_tf_tuning}}\pysiglinewithargsret{\sphinxcode{\sphinxupquote{rutinas\_PID.}}\sphinxbfcode{\sphinxupquote{system\_creator\_tf\_tuning}}}{\emph{self}, \emph{numerador}, \emph{denominador}}{}
Funcion para la creacion del sistema a partir de los coeficientes del numerador y del denominador de la funcion de transferencia, adicionalmente se realiza el auto tuning utilizando el metodo escojido por le usuario
\begin{quote}\begin{description}
\item[{Parámetros}] \leavevmode\begin{itemize}
\item {} 
\sphinxstyleliteralstrong{\sphinxupquote{A}} (\sphinxstyleliteralemphasis{\sphinxupquote{list}}) \textendash{} Matriz de estados

\item {} 
\sphinxstyleliteralstrong{\sphinxupquote{B}} (\sphinxstyleliteralemphasis{\sphinxupquote{list}}) \textendash{} Matriz de entrada

\item {} 
\sphinxstyleliteralstrong{\sphinxupquote{C}} (\sphinxstyleliteralemphasis{\sphinxupquote{list}}) \textendash{} Matriz de salida

\item {} 
\sphinxstyleliteralstrong{\sphinxupquote{D}} (\sphinxstyleliteralemphasis{\sphinxupquote{list}}) \textendash{} Matriz de transmision directa

\end{itemize}

\end{description}\end{quote}

\end{fulllineitems}



\subsection{Archivo de rutinas para la función de entonación de PID utilizando data de un CSV}
\label{\detokenize{codigos/rutinas_CSV:module-rutinas_CSV}}\label{\detokenize{codigos/rutinas_CSV:archivo-de-rutinas-para-la-funcion-de-entonacion-de-pid-utilizando-data-de-un-csv}}\label{\detokenize{codigos/rutinas_CSV::doc}}\index{rutinas\_CSV (módulo)@\spxentry{rutinas\_CSV}\spxextra{módulo}}
Archivo que contiene todas las rutinas necesarias para la funcionalidad de identificacion de modelo y tunning con csv
\index{actualizar\_Datos() (en el módulo rutinas\_CSV)@\spxentry{actualizar\_Datos()}\spxextra{en el módulo rutinas\_CSV}}

\begin{fulllineitems}
\phantomsection\label{\detokenize{codigos/rutinas_CSV:rutinas_CSV.actualizar_Datos}}\pysiglinewithargsret{\sphinxcode{\sphinxupquote{rutinas\_CSV.}}\sphinxbfcode{\sphinxupquote{actualizar\_Datos}}}{\emph{self}, \emph{Kc}, \emph{t0}, \emph{t1}, \emph{t2}, \emph{kp}, \emph{ki}, \emph{kd}}{}
Funcion para mostrar los resultados obtenidos del modelo en un TextEdit
\begin{quote}\begin{description}
\item[{Parámetros}] \leavevmode\begin{itemize}
\item {} 
\sphinxstyleliteralstrong{\sphinxupquote{Kc}} (\sphinxstyleliteralemphasis{\sphinxupquote{float}}) \textendash{} Ganancia del proceso

\item {} 
\sphinxstyleliteralstrong{\sphinxupquote{t0}} (\sphinxstyleliteralemphasis{\sphinxupquote{float}}) \textendash{} Tiempo del inicio del escalon

\item {} 
\sphinxstyleliteralstrong{\sphinxupquote{t1}} (\sphinxstyleliteralemphasis{\sphinxupquote{float}}) \textendash{} Tiempo del inicio de la respuesta del proceso ante el escalon

\item {} 
\sphinxstyleliteralstrong{\sphinxupquote{t2}} (\sphinxstyleliteralemphasis{\sphinxupquote{float}}) \textendash{} Tiempo en el que el proceso alcanza el 63\% de su valor final respecto al cambio

\item {} 
\sphinxstyleliteralstrong{\sphinxupquote{kp}} (\sphinxstyleliteralemphasis{\sphinxupquote{float}}) \textendash{} Ganancia proporcional

\item {} 
\sphinxstyleliteralstrong{\sphinxupquote{ki}} (\sphinxstyleliteralemphasis{\sphinxupquote{float}}) \textendash{} Ganancia integral

\item {} 
\sphinxstyleliteralstrong{\sphinxupquote{kd}} (\sphinxstyleliteralemphasis{\sphinxupquote{float}}) \textendash{} Ganancia derivativa

\end{itemize}

\end{description}\end{quote}

\end{fulllineitems}

\index{auto\_tuning\_method\_csv() (en el módulo rutinas\_CSV)@\spxentry{auto\_tuning\_method\_csv()}\spxextra{en el módulo rutinas\_CSV}}

\begin{fulllineitems}
\phantomsection\label{\detokenize{codigos/rutinas_CSV:rutinas_CSV.auto_tuning_method_csv}}\pysiglinewithargsret{\sphinxcode{\sphinxupquote{rutinas\_CSV.}}\sphinxbfcode{\sphinxupquote{auto\_tuning\_method\_csv}}}{\emph{self}, \emph{k\_proceso}, \emph{tau}, \emph{alpha}, \emph{metodo}}{}
Funcion para obtener las ganancias del controlador PID a partir de los parametros del modelo de primer orden obtenidos de una respuesta escalon, las formulas son las dadas por Ziegler\sphinxhyphen{}Nichols y Cohen\sphinxhyphen{}Coon para una respuesta escalon en lazo abierto
\begin{quote}\begin{description}
\item[{Parámetros}] \leavevmode\begin{itemize}
\item {} 
\sphinxstyleliteralstrong{\sphinxupquote{k\_proceso}} (\sphinxstyleliteralemphasis{\sphinxupquote{float}}) \textendash{} Ganancia del proceso

\item {} 
\sphinxstyleliteralstrong{\sphinxupquote{tau}} (\sphinxstyleliteralemphasis{\sphinxupquote{float}}) \textendash{} Constante de tiempo del proceso

\item {} 
\sphinxstyleliteralstrong{\sphinxupquote{alpha}} (\sphinxstyleliteralemphasis{\sphinxupquote{float}}) \textendash{} Tiempo muerto o delay del proceso

\item {} 
\sphinxstyleliteralstrong{\sphinxupquote{metodo}} (\sphinxstyleliteralemphasis{\sphinxupquote{str}}) \textendash{} Metodo a utilizar

\end{itemize}

\end{description}\end{quote}

\end{fulllineitems}

\index{calcular\_modelo() (en el módulo rutinas\_CSV)@\spxentry{calcular\_modelo()}\spxextra{en el módulo rutinas\_CSV}}

\begin{fulllineitems}
\phantomsection\label{\detokenize{codigos/rutinas_CSV:rutinas_CSV.calcular_modelo}}\pysiglinewithargsret{\sphinxcode{\sphinxupquote{rutinas\_CSV.}}\sphinxbfcode{\sphinxupquote{calcular\_modelo}}}{\emph{self}, \emph{dict\_data}, \emph{indexTime}, \emph{indexVp}, \emph{indexEFC}, \emph{MinVP}, \emph{MaxVP}, \emph{MinEFC}, \emph{MaxEFC}}{}
Fucion para calcular los parametros del modelo de primer orden
\begin{quote}\begin{description}
\item[{Parámetros}] \leavevmode\begin{itemize}
\item {} 
\sphinxstyleliteralstrong{\sphinxupquote{dict\_data}} (\sphinxstyleliteralemphasis{\sphinxupquote{dict}}) \textendash{} Diccionario con la data procesada del csv

\item {} 
\sphinxstyleliteralstrong{\sphinxupquote{indexTime}} (\sphinxstyleliteralemphasis{\sphinxupquote{int}}) \textendash{} Indice que identifica al tiempo

\item {} 
\sphinxstyleliteralstrong{\sphinxupquote{indexVp}} (\sphinxstyleliteralemphasis{\sphinxupquote{int}}) \textendash{} Indice que identifica a Vp

\item {} 
\sphinxstyleliteralstrong{\sphinxupquote{indexEFC}} (\sphinxstyleliteralemphasis{\sphinxupquote{int}}) \textendash{} Indice que identifica al EFC

\item {} 
\sphinxstyleliteralstrong{\sphinxupquote{MinVP}} (\sphinxstyleliteralemphasis{\sphinxupquote{float}}) \textendash{} Limite inferior de Vp

\item {} 
\sphinxstyleliteralstrong{\sphinxupquote{MaxVP}} (\sphinxstyleliteralemphasis{\sphinxupquote{float}}) \textendash{} Limite superior de Vp

\item {} 
\sphinxstyleliteralstrong{\sphinxupquote{MinEFC}} (\sphinxstyleliteralemphasis{\sphinxupquote{float}}) \textendash{} Limite inferior de EFC

\item {} 
\sphinxstyleliteralstrong{\sphinxupquote{MaxEFC}} (\sphinxstyleliteralemphasis{\sphinxupquote{float}}) \textendash{} Limite superior de EFC

\end{itemize}

\end{description}\end{quote}

\end{fulllineitems}

\index{calculos\_manual() (en el módulo rutinas\_CSV)@\spxentry{calculos\_manual()}\spxextra{en el módulo rutinas\_CSV}}

\begin{fulllineitems}
\phantomsection\label{\detokenize{codigos/rutinas_CSV:rutinas_CSV.calculos_manual}}\pysiglinewithargsret{\sphinxcode{\sphinxupquote{rutinas\_CSV.}}\sphinxbfcode{\sphinxupquote{calculos\_manual}}}{\emph{self}, \emph{GraphObjets}, \emph{Kc}, \emph{t0}, \emph{t1}, \emph{t2}, \emph{slop}, \emph{y1}}{}
Funcion para recalcular el controlador PID a partir de los datos del modelo de primer orden con el nueto tiempo t1, ademas, se grafica la data del csv junto con algunos parametros de la identificacion del modelo y la nueva recta
\begin{quote}\begin{description}
\item[{Parámetros}] \leavevmode\begin{itemize}
\item {} 
\sphinxstyleliteralstrong{\sphinxupquote{GraphObjets}} (\sphinxstyleliteralemphasis{\sphinxupquote{list}}) \textendash{} Lista de objetos de graficacion

\item {} 
\sphinxstyleliteralstrong{\sphinxupquote{Kc}} (\sphinxstyleliteralemphasis{\sphinxupquote{float}}) \textendash{} Ganancia del proceso

\item {} 
\sphinxstyleliteralstrong{\sphinxupquote{t0}} (\sphinxstyleliteralemphasis{\sphinxupquote{float}}) \textendash{} Tiempo del inicio del escalon

\item {} 
\sphinxstyleliteralstrong{\sphinxupquote{t1}} (\sphinxstyleliteralemphasis{\sphinxupquote{float}}) \textendash{} Tiempo del inicio de la respuesta del proceso ante el escalon

\item {} 
\sphinxstyleliteralstrong{\sphinxupquote{t2}} (\sphinxstyleliteralemphasis{\sphinxupquote{float}}) \textendash{} Tiempo en el que el proceso alcanza el 63\% de su valor final respecto al cambio

\item {} 
\sphinxstyleliteralstrong{\sphinxupquote{slop}} (\sphinxstyleliteralemphasis{\sphinxupquote{float}}) \textendash{} Pendiente de la recta de identificacion

\item {} 
\sphinxstyleliteralstrong{\sphinxupquote{y1}} (\sphinxstyleliteralemphasis{\sphinxupquote{float}}) \textendash{} Punto y1 de la recta de identifiacion, en este punto se encuentra el mayor cambio respecto al tiempo

\end{itemize}

\end{description}\end{quote}

\end{fulllineitems}

\index{entonar\_y\_graficar() (en el módulo rutinas\_CSV)@\spxentry{entonar\_y\_graficar()}\spxextra{en el módulo rutinas\_CSV}}

\begin{fulllineitems}
\phantomsection\label{\detokenize{codigos/rutinas_CSV:rutinas_CSV.entonar_y_graficar}}\pysiglinewithargsret{\sphinxcode{\sphinxupquote{rutinas\_CSV.}}\sphinxbfcode{\sphinxupquote{entonar\_y\_graficar}}}{\emph{self}, \emph{dict\_data}, \emph{Kc}, \emph{tau}, \emph{y1}, \emph{y2}, \emph{t0}, \emph{t1}, \emph{t2}}{}
Funcion para calcular el controlador PID a partir de los datos del modelo de primer orden, ademas, se grafica la data del csv junto con algunos parametros de la identificacion del modelo
\begin{quote}\begin{description}
\item[{Parámetros}] \leavevmode\begin{itemize}
\item {} 
\sphinxstyleliteralstrong{\sphinxupquote{dict\_data}} (\sphinxstyleliteralemphasis{\sphinxupquote{dict}}) \textendash{} Diccionario con la data procesada del csv

\item {} 
\sphinxstyleliteralstrong{\sphinxupquote{Kc}} (\sphinxstyleliteralemphasis{\sphinxupquote{float}}) \textendash{} Ganancia del proceso

\item {} 
\sphinxstyleliteralstrong{\sphinxupquote{tau}} (\sphinxstyleliteralemphasis{\sphinxupquote{float}}) \textendash{} Constante de tiempo del proceso

\item {} 
\sphinxstyleliteralstrong{\sphinxupquote{y1}} (\sphinxstyleliteralemphasis{\sphinxupquote{float}}) \textendash{} Punto y1 de la recta de identifiacion, en este punto se encuentra el mayor cambio respecto al tiempo

\item {} 
\sphinxstyleliteralstrong{\sphinxupquote{y2}} (\sphinxstyleliteralemphasis{\sphinxupquote{float}}) \textendash{} Punto y2 de la recta de identificacion

\item {} 
\sphinxstyleliteralstrong{\sphinxupquote{t0}} (\sphinxstyleliteralemphasis{\sphinxupquote{float}}) \textendash{} Tiempo del inicio del escalon

\item {} 
\sphinxstyleliteralstrong{\sphinxupquote{t1}} (\sphinxstyleliteralemphasis{\sphinxupquote{float}}) \textendash{} Tiempo del inicio de la respuesta del proceso ante el escalon

\item {} 
\sphinxstyleliteralstrong{\sphinxupquote{t2}} (\sphinxstyleliteralemphasis{\sphinxupquote{float}}) \textendash{} Tiempo en el que el proceso alcanza el 63\% de su valor final respecto al cambio

\end{itemize}

\end{description}\end{quote}

\end{fulllineitems}

\index{procesar\_csv() (en el módulo rutinas\_CSV)@\spxentry{procesar\_csv()}\spxextra{en el módulo rutinas\_CSV}}

\begin{fulllineitems}
\phantomsection\label{\detokenize{codigos/rutinas_CSV:rutinas_CSV.procesar_csv}}\pysiglinewithargsret{\sphinxcode{\sphinxupquote{rutinas\_CSV.}}\sphinxbfcode{\sphinxupquote{procesar\_csv}}}{\emph{self}, \emph{csv\_data}}{}
Funcion para procesar la data del archivo csv, se crea una nueva data en un diccionario, se normalizan las escalas con el span y se transforma el tiempo a segundos. Para la transformacion de tiempo a segundos los formatos aceptados son

hh:mm:ss

mm:ss

ss

En cualquiera de los casos se llevara a segundos y se restara el tiempo inicial para que empiece en cero
\begin{quote}\begin{description}
\item[{Parámetros}] \leavevmode
\sphinxstyleliteralstrong{\sphinxupquote{csv\_data}} (\sphinxstyleliteralemphasis{\sphinxupquote{numpyArray}}) \textendash{} Data del csv

\end{description}\end{quote}

\end{fulllineitems}

\phantomsection\label{\detokenize{codigos/TuningHandler:module-TuningHandler}}\index{TuningHandler (módulo)@\spxentry{TuningHandler}\spxextra{módulo}}
Archivo para el manejo de la funcion de Tunning, sirve de intermediario entre la interfaz grafica y las rutinas de entonacion de controladores PID y la identificacion de modelos a partir de un archivo CSV y entonacion de PID para el mismo
\index{LEFC\_validator() (en el módulo TuningHandler)@\spxentry{LEFC\_validator()}\spxextra{en el módulo TuningHandler}}

\begin{fulllineitems}
\phantomsection\label{\detokenize{codigos/TuningHandler:TuningHandler.LEFC_validator}}\pysiglinewithargsret{\sphinxcode{\sphinxupquote{TuningHandler.}}\sphinxbfcode{\sphinxupquote{LEFC\_validator}}}{\emph{self}}{}
Validacion del limite inferior del span de EFC

\end{fulllineitems}

\index{LVP\_validator() (en el módulo TuningHandler)@\spxentry{LVP\_validator()}\spxextra{en el módulo TuningHandler}}

\begin{fulllineitems}
\phantomsection\label{\detokenize{codigos/TuningHandler:TuningHandler.LVP_validator}}\pysiglinewithargsret{\sphinxcode{\sphinxupquote{TuningHandler.}}\sphinxbfcode{\sphinxupquote{LVP\_validator}}}{\emph{self}}{}
Validacion del limite inferior del span de VP

\end{fulllineitems}

\index{PID\_bool\_discreto() (en el módulo TuningHandler)@\spxentry{PID\_bool\_discreto()}\spxextra{en el módulo TuningHandler}}

\begin{fulllineitems}
\phantomsection\label{\detokenize{codigos/TuningHandler:TuningHandler.PID_bool_discreto}}\pysiglinewithargsret{\sphinxcode{\sphinxupquote{TuningHandler.}}\sphinxbfcode{\sphinxupquote{PID\_bool\_discreto}}}{\emph{self}}{}
Funcion para habilitar y deshabilitar el periodo de muestreo

\end{fulllineitems}

\index{PID\_stacked\_to\_csv() (en el módulo TuningHandler)@\spxentry{PID\_stacked\_to\_csv()}\spxextra{en el módulo TuningHandler}}

\begin{fulllineitems}
\phantomsection\label{\detokenize{codigos/TuningHandler:TuningHandler.PID_stacked_to_csv}}\pysiglinewithargsret{\sphinxcode{\sphinxupquote{TuningHandler.}}\sphinxbfcode{\sphinxupquote{PID\_stacked\_to\_csv}}}{\emph{self}}{}
Funcion para cambiar a csv

\end{fulllineitems}

\index{PID\_stacked\_to\_ss() (en el módulo TuningHandler)@\spxentry{PID\_stacked\_to\_ss()}\spxextra{en el módulo TuningHandler}}

\begin{fulllineitems}
\phantomsection\label{\detokenize{codigos/TuningHandler:TuningHandler.PID_stacked_to_ss}}\pysiglinewithargsret{\sphinxcode{\sphinxupquote{TuningHandler.}}\sphinxbfcode{\sphinxupquote{PID\_stacked\_to\_ss}}}{\emph{self}}{}
Funcion para cambiar a ecuacion de espacio de estados

\end{fulllineitems}

\index{PID\_stacked\_to\_tf() (en el módulo TuningHandler)@\spxentry{PID\_stacked\_to\_tf()}\spxextra{en el módulo TuningHandler}}

\begin{fulllineitems}
\phantomsection\label{\detokenize{codigos/TuningHandler:TuningHandler.PID_stacked_to_tf}}\pysiglinewithargsret{\sphinxcode{\sphinxupquote{TuningHandler.}}\sphinxbfcode{\sphinxupquote{PID\_stacked\_to\_tf}}}{\emph{self}}{}
Funcion para cambiar a funcion de transferencia

\end{fulllineitems}

\index{TuningHandler() (en el módulo TuningHandler)@\spxentry{TuningHandler()}\spxextra{en el módulo TuningHandler}}

\begin{fulllineitems}
\phantomsection\label{\detokenize{codigos/TuningHandler:TuningHandler.TuningHandler}}\pysiglinewithargsret{\sphinxcode{\sphinxupquote{TuningHandler.}}\sphinxbfcode{\sphinxupquote{TuningHandler}}}{\emph{self}}{}
Funcion principal para el manejo de la funcionalida de Tunning, se crean las señales a ejecutar cuando se interactua con los widgets incluyendo las validaciones de entradas

\end{fulllineitems}

\index{UEFC\_validator() (en el módulo TuningHandler)@\spxentry{UEFC\_validator()}\spxextra{en el módulo TuningHandler}}

\begin{fulllineitems}
\phantomsection\label{\detokenize{codigos/TuningHandler:TuningHandler.UEFC_validator}}\pysiglinewithargsret{\sphinxcode{\sphinxupquote{TuningHandler.}}\sphinxbfcode{\sphinxupquote{UEFC\_validator}}}{\emph{self}}{}
Validacion del limite superior del span de EFC

\end{fulllineitems}

\index{UVP\_validator() (en el módulo TuningHandler)@\spxentry{UVP\_validator()}\spxextra{en el módulo TuningHandler}}

\begin{fulllineitems}
\phantomsection\label{\detokenize{codigos/TuningHandler:TuningHandler.UVP_validator}}\pysiglinewithargsret{\sphinxcode{\sphinxupquote{TuningHandler.}}\sphinxbfcode{\sphinxupquote{UVP\_validator}}}{\emph{self}}{}
Validacion del limite superior del span de VP

\end{fulllineitems}

\index{actualizar\_sliders\_ss() (en el módulo TuningHandler)@\spxentry{actualizar\_sliders\_ss()}\spxextra{en el módulo TuningHandler}}

\begin{fulllineitems}
\phantomsection\label{\detokenize{codigos/TuningHandler:TuningHandler.actualizar_sliders_ss}}\pysiglinewithargsret{\sphinxcode{\sphinxupquote{TuningHandler.}}\sphinxbfcode{\sphinxupquote{actualizar\_sliders\_ss}}}{\emph{self}}{}
Funcion para ajustar la resolucion de los sliders con ecuacion de espacio de estados

\end{fulllineitems}

\index{actualizar\_sliders\_tf() (en el módulo TuningHandler)@\spxentry{actualizar\_sliders\_tf()}\spxextra{en el módulo TuningHandler}}

\begin{fulllineitems}
\phantomsection\label{\detokenize{codigos/TuningHandler:TuningHandler.actualizar_sliders_tf}}\pysiglinewithargsret{\sphinxcode{\sphinxupquote{TuningHandler.}}\sphinxbfcode{\sphinxupquote{actualizar\_sliders\_tf}}}{\emph{self}}{}
Funcion para ajustar la resolucion de los sliders con funcion de transferencia

\end{fulllineitems}

\index{ajustar\_atraso\_manual() (en el módulo TuningHandler)@\spxentry{ajustar\_atraso\_manual()}\spxextra{en el módulo TuningHandler}}

\begin{fulllineitems}
\phantomsection\label{\detokenize{codigos/TuningHandler:TuningHandler.ajustar_atraso_manual}}\pysiglinewithargsret{\sphinxcode{\sphinxupquote{TuningHandler.}}\sphinxbfcode{\sphinxupquote{ajustar\_atraso\_manual}}}{\emph{self}}{}
Funcion para ajutar el tiempo t1, despues de realizar el calculo para un archivo csv, se utiliza en caso de que la estimacion automatica no sea lo suficientemente buena

\end{fulllineitems}

\index{calcular\_PID() (en el módulo TuningHandler)@\spxentry{calcular\_PID()}\spxextra{en el módulo TuningHandler}}

\begin{fulllineitems}
\phantomsection\label{\detokenize{codigos/TuningHandler:TuningHandler.calcular_PID}}\pysiglinewithargsret{\sphinxcode{\sphinxupquote{TuningHandler.}}\sphinxbfcode{\sphinxupquote{calcular\_PID}}}{\emph{self}}{}
Funcion para realizar el los calculos necesarios para la funcionalidad de entonacion de controaldores PID, el llamado a esta funcion se realizar por medio del boton calcular o cada vez que se modifique alguno de los sliders

\end{fulllineitems}

\index{calcular\_autotuning() (en el módulo TuningHandler)@\spxentry{calcular\_autotuning()}\spxextra{en el módulo TuningHandler}}

\begin{fulllineitems}
\phantomsection\label{\detokenize{codigos/TuningHandler:TuningHandler.calcular_autotuning}}\pysiglinewithargsret{\sphinxcode{\sphinxupquote{TuningHandler.}}\sphinxbfcode{\sphinxupquote{calcular\_autotuning}}}{\emph{self}}{}
Funcion para realizar el los calculos necesarios para la funcionalidad de entonacion de controaldores PID con auto tunning, el llamado a esta funcion se realizar por medio del boton calcular si previamente se habilito la funcionalidad de auto tunning

\end{fulllineitems}

\index{calcular\_csv() (en el módulo TuningHandler)@\spxentry{calcular\_csv()}\spxextra{en el módulo TuningHandler}}

\begin{fulllineitems}
\phantomsection\label{\detokenize{codigos/TuningHandler:TuningHandler.calcular_csv}}\pysiglinewithargsret{\sphinxcode{\sphinxupquote{TuningHandler.}}\sphinxbfcode{\sphinxupquote{calcular\_csv}}}{\emph{self}}{}
Funcion para realizar el los calculos necesarios para la funcionalidad de identificacion de modelos y entonacion de controlador PID, el llamado a esta funcion se realizar por medio del boton calcular

\end{fulllineitems}

\index{chequeo\_de\_accion() (en el módulo TuningHandler)@\spxentry{chequeo\_de\_accion()}\spxextra{en el módulo TuningHandler}}

\begin{fulllineitems}
\phantomsection\label{\detokenize{codigos/TuningHandler:TuningHandler.chequeo_de_accion}}\pysiglinewithargsret{\sphinxcode{\sphinxupquote{TuningHandler.}}\sphinxbfcode{\sphinxupquote{chequeo\_de\_accion}}}{\emph{self}}{}
Para discriminar entre entonacion con funcion de transferencia, ecuacion de espacio de estados o identificacion de modelo con archivo csv

\end{fulllineitems}

\index{csv\_path() (en el módulo TuningHandler)@\spxentry{csv\_path()}\spxextra{en el módulo TuningHandler}}

\begin{fulllineitems}
\phantomsection\label{\detokenize{codigos/TuningHandler:TuningHandler.csv_path}}\pysiglinewithargsret{\sphinxcode{\sphinxupquote{TuningHandler.}}\sphinxbfcode{\sphinxupquote{csv\_path}}}{\emph{self}}{}
Funcion para cargar el archivo csv

\end{fulllineitems}

\index{ssA\_validator() (en el módulo TuningHandler)@\spxentry{ssA\_validator()}\spxextra{en el módulo TuningHandler}}

\begin{fulllineitems}
\phantomsection\label{\detokenize{codigos/TuningHandler:TuningHandler.ssA_validator}}\pysiglinewithargsret{\sphinxcode{\sphinxupquote{TuningHandler.}}\sphinxbfcode{\sphinxupquote{ssA\_validator}}}{\emph{self}}{}
Validacion de la matriz de estados de la ecuacion de espacio de estados

\end{fulllineitems}

\index{ssB\_validator() (en el módulo TuningHandler)@\spxentry{ssB\_validator()}\spxextra{en el módulo TuningHandler}}

\begin{fulllineitems}
\phantomsection\label{\detokenize{codigos/TuningHandler:TuningHandler.ssB_validator}}\pysiglinewithargsret{\sphinxcode{\sphinxupquote{TuningHandler.}}\sphinxbfcode{\sphinxupquote{ssB\_validator}}}{\emph{self}}{}
Validacion de la matriz de entrada de la ecuacion de espacio de estados

\end{fulllineitems}

\index{ssC\_validator() (en el módulo TuningHandler)@\spxentry{ssC\_validator()}\spxextra{en el módulo TuningHandler}}

\begin{fulllineitems}
\phantomsection\label{\detokenize{codigos/TuningHandler:TuningHandler.ssC_validator}}\pysiglinewithargsret{\sphinxcode{\sphinxupquote{TuningHandler.}}\sphinxbfcode{\sphinxupquote{ssC\_validator}}}{\emph{self}}{}
Validacion de la matriz de salida de la ecuacion de espacio de estados

\end{fulllineitems}

\index{ssD\_validator() (en el módulo TuningHandler)@\spxentry{ssD\_validator()}\spxextra{en el módulo TuningHandler}}

\begin{fulllineitems}
\phantomsection\label{\detokenize{codigos/TuningHandler:TuningHandler.ssD_validator}}\pysiglinewithargsret{\sphinxcode{\sphinxupquote{TuningHandler.}}\sphinxbfcode{\sphinxupquote{ssD\_validator}}}{\emph{self}}{}
Validacion de la matriz de transmision directa de la ecuacion de espacio de estados

\end{fulllineitems}

\index{ss\_habilitar\_sliders\_checkbox() (en el módulo TuningHandler)@\spxentry{ss\_habilitar\_sliders\_checkbox()}\spxextra{en el módulo TuningHandler}}

\begin{fulllineitems}
\phantomsection\label{\detokenize{codigos/TuningHandler:TuningHandler.ss_habilitar_sliders_checkbox}}\pysiglinewithargsret{\sphinxcode{\sphinxupquote{TuningHandler.}}\sphinxbfcode{\sphinxupquote{ss\_habilitar\_sliders\_checkbox}}}{\emph{self}}{}
Funcion para habilitar ganancias antes y despues del auto tuning con ecuacion de espacio de estados

\end{fulllineitems}

\index{ssdelay\_validator() (en el módulo TuningHandler)@\spxentry{ssdelay\_validator()}\spxextra{en el módulo TuningHandler}}

\begin{fulllineitems}
\phantomsection\label{\detokenize{codigos/TuningHandler:TuningHandler.ssdelay_validator}}\pysiglinewithargsret{\sphinxcode{\sphinxupquote{TuningHandler.}}\sphinxbfcode{\sphinxupquote{ssdelay\_validator}}}{\emph{self}}{}
Validacion del delay de la ecuacion de espacio de estados

\end{fulllineitems}

\index{ssperiodo\_validator() (en el módulo TuningHandler)@\spxentry{ssperiodo\_validator()}\spxextra{en el módulo TuningHandler}}

\begin{fulllineitems}
\phantomsection\label{\detokenize{codigos/TuningHandler:TuningHandler.ssperiodo_validator}}\pysiglinewithargsret{\sphinxcode{\sphinxupquote{TuningHandler.}}\sphinxbfcode{\sphinxupquote{ssperiodo\_validator}}}{\emph{self}}{}
Validacion del periodo de muestreo de la ecuacion de espacio de estados

\end{fulllineitems}

\index{tf\_habilitar\_sliders\_checkbox() (en el módulo TuningHandler)@\spxentry{tf\_habilitar\_sliders\_checkbox()}\spxextra{en el módulo TuningHandler}}

\begin{fulllineitems}
\phantomsection\label{\detokenize{codigos/TuningHandler:TuningHandler.tf_habilitar_sliders_checkbox}}\pysiglinewithargsret{\sphinxcode{\sphinxupquote{TuningHandler.}}\sphinxbfcode{\sphinxupquote{tf\_habilitar\_sliders\_checkbox}}}{\emph{self}}{}
Funcion para habilitar ganancias antes y despues del auto tuning con funcion de transferencia

\end{fulllineitems}

\index{tfdelay\_validator() (en el módulo TuningHandler)@\spxentry{tfdelay\_validator()}\spxextra{en el módulo TuningHandler}}

\begin{fulllineitems}
\phantomsection\label{\detokenize{codigos/TuningHandler:TuningHandler.tfdelay_validator}}\pysiglinewithargsret{\sphinxcode{\sphinxupquote{TuningHandler.}}\sphinxbfcode{\sphinxupquote{tfdelay\_validator}}}{\emph{self}}{}
Validacion del delay de la funcion de transferencia

\end{fulllineitems}

\index{tfdem\_validator() (en el módulo TuningHandler)@\spxentry{tfdem\_validator()}\spxextra{en el módulo TuningHandler}}

\begin{fulllineitems}
\phantomsection\label{\detokenize{codigos/TuningHandler:TuningHandler.tfdem_validator}}\pysiglinewithargsret{\sphinxcode{\sphinxupquote{TuningHandler.}}\sphinxbfcode{\sphinxupquote{tfdem\_validator}}}{\emph{self}}{}
Validacion del denominador de la funcion de transferencia

\end{fulllineitems}

\index{tfnum\_validator() (en el módulo TuningHandler)@\spxentry{tfnum\_validator()}\spxextra{en el módulo TuningHandler}}

\begin{fulllineitems}
\phantomsection\label{\detokenize{codigos/TuningHandler:TuningHandler.tfnum_validator}}\pysiglinewithargsret{\sphinxcode{\sphinxupquote{TuningHandler.}}\sphinxbfcode{\sphinxupquote{tfnum\_validator}}}{\emph{self}}{}
Validacion del numerador de la funcion de transferencia

\end{fulllineitems}

\index{tfperiodo\_validator() (en el módulo TuningHandler)@\spxentry{tfperiodo\_validator()}\spxextra{en el módulo TuningHandler}}

\begin{fulllineitems}
\phantomsection\label{\detokenize{codigos/TuningHandler:TuningHandler.tfperiodo_validator}}\pysiglinewithargsret{\sphinxcode{\sphinxupquote{TuningHandler.}}\sphinxbfcode{\sphinxupquote{tfperiodo\_validator}}}{\emph{self}}{}
Validacion del periodo de muestreo de la funcion de transferencia

\end{fulllineitems}

\index{tiempo\_slider\_cambio() (en el módulo TuningHandler)@\spxentry{tiempo\_slider\_cambio()}\spxextra{en el módulo TuningHandler}}

\begin{fulllineitems}
\phantomsection\label{\detokenize{codigos/TuningHandler:TuningHandler.tiempo_slider_cambio}}\pysiglinewithargsret{\sphinxcode{\sphinxupquote{TuningHandler.}}\sphinxbfcode{\sphinxupquote{tiempo\_slider\_cambio}}}{\emph{self}}{}
Para discriminar entre entonacion e identificacion de modelo con archivo csv

\end{fulllineitems}

\index{update\_gain\_labels() (en el módulo TuningHandler)@\spxentry{update\_gain\_labels()}\spxextra{en el módulo TuningHandler}}

\begin{fulllineitems}
\phantomsection\label{\detokenize{codigos/TuningHandler:TuningHandler.update_gain_labels}}\pysiglinewithargsret{\sphinxcode{\sphinxupquote{TuningHandler.}}\sphinxbfcode{\sphinxupquote{update\_gain\_labels}}}{\emph{self}, \emph{kp=0}, \emph{ki=0}, \emph{kd=0}, \emph{autotuning=False}, \emph{resolution=50}}{}
Funcion para actualizar los labels que representan las ganancias, se ejecuta cada vez que un slider de ganancias cambia
\begin{quote}\begin{description}
\item[{Parámetros}] \leavevmode\begin{itemize}
\item {} 
\sphinxstyleliteralstrong{\sphinxupquote{kp}} (\sphinxstyleliteralemphasis{\sphinxupquote{float}}\sphinxstyleliteralemphasis{\sphinxupquote{, }}\sphinxstyleliteralemphasis{\sphinxupquote{optional}}) \textendash{} Ganancia proporcional, defaults to 0

\item {} 
\sphinxstyleliteralstrong{\sphinxupquote{ki}} (\sphinxstyleliteralemphasis{\sphinxupquote{float}}\sphinxstyleliteralemphasis{\sphinxupquote{, }}\sphinxstyleliteralemphasis{\sphinxupquote{optional}}) \textendash{} Ganancia integral, defaults to 0

\item {} 
\sphinxstyleliteralstrong{\sphinxupquote{kd}} (\sphinxstyleliteralemphasis{\sphinxupquote{float}}\sphinxstyleliteralemphasis{\sphinxupquote{, }}\sphinxstyleliteralemphasis{\sphinxupquote{optional}}) \textendash{} Ganancia derivativa, defaults to 0

\item {} 
\sphinxstyleliteralstrong{\sphinxupquote{autotuning}} (\sphinxstyleliteralemphasis{\sphinxupquote{bool}}\sphinxstyleliteralemphasis{\sphinxupquote{, }}\sphinxstyleliteralemphasis{\sphinxupquote{optional}}) \textendash{} Bandera para señar si es o no una operacion con auto tunning, defaults to False

\item {} 
\sphinxstyleliteralstrong{\sphinxupquote{resolution}} (\sphinxstyleliteralemphasis{\sphinxupquote{int}}\sphinxstyleliteralemphasis{\sphinxupquote{, }}\sphinxstyleliteralemphasis{\sphinxupquote{optional}}) \textendash{} Resolucion de los sliders, defaults to 50

\end{itemize}

\end{description}\end{quote}

\end{fulllineitems}

\index{update\_time\_and\_N\_labels() (en el módulo TuningHandler)@\spxentry{update\_time\_and\_N\_labels()}\spxextra{en el módulo TuningHandler}}

\begin{fulllineitems}
\phantomsection\label{\detokenize{codigos/TuningHandler:TuningHandler.update_time_and_N_labels}}\pysiglinewithargsret{\sphinxcode{\sphinxupquote{TuningHandler.}}\sphinxbfcode{\sphinxupquote{update\_time\_and\_N\_labels}}}{\emph{self}}{}
Funcion para actualizar los labels que representan al tiempo y al valor N

\end{fulllineitems}



\section{Archivo Handler para la función de lógica difusa}
\label{\detokenize{codigos/FuzzyHandler:archivo-handler-para-la-funcion-de-logica-difusa}}\label{\detokenize{codigos/FuzzyHandler::doc}}

\subsection{Archivo de rutinas que contiene las clases FuzzyController y FISParser}
\label{\detokenize{codigos/rutinas_fuzzy:module-rutinas_fuzzy}}\label{\detokenize{codigos/rutinas_fuzzy:archivo-de-rutinas-que-contiene-las-clases-fuzzycontroller-y-fisparser}}\label{\detokenize{codigos/rutinas_fuzzy::doc}}\index{rutinas\_fuzzy (módulo)@\spxentry{rutinas\_fuzzy}\spxextra{módulo}}
Archivo que contiene las clases FuzzyController y FISParser, para administrar el controlador difuso y cargar y exportar archivos .fis respectivamente
\index{FISParser (clase en rutinas\_fuzzy)@\spxentry{FISParser}\spxextra{clase en rutinas\_fuzzy}}

\begin{fulllineitems}
\phantomsection\label{\detokenize{codigos/rutinas_fuzzy:rutinas_fuzzy.FISParser}}\pysiglinewithargsret{\sphinxbfcode{\sphinxupquote{class }}\sphinxcode{\sphinxupquote{rutinas\_fuzzy.}}\sphinxbfcode{\sphinxupquote{FISParser}}}{\emph{file}, \emph{InputList=None}, \emph{OutputList=None}, \emph{RuleEtiquetas=None}}{}
Clase para cargar y exportar archivos .fis, para cargar los archivos FIS las funciones get\_system, get\_vars,get\_var y get\_rules fueron tomadas de yapflm:

Yet Another Python Fuzzy Logic Module: \sphinxurl{https://github.com/sputnick1124/yapflm}

Para obtener los datos necesarias del .fis, de allí, se aplica la función fis\_to\_json para completar el parsin. En el caso de la exportación, se realiza utilizando la función json\_to\_fis
\index{\_\_init\_\_() (método de rutinas\_fuzzy.FISParser)@\spxentry{\_\_init\_\_()}\spxextra{método de rutinas\_fuzzy.FISParser}}

\begin{fulllineitems}
\phantomsection\label{\detokenize{codigos/rutinas_fuzzy:rutinas_fuzzy.FISParser.__init__}}\pysiglinewithargsret{\sphinxbfcode{\sphinxupquote{\_\_init\_\_}}}{\emph{file}, \emph{InputList=None}, \emph{OutputList=None}, \emph{RuleEtiquetas=None}}{}
Constructor de la clase, inicializa las variables a utilizar y selecciona entre cargar el fis o exportarlo dependiendo de las variables con las que se cree el objeto
\begin{quote}\begin{description}
\item[{Parámetros}] \leavevmode\begin{itemize}
\item {} 
\sphinxstyleliteralstrong{\sphinxupquote{file}} (\sphinxstyleliteralemphasis{\sphinxupquote{str}}) \textendash{} Dirección del archivo a cargar o exportar

\item {} 
\sphinxstyleliteralstrong{\sphinxupquote{inputlist}} (\sphinxstyleliteralemphasis{\sphinxupquote{list}}\sphinxstyleliteralemphasis{\sphinxupquote{, }}\sphinxstyleliteralemphasis{\sphinxupquote{optional}}) \textendash{} Lista de variables de entrada, defaults to None

\item {} 
\sphinxstyleliteralstrong{\sphinxupquote{OutputList}} (\sphinxstyleliteralemphasis{\sphinxupquote{list}}\sphinxstyleliteralemphasis{\sphinxupquote{, }}\sphinxstyleliteralemphasis{\sphinxupquote{optional}}) \textendash{} Lista de variables de entrada, defaults to None

\item {} 
\sphinxstyleliteralstrong{\sphinxupquote{RuleEtiquetas}} (\sphinxstyleliteralemphasis{\sphinxupquote{list}}\sphinxstyleliteralemphasis{\sphinxupquote{, }}\sphinxstyleliteralemphasis{\sphinxupquote{optional}}) \textendash{} Lista con la información necesaria para crear las reglas, defaults to None

\end{itemize}

\end{description}\end{quote}

\end{fulllineitems}

\index{fis\_to\_json() (método de rutinas\_fuzzy.FISParser)@\spxentry{fis\_to\_json()}\spxextra{método de rutinas\_fuzzy.FISParser}}

\begin{fulllineitems}
\phantomsection\label{\detokenize{codigos/rutinas_fuzzy:rutinas_fuzzy.FISParser.fis_to_json}}\pysiglinewithargsret{\sphinxbfcode{\sphinxupquote{fis\_to\_json}}}{}{}
Función para completar la creación del controlador a partir de un archivo .fis

\end{fulllineitems}

\index{get\_rules() (método de rutinas\_fuzzy.FISParser)@\spxentry{get\_rules()}\spxextra{método de rutinas\_fuzzy.FISParser}}

\begin{fulllineitems}
\phantomsection\label{\detokenize{codigos/rutinas_fuzzy:rutinas_fuzzy.FISParser.get_rules}}\pysiglinewithargsret{\sphinxbfcode{\sphinxupquote{get\_rules}}}{}{}
Función tomada de yapflm (Yet Another Python Fuzzy Logic Module)

\end{fulllineitems}

\index{get\_system() (método de rutinas\_fuzzy.FISParser)@\spxentry{get\_system()}\spxextra{método de rutinas\_fuzzy.FISParser}}

\begin{fulllineitems}
\phantomsection\label{\detokenize{codigos/rutinas_fuzzy:rutinas_fuzzy.FISParser.get_system}}\pysiglinewithargsret{\sphinxbfcode{\sphinxupquote{get\_system}}}{}{}
Funcion tomada de yapflm (Yet Another Python Fuzzy Logic Module)

\end{fulllineitems}

\index{get\_var() (método de rutinas\_fuzzy.FISParser)@\spxentry{get\_var()}\spxextra{método de rutinas\_fuzzy.FISParser}}

\begin{fulllineitems}
\phantomsection\label{\detokenize{codigos/rutinas_fuzzy:rutinas_fuzzy.FISParser.get_var}}\pysiglinewithargsret{\sphinxbfcode{\sphinxupquote{get\_var}}}{\emph{vartype}, \emph{varnum}, \emph{start\_line}, \emph{end\_line}}{}
Funcion tomada de yapflm (Yet Another Python Fuzzy Logic Module)

\end{fulllineitems}

\index{get\_vars() (método de rutinas\_fuzzy.FISParser)@\spxentry{get\_vars()}\spxextra{método de rutinas\_fuzzy.FISParser}}

\begin{fulllineitems}
\phantomsection\label{\detokenize{codigos/rutinas_fuzzy:rutinas_fuzzy.FISParser.get_vars}}\pysiglinewithargsret{\sphinxbfcode{\sphinxupquote{get\_vars}}}{}{}
Función tomada de yapflm (Yet Another Python Fuzzy Logic Module)

\end{fulllineitems}

\index{json\_to\_fis() (método de rutinas\_fuzzy.FISParser)@\spxentry{json\_to\_fis()}\spxextra{método de rutinas\_fuzzy.FISParser}}

\begin{fulllineitems}
\phantomsection\label{\detokenize{codigos/rutinas_fuzzy:rutinas_fuzzy.FISParser.json_to_fis}}\pysiglinewithargsret{\sphinxbfcode{\sphinxupquote{json\_to\_fis}}}{}{}
Función para exportar el controlador en formato .fis

\end{fulllineitems}


\end{fulllineitems}

\index{FuzzyController (clase en rutinas\_fuzzy)@\spxentry{FuzzyController}\spxextra{clase en rutinas\_fuzzy}}

\begin{fulllineitems}
\phantomsection\label{\detokenize{codigos/rutinas_fuzzy:rutinas_fuzzy.FuzzyController}}\pysiglinewithargsret{\sphinxbfcode{\sphinxupquote{class }}\sphinxcode{\sphinxupquote{rutinas\_fuzzy.}}\sphinxbfcode{\sphinxupquote{FuzzyController}}}{\emph{inputlist}, \emph{outputlist}, \emph{rulelist={[}{]}}}{}
Clase para administrar el controlador difuso, a partir de la misma se puede crear el controlador difuso e ir creandolo de forma programatica por medio de la interfaz grafica definida en Ui\_VentanaPrincipal.py y manejada en FuzzyHandler.py
\index{\_\_init\_\_() (método de rutinas\_fuzzy.FuzzyController)@\spxentry{\_\_init\_\_()}\spxextra{método de rutinas\_fuzzy.FuzzyController}}

\begin{fulllineitems}
\phantomsection\label{\detokenize{codigos/rutinas_fuzzy:rutinas_fuzzy.FuzzyController.__init__}}\pysiglinewithargsret{\sphinxbfcode{\sphinxupquote{\_\_init\_\_}}}{\emph{inputlist}, \emph{outputlist}, \emph{rulelist={[}{]}}}{}
Se utiliza para inicializar el controlador con las entradas y salidas del mismo, en caso de que se envie el parametro opcional, rulelist, se crea el controlador a partir de las reglas suministradas y queda listo para usar
\begin{quote}\begin{description}
\item[{Parámetros}] \leavevmode\begin{itemize}
\item {} 
\sphinxstyleliteralstrong{\sphinxupquote{inputlist}} (\sphinxstyleliteralemphasis{\sphinxupquote{list}}) \textendash{} Lista de variables de entrada

\item {} 
\sphinxstyleliteralstrong{\sphinxupquote{outputlist}} (\sphinxstyleliteralemphasis{\sphinxupquote{list}}) \textendash{} Lista de variables de salida

\item {} 
\sphinxstyleliteralstrong{\sphinxupquote{rulelist}} (\sphinxstyleliteralemphasis{\sphinxupquote{list}}\sphinxstyleliteralemphasis{\sphinxupquote{, }}\sphinxstyleliteralemphasis{\sphinxupquote{optional}}) \textendash{} Lista de reglas, defaults to {[}{]}

\end{itemize}

\end{description}\end{quote}

\end{fulllineitems}

\index{agregar\_regla() (método de rutinas\_fuzzy.FuzzyController)@\spxentry{agregar\_regla()}\spxextra{método de rutinas\_fuzzy.FuzzyController}}

\begin{fulllineitems}
\phantomsection\label{\detokenize{codigos/rutinas_fuzzy:rutinas_fuzzy.FuzzyController.agregar_regla}}\pysiglinewithargsret{\sphinxbfcode{\sphinxupquote{agregar\_regla}}}{\emph{window}, \emph{Etiquetasin}, \emph{Etiquetasout}, \emph{logica}}{}
Funcion para crear una regla a partir de un set
\begin{quote}\begin{description}
\item[{Parámetros}] \leavevmode\begin{itemize}
\item {} 
\sphinxstyleliteralstrong{\sphinxupquote{window}} (\sphinxstyleliteralemphasis{\sphinxupquote{object}}) \textendash{} Objeto que contiene a la ventana principal

\item {} 
\sphinxstyleliteralstrong{\sphinxupquote{Etiquetasin}} (\sphinxstyleliteralemphasis{\sphinxupquote{list}}) \textendash{} set de entrada

\item {} 
\sphinxstyleliteralstrong{\sphinxupquote{Etiquetasout}} (\sphinxstyleliteralemphasis{\sphinxupquote{list}}) \textendash{} set de salida

\item {} 
\sphinxstyleliteralstrong{\sphinxupquote{logica}} (\sphinxstyleliteralemphasis{\sphinxupquote{bool}}) \textendash{} Logica a utilizar

\end{itemize}

\end{description}\end{quote}

\end{fulllineitems}

\index{calcular\_valor() (método de rutinas\_fuzzy.FuzzyController)@\spxentry{calcular\_valor()}\spxextra{método de rutinas\_fuzzy.FuzzyController}}

\begin{fulllineitems}
\phantomsection\label{\detokenize{codigos/rutinas_fuzzy:rutinas_fuzzy.FuzzyController.calcular_valor}}\pysiglinewithargsret{\sphinxbfcode{\sphinxupquote{calcular\_valor}}}{\emph{inputs}, \emph{outputs}}{}
Funcion para calcular las salidas del controlador dado sus entradas, esta funcion se utiliza en la funcionalidad de simulacion de sistemas de control
\begin{quote}\begin{description}
\item[{Parámetros}] \leavevmode\begin{itemize}
\item {} 
\sphinxstyleliteralstrong{\sphinxupquote{inputs}} (\sphinxstyleliteralemphasis{\sphinxupquote{list}}) \textendash{} Lista con los valores de entrada

\item {} 
\sphinxstyleliteralstrong{\sphinxupquote{outputs}} (\sphinxstyleliteralemphasis{\sphinxupquote{list}}) \textendash{} Lista vacia del tamaño del numero de salidas

\end{itemize}

\end{description}\end{quote}

\end{fulllineitems}

\index{cambiar\_metodo() (método de rutinas\_fuzzy.FuzzyController)@\spxentry{cambiar\_metodo()}\spxextra{método de rutinas\_fuzzy.FuzzyController}}

\begin{fulllineitems}
\phantomsection\label{\detokenize{codigos/rutinas_fuzzy:rutinas_fuzzy.FuzzyController.cambiar_metodo}}\pysiglinewithargsret{\sphinxbfcode{\sphinxupquote{cambiar\_metodo}}}{\emph{window}, \emph{o}, \emph{metodo}}{}
Funcion para cambiar el metodo de defuzzificacion de una salida
\begin{quote}\begin{description}
\item[{Parámetros}] \leavevmode\begin{itemize}
\item {} 
\sphinxstyleliteralstrong{\sphinxupquote{window}} (\sphinxstyleliteralemphasis{\sphinxupquote{object}}) \textendash{} Objeto que contiene a la ventana principal

\item {} 
\sphinxstyleliteralstrong{\sphinxupquote{o}} (\sphinxstyleliteralemphasis{\sphinxupquote{str}}) \textendash{} Numero de salida

\item {} 
\sphinxstyleliteralstrong{\sphinxupquote{metodo}} \textendash{} Nombre del nuevo metodo de defuzzificacion

\end{itemize}

\end{description}\end{quote}

\end{fulllineitems}

\index{cambiar\_nombre\_input() (método de rutinas\_fuzzy.FuzzyController)@\spxentry{cambiar\_nombre\_input()}\spxextra{método de rutinas\_fuzzy.FuzzyController}}

\begin{fulllineitems}
\phantomsection\label{\detokenize{codigos/rutinas_fuzzy:rutinas_fuzzy.FuzzyController.cambiar_nombre_input}}\pysiglinewithargsret{\sphinxbfcode{\sphinxupquote{cambiar\_nombre\_input}}}{\emph{window}, \emph{i}, \emph{nombre}}{}
Funcio para cambiar el nombre de una entrada
\begin{quote}\begin{description}
\item[{Parámetros}] \leavevmode\begin{itemize}
\item {} 
\sphinxstyleliteralstrong{\sphinxupquote{window}} (\sphinxstyleliteralemphasis{\sphinxupquote{object}}) \textendash{} Objeto que contiene a la ventana principal

\item {} 
\sphinxstyleliteralstrong{\sphinxupquote{i}} (\sphinxstyleliteralemphasis{\sphinxupquote{int}}) \textendash{} Numero de entrada

\item {} 
\sphinxstyleliteralstrong{\sphinxupquote{nombre}} (\sphinxstyleliteralemphasis{\sphinxupquote{str}}) \textendash{} Nuevo nombre de la entrada

\end{itemize}

\end{description}\end{quote}

\end{fulllineitems}

\index{cambiar\_nombre\_output() (método de rutinas\_fuzzy.FuzzyController)@\spxentry{cambiar\_nombre\_output()}\spxextra{método de rutinas\_fuzzy.FuzzyController}}

\begin{fulllineitems}
\phantomsection\label{\detokenize{codigos/rutinas_fuzzy:rutinas_fuzzy.FuzzyController.cambiar_nombre_output}}\pysiglinewithargsret{\sphinxbfcode{\sphinxupquote{cambiar\_nombre\_output}}}{\emph{window}, \emph{o}, \emph{nombre}}{}~\begin{quote}

Funcio para cambiar el nombre de una salida
\end{quote}
\begin{quote}\begin{description}
\item[{Parámetros}] \leavevmode\begin{itemize}
\item {} 
\sphinxstyleliteralstrong{\sphinxupquote{window}} (\sphinxstyleliteralemphasis{\sphinxupquote{object}}) \textendash{} Objeto que contiene a la ventana principal

\item {} 
\sphinxstyleliteralstrong{\sphinxupquote{o}} (\sphinxstyleliteralemphasis{\sphinxupquote{int}}) \textendash{} Numero de salida

\item {} 
\sphinxstyleliteralstrong{\sphinxupquote{nombre}} (\sphinxstyleliteralemphasis{\sphinxupquote{str}}) \textendash{} Nuevo nombre de la salida

\end{itemize}

\end{description}\end{quote}

\end{fulllineitems}

\index{cambiar\_regla() (método de rutinas\_fuzzy.FuzzyController)@\spxentry{cambiar\_regla()}\spxextra{método de rutinas\_fuzzy.FuzzyController}}

\begin{fulllineitems}
\phantomsection\label{\detokenize{codigos/rutinas_fuzzy:rutinas_fuzzy.FuzzyController.cambiar_regla}}\pysiglinewithargsret{\sphinxbfcode{\sphinxupquote{cambiar\_regla}}}{\emph{window}, \emph{Etiquetasin}, \emph{Etiquetasout}, \emph{index\_rule}, \emph{logica}}{}
Funcion para cambiar una regla a partir de un nuevo set
\begin{quote}\begin{description}
\item[{Parámetros}] \leavevmode\begin{itemize}
\item {} 
\sphinxstyleliteralstrong{\sphinxupquote{window}} (\sphinxstyleliteralemphasis{\sphinxupquote{object}}) \textendash{} Objeto que contiene a la ventana principal

\item {} 
\sphinxstyleliteralstrong{\sphinxupquote{Etiquetasin}} (\sphinxstyleliteralemphasis{\sphinxupquote{list}}) \textendash{} set de entrada

\item {} 
\sphinxstyleliteralstrong{\sphinxupquote{Etiquetasout}} (\sphinxstyleliteralemphasis{\sphinxupquote{list}}) \textendash{} set de salida

\item {} 
\sphinxstyleliteralstrong{\sphinxupquote{index\_rule}} (\sphinxstyleliteralemphasis{\sphinxupquote{int}}) \textendash{} Indice indicando la regla a cambiar

\item {} 
\sphinxstyleliteralstrong{\sphinxupquote{logica}} (\sphinxstyleliteralemphasis{\sphinxupquote{bool}}) \textendash{} Logica a utilizar

\end{itemize}

\end{description}\end{quote}

\end{fulllineitems}

\index{cambio\_etinombre\_input() (método de rutinas\_fuzzy.FuzzyController)@\spxentry{cambio\_etinombre\_input()}\spxextra{método de rutinas\_fuzzy.FuzzyController}}

\begin{fulllineitems}
\phantomsection\label{\detokenize{codigos/rutinas_fuzzy:rutinas_fuzzy.FuzzyController.cambio_etinombre_input}}\pysiglinewithargsret{\sphinxbfcode{\sphinxupquote{cambio\_etinombre\_input}}}{\emph{window}, \emph{inputlist}, \emph{i}, \emph{n}, \emph{old\_name}}{}
Funcio para cambiar el nombre de una etiqueta en la entrada seleccionada
\begin{quote}\begin{description}
\item[{Parámetros}] \leavevmode\begin{itemize}
\item {} 
\sphinxstyleliteralstrong{\sphinxupquote{window}} (\sphinxstyleliteralemphasis{\sphinxupquote{object}}) \textendash{} Objeto que contiene a la ventana principal

\item {} 
\sphinxstyleliteralstrong{\sphinxupquote{inputlist}} (\sphinxstyleliteralemphasis{\sphinxupquote{list}}) \textendash{} Lista de variables de entrada

\item {} 
\sphinxstyleliteralstrong{\sphinxupquote{i}} (\sphinxstyleliteralemphasis{\sphinxupquote{int}}) \textendash{} Numero de entrada

\item {} 
\sphinxstyleliteralstrong{\sphinxupquote{n}} (\sphinxstyleliteralemphasis{\sphinxupquote{int}}) \textendash{} Numero de etiqueta

\item {} 
\sphinxstyleliteralstrong{\sphinxupquote{old\_name}} (\sphinxstyleliteralemphasis{\sphinxupquote{str}}) \textendash{} Nombre anterior

\end{itemize}

\end{description}\end{quote}

\end{fulllineitems}

\index{cambio\_etinombre\_output() (método de rutinas\_fuzzy.FuzzyController)@\spxentry{cambio\_etinombre\_output()}\spxextra{método de rutinas\_fuzzy.FuzzyController}}

\begin{fulllineitems}
\phantomsection\label{\detokenize{codigos/rutinas_fuzzy:rutinas_fuzzy.FuzzyController.cambio_etinombre_output}}\pysiglinewithargsret{\sphinxbfcode{\sphinxupquote{cambio\_etinombre\_output}}}{\emph{window}, \emph{outputlist}, \emph{o}, \emph{n}, \emph{old\_name}}{}
Funcio para cambiar el nombre de una etiqueta en la salida seleccionada
\begin{quote}\begin{description}
\item[{Parámetros}] \leavevmode\begin{itemize}
\item {} 
\sphinxstyleliteralstrong{\sphinxupquote{window}} (\sphinxstyleliteralemphasis{\sphinxupquote{object}}) \textendash{} Objeto que contiene a la ventana principal

\item {} 
\sphinxstyleliteralstrong{\sphinxupquote{outputlist}} (\sphinxstyleliteralemphasis{\sphinxupquote{list}}) \textendash{} Lista de variables de salida

\item {} 
\sphinxstyleliteralstrong{\sphinxupquote{o}} (\sphinxstyleliteralemphasis{\sphinxupquote{int}}) \textendash{} Numero de salida

\item {} 
\sphinxstyleliteralstrong{\sphinxupquote{n}} (\sphinxstyleliteralemphasis{\sphinxupquote{int}}) \textendash{} Numero de etiqueta

\item {} 
\sphinxstyleliteralstrong{\sphinxupquote{old\_name}} (\sphinxstyleliteralemphasis{\sphinxupquote{str}}) \textendash{} Nombre anterior

\end{itemize}

\end{description}\end{quote}

\end{fulllineitems}

\index{cambio\_etiquetas\_input() (método de rutinas\_fuzzy.FuzzyController)@\spxentry{cambio\_etiquetas\_input()}\spxextra{método de rutinas\_fuzzy.FuzzyController}}

\begin{fulllineitems}
\phantomsection\label{\detokenize{codigos/rutinas_fuzzy:rutinas_fuzzy.FuzzyController.cambio_etiquetas_input}}\pysiglinewithargsret{\sphinxbfcode{\sphinxupquote{cambio\_etiquetas\_input}}}{\emph{window}, \emph{inputlist}, \emph{i}}{}
Funcion para actualizar las etiquetas de entrada del controlador
\begin{quote}\begin{description}
\item[{Parámetros}] \leavevmode\begin{itemize}
\item {} 
\sphinxstyleliteralstrong{\sphinxupquote{window}} (\sphinxstyleliteralemphasis{\sphinxupquote{object}}) \textendash{} Objeto que contiene a la ventana principal

\item {} 
\sphinxstyleliteralstrong{\sphinxupquote{inputlist}} (\sphinxstyleliteralemphasis{\sphinxupquote{list}}) \textendash{} Lista de variables de entrada

\item {} 
\sphinxstyleliteralstrong{\sphinxupquote{i}} (\sphinxstyleliteralemphasis{\sphinxupquote{int}}) \textendash{} Numero de entrada

\end{itemize}

\end{description}\end{quote}

\end{fulllineitems}

\index{cambio\_etiquetas\_output() (método de rutinas\_fuzzy.FuzzyController)@\spxentry{cambio\_etiquetas\_output()}\spxextra{método de rutinas\_fuzzy.FuzzyController}}

\begin{fulllineitems}
\phantomsection\label{\detokenize{codigos/rutinas_fuzzy:rutinas_fuzzy.FuzzyController.cambio_etiquetas_output}}\pysiglinewithargsret{\sphinxbfcode{\sphinxupquote{cambio\_etiquetas\_output}}}{\emph{window}, \emph{outputlist}, \emph{o}}{}
Funcion para actualizar las etiquetas de salida del controlador
\begin{quote}\begin{description}
\item[{Parámetros}] \leavevmode\begin{itemize}
\item {} 
\sphinxstyleliteralstrong{\sphinxupquote{window}} (\sphinxstyleliteralemphasis{\sphinxupquote{object}}) \textendash{} Objeto que contiene a la ventana principal

\item {} 
\sphinxstyleliteralstrong{\sphinxupquote{outputlist}} (\sphinxstyleliteralemphasis{\sphinxupquote{list}}) \textendash{} Lista de variables de salida

\item {} 
\sphinxstyleliteralstrong{\sphinxupquote{o}} (\sphinxstyleliteralemphasis{\sphinxupquote{int}}) \textendash{} Numero de salida

\end{itemize}

\end{description}\end{quote}

\end{fulllineitems}

\index{crear\_controlador() (método de rutinas\_fuzzy.FuzzyController)@\spxentry{crear\_controlador()}\spxextra{método de rutinas\_fuzzy.FuzzyController}}

\begin{fulllineitems}
\phantomsection\label{\detokenize{codigos/rutinas_fuzzy:rutinas_fuzzy.FuzzyController.crear_controlador}}\pysiglinewithargsret{\sphinxbfcode{\sphinxupquote{crear\_controlador}}}{}{}
Funcion para crear el controlador difuso a partir de todas las reglas creadas

\end{fulllineitems}

\index{crear\_etiquetas\_input() (método de rutinas\_fuzzy.FuzzyController)@\spxentry{crear\_etiquetas\_input()}\spxextra{método de rutinas\_fuzzy.FuzzyController}}

\begin{fulllineitems}
\phantomsection\label{\detokenize{codigos/rutinas_fuzzy:rutinas_fuzzy.FuzzyController.crear_etiquetas_input}}\pysiglinewithargsret{\sphinxbfcode{\sphinxupquote{crear\_etiquetas\_input}}}{\emph{inputlist}}{}
Funcion para crear las etiquetas de una entrada a partir de la lista de variables de entrada
\begin{quote}\begin{description}
\item[{Parámetros}] \leavevmode
\sphinxstyleliteralstrong{\sphinxupquote{inputlist}} (\sphinxstyleliteralemphasis{\sphinxupquote{list}}) \textendash{} Lista de variables de entrada

\end{description}\end{quote}

\end{fulllineitems}

\index{crear\_etiquetas\_output() (método de rutinas\_fuzzy.FuzzyController)@\spxentry{crear\_etiquetas\_output()}\spxextra{método de rutinas\_fuzzy.FuzzyController}}

\begin{fulllineitems}
\phantomsection\label{\detokenize{codigos/rutinas_fuzzy:rutinas_fuzzy.FuzzyController.crear_etiquetas_output}}\pysiglinewithargsret{\sphinxbfcode{\sphinxupquote{crear\_etiquetas\_output}}}{\emph{outputlist}}{}
Funcion para crear las etiquetas de una salida a partir de la lista de variables de salida
\begin{quote}\begin{description}
\item[{Parámetros}] \leavevmode
\sphinxstyleliteralstrong{\sphinxupquote{outputlist}} (\sphinxstyleliteralemphasis{\sphinxupquote{list}}) \textendash{} Lista de variables de salida

\end{description}\end{quote}

\end{fulllineitems}

\index{crear\_input() (método de rutinas\_fuzzy.FuzzyController)@\spxentry{crear\_input()}\spxextra{método de rutinas\_fuzzy.FuzzyController}}

\begin{fulllineitems}
\phantomsection\label{\detokenize{codigos/rutinas_fuzzy:rutinas_fuzzy.FuzzyController.crear_input}}\pysiglinewithargsret{\sphinxbfcode{\sphinxupquote{crear\_input}}}{\emph{inputlist}}{}
Funcion para crear las variables de entrada a partir de la lista de variables de entrada
\begin{quote}\begin{description}
\item[{Parámetros}] \leavevmode
\sphinxstyleliteralstrong{\sphinxupquote{inputlist}} (\sphinxstyleliteralemphasis{\sphinxupquote{list}}) \textendash{} Lista de variables de entrada

\end{description}\end{quote}

\end{fulllineitems}

\index{crear\_output() (método de rutinas\_fuzzy.FuzzyController)@\spxentry{crear\_output()}\spxextra{método de rutinas\_fuzzy.FuzzyController}}

\begin{fulllineitems}
\phantomsection\label{\detokenize{codigos/rutinas_fuzzy:rutinas_fuzzy.FuzzyController.crear_output}}\pysiglinewithargsret{\sphinxbfcode{\sphinxupquote{crear\_output}}}{\emph{outputlist}}{}
Funcion para crear las variables de salida a partir de la lista de variables de salida
\begin{quote}\begin{description}
\item[{Parámetros}] \leavevmode
\sphinxstyleliteralstrong{\sphinxupquote{outputlist}} (\sphinxstyleliteralemphasis{\sphinxupquote{list}}) \textendash{} Lista de variables de salida

\end{description}\end{quote}

\end{fulllineitems}

\index{crear\_plots\_in() (método de rutinas\_fuzzy.FuzzyController)@\spxentry{crear\_plots\_in()}\spxextra{método de rutinas\_fuzzy.FuzzyController}}

\begin{fulllineitems}
\phantomsection\label{\detokenize{codigos/rutinas_fuzzy:rutinas_fuzzy.FuzzyController.crear_plots_in}}\pysiglinewithargsret{\sphinxbfcode{\sphinxupquote{crear\_plots\_in}}}{\emph{window}, \emph{ni}}{}
Funcion para crear los objetos de graficacion de PyQtGraph de la entrada, el codigo para la obtencion de los valores de salida y el graficado es una version altamente modificada de la funcion .view() de Scikit\sphinxhyphen{}Fuzzy. Las modificaciones realizadas fueron necesarias para cambiar matplotlib por PyQtGraph
\begin{quote}\begin{description}
\item[{Parámetros}] \leavevmode\begin{itemize}
\item {} 
\sphinxstyleliteralstrong{\sphinxupquote{window}} (\sphinxstyleliteralemphasis{\sphinxupquote{object}}) \textendash{} Objeto que contiene a la ventana principal

\item {} 
\sphinxstyleliteralstrong{\sphinxupquote{ni}} (\sphinxstyleliteralemphasis{\sphinxupquote{int}}) \textendash{} Numero de entradas

\end{itemize}

\end{description}\end{quote}

\end{fulllineitems}

\index{crear\_plots\_out() (método de rutinas\_fuzzy.FuzzyController)@\spxentry{crear\_plots\_out()}\spxextra{método de rutinas\_fuzzy.FuzzyController}}

\begin{fulllineitems}
\phantomsection\label{\detokenize{codigos/rutinas_fuzzy:rutinas_fuzzy.FuzzyController.crear_plots_out}}\pysiglinewithargsret{\sphinxbfcode{\sphinxupquote{crear\_plots\_out}}}{\emph{window}, \emph{no}}{}
Funcion para crear los objetos de graficacion de PyQtGraph de la salida, el codigo para la obtencion de los valores de salida y el graficado es una version altamente modificada de la funcion .view() de Scikit\sphinxhyphen{}Fuzzy. Las modificaciones realizadas fueron necesarias para cambiar matplotlib por PyQtGraph
\begin{quote}\begin{description}
\item[{Parámetros}] \leavevmode\begin{itemize}
\item {} 
\sphinxstyleliteralstrong{\sphinxupquote{window}} (\sphinxstyleliteralemphasis{\sphinxupquote{object}}) \textendash{} Objeto que contiene a la ventana principal

\item {} 
\sphinxstyleliteralstrong{\sphinxupquote{no}} (\sphinxstyleliteralemphasis{\sphinxupquote{int}}) \textendash{} Numero de salidas

\end{itemize}

\end{description}\end{quote}

\end{fulllineitems}

\index{crear\_reglas() (método de rutinas\_fuzzy.FuzzyController)@\spxentry{crear\_reglas()}\spxextra{método de rutinas\_fuzzy.FuzzyController}}

\begin{fulllineitems}
\phantomsection\label{\detokenize{codigos/rutinas_fuzzy:rutinas_fuzzy.FuzzyController.crear_reglas}}\pysiglinewithargsret{\sphinxbfcode{\sphinxupquote{crear\_reglas}}}{\emph{rulelistC}}{}
Funcion para crear las reglas a partir de una lista que contiene toda la informacion necesaria, esta lista es creada en FuzzyHandler.py:

Cada posicion en la lista contiene un set de entradas, salidas y la logica a utilizar (AND o OR), a su vez, cada set es una lista que posee en cada posicion otra lista con la etiqueta, el numero de entrada/salida y si esta o no negada para el caso de las entradas, en caso de ser salida contiene el peso asignado
\begin{quote}\begin{description}
\item[{Parámetros}] \leavevmode
\sphinxstyleliteralstrong{\sphinxupquote{rulelistC}} (\sphinxstyleliteralemphasis{\sphinxupquote{list}}) \textendash{} Lista con la informacion necesaria para crear las reglas

\end{description}\end{quote}

\end{fulllineitems}

\index{eliminar\_regla() (método de rutinas\_fuzzy.FuzzyController)@\spxentry{eliminar\_regla()}\spxextra{método de rutinas\_fuzzy.FuzzyController}}

\begin{fulllineitems}
\phantomsection\label{\detokenize{codigos/rutinas_fuzzy:rutinas_fuzzy.FuzzyController.eliminar_regla}}\pysiglinewithargsret{\sphinxbfcode{\sphinxupquote{eliminar\_regla}}}{\emph{index\_rule}}{}
Funcion para eliminar una regla
\begin{quote}\begin{description}
\item[{Parámetros}] \leavevmode
\sphinxstyleliteralstrong{\sphinxupquote{index\_rule}} (\sphinxstyleliteralemphasis{\sphinxupquote{int}}) \textendash{} Indice indicando la regla a eliminar

\end{description}\end{quote}

\end{fulllineitems}

\index{graficar\_mf\_in() (método de rutinas\_fuzzy.FuzzyController)@\spxentry{graficar\_mf\_in()}\spxextra{método de rutinas\_fuzzy.FuzzyController}}

\begin{fulllineitems}
\phantomsection\label{\detokenize{codigos/rutinas_fuzzy:rutinas_fuzzy.FuzzyController.graficar_mf_in}}\pysiglinewithargsret{\sphinxbfcode{\sphinxupquote{graficar\_mf\_in}}}{\emph{window}, \emph{i}}{}
Funcion para graficar las funciones de membresia de una entrada
\begin{quote}\begin{description}
\item[{Parámetros}] \leavevmode\begin{itemize}
\item {} 
\sphinxstyleliteralstrong{\sphinxupquote{window}} (\sphinxstyleliteralemphasis{\sphinxupquote{object}}) \textendash{} Objeto que contiene a la ventana principal

\item {} 
\sphinxstyleliteralstrong{\sphinxupquote{i}} (\sphinxstyleliteralemphasis{\sphinxupquote{int}}) \textendash{} Numero de entrada

\end{itemize}

\end{description}\end{quote}

\end{fulllineitems}

\index{graficar\_mf\_out() (método de rutinas\_fuzzy.FuzzyController)@\spxentry{graficar\_mf\_out()}\spxextra{método de rutinas\_fuzzy.FuzzyController}}

\begin{fulllineitems}
\phantomsection\label{\detokenize{codigos/rutinas_fuzzy:rutinas_fuzzy.FuzzyController.graficar_mf_out}}\pysiglinewithargsret{\sphinxbfcode{\sphinxupquote{graficar\_mf\_out}}}{\emph{window}, \emph{o}}{}
Funcion para graficar las funciones de membresia de una salida
\begin{quote}\begin{description}
\item[{Parámetros}] \leavevmode\begin{itemize}
\item {} 
\sphinxstyleliteralstrong{\sphinxupquote{window}} (\sphinxstyleliteralemphasis{\sphinxupquote{object}}) \textendash{} Objeto que contiene a la ventana principal

\item {} 
\sphinxstyleliteralstrong{\sphinxupquote{o}} (\sphinxstyleliteralemphasis{\sphinxupquote{int}}) \textendash{} Numero de salida

\end{itemize}

\end{description}\end{quote}

\end{fulllineitems}

\index{graficar\_prueba\_pyqtgraph() (método de rutinas\_fuzzy.FuzzyController)@\spxentry{graficar\_prueba\_pyqtgraph()}\spxextra{método de rutinas\_fuzzy.FuzzyController}}

\begin{fulllineitems}
\phantomsection\label{\detokenize{codigos/rutinas_fuzzy:rutinas_fuzzy.FuzzyController.graficar_prueba_pyqtgraph}}\pysiglinewithargsret{\sphinxbfcode{\sphinxupquote{graficar\_prueba\_pyqtgraph}}}{\emph{window}, \emph{ni}, \emph{no}}{}
Funcion para actualizar la grafica en funcion de las nuevas entradas, codigo tomado y modificado de la funcion .view() de Scikit\sphinxhyphen{}Fuzzy y adaptado para su uso con PyQtGraph
\begin{quote}\begin{description}
\item[{Parámetros}] \leavevmode\begin{itemize}
\item {} 
\sphinxstyleliteralstrong{\sphinxupquote{window}} (\sphinxstyleliteralemphasis{\sphinxupquote{object}}) \textendash{} Objeto que contiene a la ventana principal

\item {} 
\sphinxstyleliteralstrong{\sphinxupquote{ni}} (\sphinxstyleliteralemphasis{\sphinxupquote{int}}) \textendash{} Numero de entradas

\item {} 
\sphinxstyleliteralstrong{\sphinxupquote{no}} (\sphinxstyleliteralemphasis{\sphinxupquote{int}}) \textendash{} Numero de salidas

\end{itemize}

\end{description}\end{quote}

\end{fulllineitems}

\index{graficar\_respuesta\_2d() (método de rutinas\_fuzzy.FuzzyController)@\spxentry{graficar\_respuesta\_2d()}\spxextra{método de rutinas\_fuzzy.FuzzyController}}

\begin{fulllineitems}
\phantomsection\label{\detokenize{codigos/rutinas_fuzzy:rutinas_fuzzy.FuzzyController.graficar_respuesta_2d}}\pysiglinewithargsret{\sphinxbfcode{\sphinxupquote{graficar\_respuesta\_2d}}}{\emph{window}, \emph{inrange}, \emph{no}}{}
Funcion para graficar la respuesta del controlador en caso de poseer una entrada
\begin{quote}\begin{description}
\item[{Parámetros}] \leavevmode\begin{itemize}
\item {} 
\sphinxstyleliteralstrong{\sphinxupquote{window}} (\sphinxstyleliteralemphasis{\sphinxupquote{object}}) \textendash{} Objeto que contiene a la ventana principal

\item {} 
\sphinxstyleliteralstrong{\sphinxupquote{inrange}} (\sphinxstyleliteralemphasis{\sphinxupquote{list}}) \textendash{} Rango de la variable de entrada

\item {} 
\sphinxstyleliteralstrong{\sphinxupquote{no}} (\sphinxstyleliteralemphasis{\sphinxupquote{int}}) \textendash{} Numero de salidas

\end{itemize}

\end{description}\end{quote}

\end{fulllineitems}

\index{graficar\_respuesta\_3d() (método de rutinas\_fuzzy.FuzzyController)@\spxentry{graficar\_respuesta\_3d()}\spxextra{método de rutinas\_fuzzy.FuzzyController}}

\begin{fulllineitems}
\phantomsection\label{\detokenize{codigos/rutinas_fuzzy:rutinas_fuzzy.FuzzyController.graficar_respuesta_3d}}\pysiglinewithargsret{\sphinxbfcode{\sphinxupquote{graficar\_respuesta\_3d}}}{\emph{window}, \emph{inrange1}, \emph{inrange2}, \emph{no}}{}
Funcion para graficar la superficie de respuesta del controlador en caso de poseer 2 entradas
\begin{quote}\begin{description}
\item[{Parámetros}] \leavevmode\begin{itemize}
\item {} 
\sphinxstyleliteralstrong{\sphinxupquote{window}} (\sphinxstyleliteralemphasis{\sphinxupquote{object}}) \textendash{} Objeto que contiene a la ventana principal

\item {} 
\sphinxstyleliteralstrong{\sphinxupquote{inrange1}} (\sphinxstyleliteralemphasis{\sphinxupquote{list}}) \textendash{} Rango de la variable de entrada uno

\item {} 
\sphinxstyleliteralstrong{\sphinxupquote{inrange2}} (\sphinxstyleliteralemphasis{\sphinxupquote{list}}) \textendash{} Rango de la variable de entrada dos

\item {} 
\sphinxstyleliteralstrong{\sphinxupquote{no}} (\sphinxstyleliteralemphasis{\sphinxupquote{int}}) \textendash{} Numero de salidas

\end{itemize}

\end{description}\end{quote}

\end{fulllineitems}

\index{prueba\_de\_controlador() (método de rutinas\_fuzzy.FuzzyController)@\spxentry{prueba\_de\_controlador()}\spxextra{método de rutinas\_fuzzy.FuzzyController}}

\begin{fulllineitems}
\phantomsection\label{\detokenize{codigos/rutinas_fuzzy:rutinas_fuzzy.FuzzyController.prueba_de_controlador}}\pysiglinewithargsret{\sphinxbfcode{\sphinxupquote{prueba\_de\_controlador}}}{\emph{window}, \emph{values}, \emph{ni}, \emph{no}}{}
Funcion para realizar la prueba del controlador
\begin{quote}\begin{description}
\item[{Parámetros}] \leavevmode\begin{itemize}
\item {} 
\sphinxstyleliteralstrong{\sphinxupquote{window}} (\sphinxstyleliteralemphasis{\sphinxupquote{object}}) \textendash{} Objeto que contiene a la ventana principal

\item {} 
\sphinxstyleliteralstrong{\sphinxupquote{values}} (\sphinxstyleliteralemphasis{\sphinxupquote{list}}) \textendash{} Valores de entradas dados por el usuario con los sliders

\item {} 
\sphinxstyleliteralstrong{\sphinxupquote{ni}} (\sphinxstyleliteralemphasis{\sphinxupquote{int}}) \textendash{} Numero de entradas

\item {} 
\sphinxstyleliteralstrong{\sphinxupquote{no}} (\sphinxstyleliteralemphasis{\sphinxupquote{int}}) \textendash{} Numero de salidas

\end{itemize}

\end{description}\end{quote}

\end{fulllineitems}

\index{update\_definicion\_input() (método de rutinas\_fuzzy.FuzzyController)@\spxentry{update\_definicion\_input()}\spxextra{método de rutinas\_fuzzy.FuzzyController}}

\begin{fulllineitems}
\phantomsection\label{\detokenize{codigos/rutinas_fuzzy:rutinas_fuzzy.FuzzyController.update_definicion_input}}\pysiglinewithargsret{\sphinxbfcode{\sphinxupquote{update\_definicion\_input}}}{\emph{window}, \emph{inputlist}, \emph{i}, \emph{n}}{}
Funcion para actualizar la definicion de una funcion de membresia en la entrada seleccionada
\begin{quote}\begin{description}
\item[{Parámetros}] \leavevmode\begin{itemize}
\item {} 
\sphinxstyleliteralstrong{\sphinxupquote{window}} (\sphinxstyleliteralemphasis{\sphinxupquote{object}}) \textendash{} Objeto que contiene a la ventana principal

\item {} 
\sphinxstyleliteralstrong{\sphinxupquote{inputlist}} (\sphinxstyleliteralemphasis{\sphinxupquote{list}}) \textendash{} Lista de variables de entrada

\item {} 
\sphinxstyleliteralstrong{\sphinxupquote{i}} (\sphinxstyleliteralemphasis{\sphinxupquote{int}}) \textendash{} Numero de entrada

\item {} 
\sphinxstyleliteralstrong{\sphinxupquote{n}} (\sphinxstyleliteralemphasis{\sphinxupquote{int}}) \textendash{} Numero de etiqueta

\end{itemize}

\end{description}\end{quote}

\end{fulllineitems}

\index{update\_definicion\_output() (método de rutinas\_fuzzy.FuzzyController)@\spxentry{update\_definicion\_output()}\spxextra{método de rutinas\_fuzzy.FuzzyController}}

\begin{fulllineitems}
\phantomsection\label{\detokenize{codigos/rutinas_fuzzy:rutinas_fuzzy.FuzzyController.update_definicion_output}}\pysiglinewithargsret{\sphinxbfcode{\sphinxupquote{update\_definicion\_output}}}{\emph{window}, \emph{outputlist}, \emph{o}, \emph{n}}{}
Funcion para actualizar la definicion de una funcion de membresia en la salida seleccionada
\begin{quote}\begin{description}
\item[{Parámetros}] \leavevmode\begin{itemize}
\item {} 
\sphinxstyleliteralstrong{\sphinxupquote{window}} (\sphinxstyleliteralemphasis{\sphinxupquote{object}}) \textendash{} Objeto que contiene a la ventana principal

\item {} 
\sphinxstyleliteralstrong{\sphinxupquote{outputlist}} (\sphinxstyleliteralemphasis{\sphinxupquote{list}}) \textendash{} Lista de variables de salida

\item {} 
\sphinxstyleliteralstrong{\sphinxupquote{o}} (\sphinxstyleliteralemphasis{\sphinxupquote{int}}) \textendash{} Numero de salida

\item {} 
\sphinxstyleliteralstrong{\sphinxupquote{n}} (\sphinxstyleliteralemphasis{\sphinxupquote{int}}) \textendash{} Numero de etiqueta

\end{itemize}

\end{description}\end{quote}

\end{fulllineitems}

\index{update\_rango\_input() (método de rutinas\_fuzzy.FuzzyController)@\spxentry{update\_rango\_input()}\spxextra{método de rutinas\_fuzzy.FuzzyController}}

\begin{fulllineitems}
\phantomsection\label{\detokenize{codigos/rutinas_fuzzy:rutinas_fuzzy.FuzzyController.update_rango_input}}\pysiglinewithargsret{\sphinxbfcode{\sphinxupquote{update\_rango\_input}}}{\emph{window}, \emph{inputlist}, \emph{i}}{}
Funcion para actualizar el universo de discurso de una entrada
\begin{quote}\begin{description}
\item[{Parámetros}] \leavevmode\begin{itemize}
\item {} 
\sphinxstyleliteralstrong{\sphinxupquote{window}} (\sphinxstyleliteralemphasis{\sphinxupquote{object}}) \textendash{} Objeto que contiene a la ventana principal

\item {} 
\sphinxstyleliteralstrong{\sphinxupquote{inputlist}} (\sphinxstyleliteralemphasis{\sphinxupquote{list}}) \textendash{} Lista de variables de entrada

\item {} 
\sphinxstyleliteralstrong{\sphinxupquote{i}} (\sphinxstyleliteralemphasis{\sphinxupquote{int}}) \textendash{} Numero de entrada

\end{itemize}

\end{description}\end{quote}

\end{fulllineitems}

\index{update\_rango\_output() (método de rutinas\_fuzzy.FuzzyController)@\spxentry{update\_rango\_output()}\spxextra{método de rutinas\_fuzzy.FuzzyController}}

\begin{fulllineitems}
\phantomsection\label{\detokenize{codigos/rutinas_fuzzy:rutinas_fuzzy.FuzzyController.update_rango_output}}\pysiglinewithargsret{\sphinxbfcode{\sphinxupquote{update\_rango\_output}}}{\emph{window}, \emph{outputlist}, \emph{o}}{}
Funcion para actualizar el universo de discurso de una salida
\begin{quote}\begin{description}
\item[{Parámetros}] \leavevmode\begin{itemize}
\item {} 
\sphinxstyleliteralstrong{\sphinxupquote{window}} (\sphinxstyleliteralemphasis{\sphinxupquote{object}}) \textendash{} Objeto que contiene a la ventana principal

\item {} 
\sphinxstyleliteralstrong{\sphinxupquote{outputlist}} (\sphinxstyleliteralemphasis{\sphinxupquote{list}}) \textendash{} Lista de variables de salida

\item {} 
\sphinxstyleliteralstrong{\sphinxupquote{o}} (\sphinxstyleliteralemphasis{\sphinxupquote{int}}) \textendash{} Numero de salida

\end{itemize}

\end{description}\end{quote}

\end{fulllineitems}


\end{fulllineitems}



\subsection{Archivo para el cambio de definición entre funciones de membresía}
\label{\detokenize{codigos/modificadorMF:archivo-para-el-cambio-de-definicion-entre-funciones-de-membresia}}\label{\detokenize{codigos/modificadorMF::doc}}\phantomsection\label{\detokenize{codigos/modificadorMF:module-modificadorMf}}\index{modificadorMf (módulo)@\spxentry{modificadorMf}\spxextra{módulo}}
Archivo para el cambio de definición entre funciones de membresía, los cambios se realizan en dos pasos:
\begin{quote}

old\_mf  \sphinxhyphen{}\textgreater{}  trimf
trimf   \sphinxhyphen{}\textgreater{}  new\_mf
\end{quote}

De este modo se reduce el número de casos a codificar. Por otro lado, también contiene la función para realizar la validación de las definiciones ingresadas por el usuario
\index{update\_definicionmf() (en el módulo modificadorMf)@\spxentry{update\_definicionmf()}\spxextra{en el módulo modificadorMf}}

\begin{fulllineitems}
\phantomsection\label{\detokenize{codigos/modificadorMF:modificadorMf.update_definicionmf}}\pysiglinewithargsret{\sphinxcode{\sphinxupquote{modificadorMf.}}\sphinxbfcode{\sphinxupquote{update\_definicionmf}}}{\emph{self}, \emph{old\_mf}, \emph{definicion}, \emph{new\_mf}}{}
Función para la transformación equivalente entre funciones de membresía
\begin{quote}\begin{description}
\item[{Parámetros}] \leavevmode\begin{itemize}
\item {} 
\sphinxstyleliteralstrong{\sphinxupquote{old\_mf}} (\sphinxstyleliteralemphasis{\sphinxupquote{str}}) \textendash{} Nombre de la antigua función de membresía

\item {} 
\sphinxstyleliteralstrong{\sphinxupquote{definicion}} (\sphinxstyleliteralemphasis{\sphinxupquote{list}}) \textendash{} Lista con los valores correspondiente a la definición de la antigua función de membresía

\item {} 
\sphinxstyleliteralstrong{\sphinxupquote{new\_mf}} (\sphinxstyleliteralemphasis{\sphinxupquote{str}}) \textendash{} Nombre de la nueva función de membresía

\end{itemize}

\end{description}\end{quote}

\end{fulllineitems}

\index{validacion\_mf() (en el módulo modificadorMf)@\spxentry{validacion\_mf()}\spxextra{en el módulo modificadorMf}}

\begin{fulllineitems}
\phantomsection\label{\detokenize{codigos/modificadorMF:modificadorMf.validacion_mf}}\pysiglinewithargsret{\sphinxcode{\sphinxupquote{modificadorMf.}}\sphinxbfcode{\sphinxupquote{validacion\_mf}}}{\emph{self}, \emph{\_}, \emph{mf}}{}
Funcion para validar las definiciones ingresadas por el usuario
\begin{quote}\begin{description}
\item[{Parámetros}] \leavevmode\begin{itemize}
\item {} 
\sphinxstyleliteralstrong{\sphinxupquote{\_}} (\sphinxstyleliteralemphasis{\sphinxupquote{list}}) \textendash{} Definicion

\item {} 
\sphinxstyleliteralstrong{\sphinxupquote{mf}} (\sphinxstyleliteralemphasis{\sphinxupquote{str}}) \textendash{} Nombre de la funcion de membresia a validar

\end{itemize}

\end{description}\end{quote}

\end{fulllineitems}

\phantomsection\label{\detokenize{codigos/FuzzyHandler:module-FuzzyHandler}}\index{FuzzyHandler (módulo)@\spxentry{FuzzyHandler}\spxextra{módulo}}
Archivo para el manejo de la funcion de diseño de controladores difusos, sirve de intermediario entre la interfaz grafica y la clase creada para manejar el controlador difuso definida en rutinas\_fuzzy.py
\index{EtiquetasDic\_creator() (en el módulo FuzzyHandler)@\spxentry{EtiquetasDic\_creator()}\spxextra{en el módulo FuzzyHandler}}

\begin{fulllineitems}
\phantomsection\label{\detokenize{codigos/FuzzyHandler:FuzzyHandler.EtiquetasDic_creator}}\pysiglinewithargsret{\sphinxcode{\sphinxupquote{FuzzyHandler.}}\sphinxbfcode{\sphinxupquote{EtiquetasDic\_creator}}}{\emph{self}, \emph{j}, \emph{erange}}{}
Funcion para crear etiquetas genericas
\begin{quote}\begin{description}
\item[{Parámetros}] \leavevmode\begin{itemize}
\item {} 
\sphinxstyleliteralstrong{\sphinxupquote{j}} (\sphinxstyleliteralemphasis{\sphinxupquote{int}}) \textendash{} Numero de etiqueta

\item {} 
\sphinxstyleliteralstrong{\sphinxupquote{erange}} (\sphinxstyleliteralemphasis{\sphinxupquote{list}}) \textendash{} Definicio de la funcion de membresia

\end{itemize}

\end{description}\end{quote}

\end{fulllineitems}

\index{FuzzyHandler() (en el módulo FuzzyHandler)@\spxentry{FuzzyHandler()}\spxextra{en el módulo FuzzyHandler}}

\begin{fulllineitems}
\phantomsection\label{\detokenize{codigos/FuzzyHandler:FuzzyHandler.FuzzyHandler}}\pysiglinewithargsret{\sphinxcode{\sphinxupquote{FuzzyHandler.}}\sphinxbfcode{\sphinxupquote{FuzzyHandler}}}{\emph{self}}{}
Funcion principal para el manejo de diseño de controladores difusos, se crean las señales a ejecutar cuando se interactua con los widgets

\end{fulllineitems}

\index{actualizar\_RulesEtiquetas\_in() (en el módulo FuzzyHandler)@\spxentry{actualizar\_RulesEtiquetas\_in()}\spxextra{en el módulo FuzzyHandler}}

\begin{fulllineitems}
\phantomsection\label{\detokenize{codigos/FuzzyHandler:FuzzyHandler.actualizar_RulesEtiquetas_in}}\pysiglinewithargsret{\sphinxcode{\sphinxupquote{FuzzyHandler.}}\sphinxbfcode{\sphinxupquote{actualizar\_RulesEtiquetas\_in}}}{\emph{self}, \emph{ni}, \emph{new\_name}, \emph{old\_name}}{}
Funcion para actualizar el nombre en las reglas previamente creadas con el nuevo nombre de una etiqueta
\begin{quote}\begin{description}
\item[{Parámetros}] \leavevmode\begin{itemize}
\item {} 
\sphinxstyleliteralstrong{\sphinxupquote{ni}} (\sphinxstyleliteralemphasis{\sphinxupquote{int}}) \textendash{} Numero de entrada

\item {} 
\sphinxstyleliteralstrong{\sphinxupquote{new\_name}} (\sphinxstyleliteralemphasis{\sphinxupquote{str}}) \textendash{} Nuevo nombre para la etiqueta a cambiar

\item {} 
\sphinxstyleliteralstrong{\sphinxupquote{old\_name}} (\sphinxstyleliteralemphasis{\sphinxupquote{str}}) \textendash{} Antiguo nombre de la etiqueta a cambiar

\end{itemize}

\end{description}\end{quote}

\end{fulllineitems}

\index{actualizar\_RulesEtiquetas\_out() (en el módulo FuzzyHandler)@\spxentry{actualizar\_RulesEtiquetas\_out()}\spxextra{en el módulo FuzzyHandler}}

\begin{fulllineitems}
\phantomsection\label{\detokenize{codigos/FuzzyHandler:FuzzyHandler.actualizar_RulesEtiquetas_out}}\pysiglinewithargsret{\sphinxcode{\sphinxupquote{FuzzyHandler.}}\sphinxbfcode{\sphinxupquote{actualizar\_RulesEtiquetas\_out}}}{\emph{self}, \emph{no}, \emph{new\_name}, \emph{old\_name}}{}
Funcion para actualizar el nombre en las reglas previamente creadas con el nuevo nombre de una etiqueta
\begin{quote}\begin{description}
\item[{Parámetros}] \leavevmode\begin{itemize}
\item {} 
\sphinxstyleliteralstrong{\sphinxupquote{no}} (\sphinxstyleliteralemphasis{\sphinxupquote{int}}) \textendash{} Numero de salida

\item {} 
\sphinxstyleliteralstrong{\sphinxupquote{new\_name}} (\sphinxstyleliteralemphasis{\sphinxupquote{str}}) \textendash{} Nuevo nombre para la etiqueta a cambiar

\item {} 
\sphinxstyleliteralstrong{\sphinxupquote{old\_name}} (\sphinxstyleliteralemphasis{\sphinxupquote{str}}) \textendash{} Antiguo nombre de la etiqueta a cambiar

\end{itemize}

\end{description}\end{quote}

\end{fulllineitems}

\index{cargar\_controlador() (en el módulo FuzzyHandler)@\spxentry{cargar\_controlador()}\spxextra{en el módulo FuzzyHandler}}

\begin{fulllineitems}
\phantomsection\label{\detokenize{codigos/FuzzyHandler:FuzzyHandler.cargar_controlador}}\pysiglinewithargsret{\sphinxcode{\sphinxupquote{FuzzyHandler.}}\sphinxbfcode{\sphinxupquote{cargar\_controlador}}}{\emph{self}}{}
Funcion manejar el cargado de controaldores previamente diseñados, se aceptan formatos .JSON y .FIS

\end{fulllineitems}

\index{cargar\_esquema() (en el módulo FuzzyHandler)@\spxentry{cargar\_esquema()}\spxextra{en el módulo FuzzyHandler}}

\begin{fulllineitems}
\phantomsection\label{\detokenize{codigos/FuzzyHandler:FuzzyHandler.cargar_esquema}}\pysiglinewithargsret{\sphinxcode{\sphinxupquote{FuzzyHandler.}}\sphinxbfcode{\sphinxupquote{cargar\_esquema}}}{\emph{self}}{}
Funcion para iniciar el entorno de diseño a partir de un esquema de control seleccionado

\end{fulllineitems}

\index{cerrar\_prueba() (en el módulo FuzzyHandler)@\spxentry{cerrar\_prueba()}\spxextra{en el módulo FuzzyHandler}}

\begin{fulllineitems}
\phantomsection\label{\detokenize{codigos/FuzzyHandler:FuzzyHandler.cerrar_prueba}}\pysiglinewithargsret{\sphinxcode{\sphinxupquote{FuzzyHandler.}}\sphinxbfcode{\sphinxupquote{cerrar\_prueba}}}{\emph{self}}{}
Funcion para cerrar las pestañas de pruebas ante cambios en el controlador difuso

\end{fulllineitems}

\index{check\_esquema\_show() (en el módulo FuzzyHandler)@\spxentry{check\_esquema\_show()}\spxextra{en el módulo FuzzyHandler}}

\begin{fulllineitems}
\phantomsection\label{\detokenize{codigos/FuzzyHandler:FuzzyHandler.check_esquema_show}}\pysiglinewithargsret{\sphinxcode{\sphinxupquote{FuzzyHandler.}}\sphinxbfcode{\sphinxupquote{check\_esquema\_show}}}{\emph{self}}{}
Funcion para mediar entre entradas y salidas genericas y esquemas de control

\end{fulllineitems}

\index{crear\_controlador() (en el módulo FuzzyHandler)@\spxentry{crear\_controlador()}\spxextra{en el módulo FuzzyHandler}}

\begin{fulllineitems}
\phantomsection\label{\detokenize{codigos/FuzzyHandler:FuzzyHandler.crear_controlador}}\pysiglinewithargsret{\sphinxcode{\sphinxupquote{FuzzyHandler.}}\sphinxbfcode{\sphinxupquote{crear\_controlador}}}{\emph{self}}{}
Funcion para crear el controlador a partir de toda la informacion recolectada, esta creacion se realiza con el fin de realizar la prueba del controlador y observar la superficie de respuesta del controlador en caso de poseer una o dos entradas

\end{fulllineitems}

\index{crear\_tabs() (en el módulo FuzzyHandler)@\spxentry{crear\_tabs()}\spxextra{en el módulo FuzzyHandler}}

\begin{fulllineitems}
\phantomsection\label{\detokenize{codigos/FuzzyHandler:FuzzyHandler.crear_tabs}}\pysiglinewithargsret{\sphinxcode{\sphinxupquote{FuzzyHandler.}}\sphinxbfcode{\sphinxupquote{crear\_tabs}}}{\emph{self}}{}
Funcion para iniciar el entorno de diseño para entradas y salidas genericas

\end{fulllineitems}

\index{crear\_vectores\_de\_widgets() (en el módulo FuzzyHandler)@\spxentry{crear\_vectores\_de\_widgets()}\spxextra{en el módulo FuzzyHandler}}

\begin{fulllineitems}
\phantomsection\label{\detokenize{codigos/FuzzyHandler:FuzzyHandler.crear_vectores_de_widgets}}\pysiglinewithargsret{\sphinxcode{\sphinxupquote{FuzzyHandler.}}\sphinxbfcode{\sphinxupquote{crear\_vectores\_de\_widgets}}}{\emph{self}}{}
Funcion para el almacenado de widgets en listas para acceder a ellos por indices

\end{fulllineitems}

\index{definicion\_in() (en el módulo FuzzyHandler)@\spxentry{definicion\_in()}\spxextra{en el módulo FuzzyHandler}}

\begin{fulllineitems}
\phantomsection\label{\detokenize{codigos/FuzzyHandler:FuzzyHandler.definicion_in}}\pysiglinewithargsret{\sphinxcode{\sphinxupquote{FuzzyHandler.}}\sphinxbfcode{\sphinxupquote{definicion\_in}}}{\emph{self}}{}
Funcion para manejar el cambio de definicion de la funcion de membresia correspondiente a la etiqueta actual

\end{fulllineitems}

\index{definicion\_out() (en el módulo FuzzyHandler)@\spxentry{definicion\_out()}\spxextra{en el módulo FuzzyHandler}}

\begin{fulllineitems}
\phantomsection\label{\detokenize{codigos/FuzzyHandler:FuzzyHandler.definicion_out}}\pysiglinewithargsret{\sphinxcode{\sphinxupquote{FuzzyHandler.}}\sphinxbfcode{\sphinxupquote{definicion\_out}}}{\emph{self}}{}
Funcion para manejar el cambio de definicion de la funcion de membresia correspondiente a la etiqueta actual

\end{fulllineitems}

\index{defuzz\_metodo() (en el módulo FuzzyHandler)@\spxentry{defuzz\_metodo()}\spxextra{en el módulo FuzzyHandler}}

\begin{fulllineitems}
\phantomsection\label{\detokenize{codigos/FuzzyHandler:FuzzyHandler.defuzz_metodo}}\pysiglinewithargsret{\sphinxcode{\sphinxupquote{FuzzyHandler.}}\sphinxbfcode{\sphinxupquote{defuzz\_metodo}}}{\emph{self}}{}
Funcion para manejar el metodo de defuzzificacion para la salida seleccionada

\end{fulllineitems}

\index{deinificion\_in\_validator() (en el módulo FuzzyHandler)@\spxentry{deinificion\_in\_validator()}\spxextra{en el módulo FuzzyHandler}}

\begin{fulllineitems}
\phantomsection\label{\detokenize{codigos/FuzzyHandler:FuzzyHandler.deinificion_in_validator}}\pysiglinewithargsret{\sphinxcode{\sphinxupquote{FuzzyHandler.}}\sphinxbfcode{\sphinxupquote{deinificion\_in\_validator}}}{\emph{self}}{}
Funcion para validar las definiciones de las funciones de membresia

\end{fulllineitems}

\index{deinificion\_out\_validator() (en el módulo FuzzyHandler)@\spxentry{deinificion\_out\_validator()}\spxextra{en el módulo FuzzyHandler}}

\begin{fulllineitems}
\phantomsection\label{\detokenize{codigos/FuzzyHandler:FuzzyHandler.deinificion_out_validator}}\pysiglinewithargsret{\sphinxcode{\sphinxupquote{FuzzyHandler.}}\sphinxbfcode{\sphinxupquote{deinificion\_out\_validator}}}{\emph{self}}{}
Funcion para validar las definiciones de las funciones de membresia

\end{fulllineitems}

\index{exportar\_fis() (en el módulo FuzzyHandler)@\spxentry{exportar\_fis()}\spxextra{en el módulo FuzzyHandler}}

\begin{fulllineitems}
\phantomsection\label{\detokenize{codigos/FuzzyHandler:FuzzyHandler.exportar_fis}}\pysiglinewithargsret{\sphinxcode{\sphinxupquote{FuzzyHandler.}}\sphinxbfcode{\sphinxupquote{exportar\_fis}}}{\emph{self}}{}
Funcion manejar el exportado del controlador diseñado a formato .FIS

\end{fulllineitems}

\index{guardar\_controlador() (en el módulo FuzzyHandler)@\spxentry{guardar\_controlador()}\spxextra{en el módulo FuzzyHandler}}

\begin{fulllineitems}
\phantomsection\label{\detokenize{codigos/FuzzyHandler:FuzzyHandler.guardar_controlador}}\pysiglinewithargsret{\sphinxcode{\sphinxupquote{FuzzyHandler.}}\sphinxbfcode{\sphinxupquote{guardar\_controlador}}}{\emph{self}}{}
Funcion manejar el guardado del controlador diseñado

\end{fulllineitems}

\index{guardarcomo\_controlador() (en el módulo FuzzyHandler)@\spxentry{guardarcomo\_controlador()}\spxextra{en el módulo FuzzyHandler}}

\begin{fulllineitems}
\phantomsection\label{\detokenize{codigos/FuzzyHandler:FuzzyHandler.guardarcomo_controlador}}\pysiglinewithargsret{\sphinxcode{\sphinxupquote{FuzzyHandler.}}\sphinxbfcode{\sphinxupquote{guardarcomo\_controlador}}}{\emph{self}}{}
Funcion manejar el guardado en un nuevo archivo del controlador diseñado

\end{fulllineitems}

\index{imagen\_entradas() (en el módulo FuzzyHandler)@\spxentry{imagen\_entradas()}\spxextra{en el módulo FuzzyHandler}}

\begin{fulllineitems}
\phantomsection\label{\detokenize{codigos/FuzzyHandler:FuzzyHandler.imagen_entradas}}\pysiglinewithargsret{\sphinxcode{\sphinxupquote{FuzzyHandler.}}\sphinxbfcode{\sphinxupquote{imagen\_entradas}}}{\emph{self}}{}
Funcion para establecer la imagen del numero de entradas

\end{fulllineitems}

\index{imagen\_salidas() (en el módulo FuzzyHandler)@\spxentry{imagen\_salidas()}\spxextra{en el módulo FuzzyHandler}}

\begin{fulllineitems}
\phantomsection\label{\detokenize{codigos/FuzzyHandler:FuzzyHandler.imagen_salidas}}\pysiglinewithargsret{\sphinxcode{\sphinxupquote{FuzzyHandler.}}\sphinxbfcode{\sphinxupquote{imagen\_salidas}}}{\emph{self}}{}
Funcion para establecer la imagen del numero de salidas

\end{fulllineitems}

\index{inputDic\_creator() (en el módulo FuzzyHandler)@\spxentry{inputDic\_creator()}\spxextra{en el módulo FuzzyHandler}}

\begin{fulllineitems}
\phantomsection\label{\detokenize{codigos/FuzzyHandler:FuzzyHandler.inputDic_creator}}\pysiglinewithargsret{\sphinxcode{\sphinxupquote{FuzzyHandler.}}\sphinxbfcode{\sphinxupquote{inputDic\_creator}}}{\emph{self}, \emph{i}}{}
Funcion para crear entradas genericas
\begin{quote}\begin{description}
\item[{Parámetros}] \leavevmode
\sphinxstyleliteralstrong{\sphinxupquote{i}} (\sphinxstyleliteralemphasis{\sphinxupquote{int}}) \textendash{} Numero de entrada

\end{description}\end{quote}

\end{fulllineitems}

\index{nombre\_entrada() (en el módulo FuzzyHandler)@\spxentry{nombre\_entrada()}\spxextra{en el módulo FuzzyHandler}}

\begin{fulllineitems}
\phantomsection\label{\detokenize{codigos/FuzzyHandler:FuzzyHandler.nombre_entrada}}\pysiglinewithargsret{\sphinxcode{\sphinxupquote{FuzzyHandler.}}\sphinxbfcode{\sphinxupquote{nombre\_entrada}}}{\emph{self}}{}
Funcion para manejar el cambio de nombre de la entrada seleccionada

\end{fulllineitems}

\index{nombre\_etiqueta\_in() (en el módulo FuzzyHandler)@\spxentry{nombre\_etiqueta\_in()}\spxextra{en el módulo FuzzyHandler}}

\begin{fulllineitems}
\phantomsection\label{\detokenize{codigos/FuzzyHandler:FuzzyHandler.nombre_etiqueta_in}}\pysiglinewithargsret{\sphinxcode{\sphinxupquote{FuzzyHandler.}}\sphinxbfcode{\sphinxupquote{nombre\_etiqueta\_in}}}{\emph{self}}{}
Funcion para manejar el cambio de nombre de la etiqueta seleccionada de la entrada actual

\end{fulllineitems}

\index{nombre\_etiqueta\_out() (en el módulo FuzzyHandler)@\spxentry{nombre\_etiqueta\_out()}\spxextra{en el módulo FuzzyHandler}}

\begin{fulllineitems}
\phantomsection\label{\detokenize{codigos/FuzzyHandler:FuzzyHandler.nombre_etiqueta_out}}\pysiglinewithargsret{\sphinxcode{\sphinxupquote{FuzzyHandler.}}\sphinxbfcode{\sphinxupquote{nombre\_etiqueta\_out}}}{\emph{self}}{}
Funcion para manejar el cambio de nombre de la etiqueta seleccionada de la salida actual

\end{fulllineitems}

\index{nombre\_salida() (en el módulo FuzzyHandler)@\spxentry{nombre\_salida()}\spxextra{en el módulo FuzzyHandler}}

\begin{fulllineitems}
\phantomsection\label{\detokenize{codigos/FuzzyHandler:FuzzyHandler.nombre_salida}}\pysiglinewithargsret{\sphinxcode{\sphinxupquote{FuzzyHandler.}}\sphinxbfcode{\sphinxupquote{nombre\_salida}}}{\emph{self}}{}
Funcion para manejar el cambio de nombre de la salida seleccionada

\end{fulllineitems}

\index{numero\_de\_etiquetas\_in() (en el módulo FuzzyHandler)@\spxentry{numero\_de\_etiquetas\_in()}\spxextra{en el módulo FuzzyHandler}}

\begin{fulllineitems}
\phantomsection\label{\detokenize{codigos/FuzzyHandler:FuzzyHandler.numero_de_etiquetas_in}}\pysiglinewithargsret{\sphinxcode{\sphinxupquote{FuzzyHandler.}}\sphinxbfcode{\sphinxupquote{numero\_de\_etiquetas\_in}}}{\emph{self}}{}
Funcion para manejar el numero de etiquetas para la entrada seleccionada

\end{fulllineitems}

\index{numero\_de\_etiquetas\_out() (en el módulo FuzzyHandler)@\spxentry{numero\_de\_etiquetas\_out()}\spxextra{en el módulo FuzzyHandler}}

\begin{fulllineitems}
\phantomsection\label{\detokenize{codigos/FuzzyHandler:FuzzyHandler.numero_de_etiquetas_out}}\pysiglinewithargsret{\sphinxcode{\sphinxupquote{FuzzyHandler.}}\sphinxbfcode{\sphinxupquote{numero\_de\_etiquetas\_out}}}{\emph{self}}{}
Funcion para manejar el numero de etiquetas para la salida seleccionada

\end{fulllineitems}

\index{outputDic\_creator() (en el módulo FuzzyHandler)@\spxentry{outputDic\_creator()}\spxextra{en el módulo FuzzyHandler}}

\begin{fulllineitems}
\phantomsection\label{\detokenize{codigos/FuzzyHandler:FuzzyHandler.outputDic_creator}}\pysiglinewithargsret{\sphinxcode{\sphinxupquote{FuzzyHandler.}}\sphinxbfcode{\sphinxupquote{outputDic\_creator}}}{\emph{self}, \emph{i}}{}
Funcion para crear salidas genericas
\begin{quote}\begin{description}
\item[{Parámetros}] \leavevmode
\sphinxstyleliteralstrong{\sphinxupquote{i}} (\sphinxstyleliteralemphasis{\sphinxupquote{int}}) \textendash{} Numero de salida

\end{description}\end{quote}

\end{fulllineitems}

\index{prueba\_input() (en el módulo FuzzyHandler)@\spxentry{prueba\_input()}\spxextra{en el módulo FuzzyHandler}}

\begin{fulllineitems}
\phantomsection\label{\detokenize{codigos/FuzzyHandler:FuzzyHandler.prueba_input}}\pysiglinewithargsret{\sphinxcode{\sphinxupquote{FuzzyHandler.}}\sphinxbfcode{\sphinxupquote{prueba\_input}}}{\emph{self}}{}
Funcion para la ejecucion del codigo correspondiente a la prueba del controlador

\end{fulllineitems}

\index{rango\_in() (en el módulo FuzzyHandler)@\spxentry{rango\_in()}\spxextra{en el módulo FuzzyHandler}}

\begin{fulllineitems}
\phantomsection\label{\detokenize{codigos/FuzzyHandler:FuzzyHandler.rango_in}}\pysiglinewithargsret{\sphinxcode{\sphinxupquote{FuzzyHandler.}}\sphinxbfcode{\sphinxupquote{rango\_in}}}{\emph{self}}{}
Funcion para manejar el rango para la entrada seleccionada

\end{fulllineitems}

\index{rango\_out() (en el módulo FuzzyHandler)@\spxentry{rango\_out()}\spxextra{en el módulo FuzzyHandler}}

\begin{fulllineitems}
\phantomsection\label{\detokenize{codigos/FuzzyHandler:FuzzyHandler.rango_out}}\pysiglinewithargsret{\sphinxcode{\sphinxupquote{FuzzyHandler.}}\sphinxbfcode{\sphinxupquote{rango\_out}}}{\emph{self}}{}
Funcion para manejar el rango para la salida seleccionada

\end{fulllineitems}

\index{round\_list() (en el módulo FuzzyHandler)@\spxentry{round\_list()}\spxextra{en el módulo FuzzyHandler}}

\begin{fulllineitems}
\phantomsection\label{\detokenize{codigos/FuzzyHandler:FuzzyHandler.round_list}}\pysiglinewithargsret{\sphinxcode{\sphinxupquote{FuzzyHandler.}}\sphinxbfcode{\sphinxupquote{round\_list}}}{\emph{lista}}{}
Funcion para redondear los digitos de una lista

\end{fulllineitems}

\index{rule\_list\_agregar() (en el módulo FuzzyHandler)@\spxentry{rule\_list\_agregar()}\spxextra{en el módulo FuzzyHandler}}

\begin{fulllineitems}
\phantomsection\label{\detokenize{codigos/FuzzyHandler:FuzzyHandler.rule_list_agregar}}\pysiglinewithargsret{\sphinxcode{\sphinxupquote{FuzzyHandler.}}\sphinxbfcode{\sphinxupquote{rule\_list\_agregar}}}{\emph{self}}{}
Funcion para crear una nueva regla a partir de las etiquetas seleccionadas para cada entrada y salida

\end{fulllineitems}

\index{rule\_list\_cambiar() (en el módulo FuzzyHandler)@\spxentry{rule\_list\_cambiar()}\spxextra{en el módulo FuzzyHandler}}

\begin{fulllineitems}
\phantomsection\label{\detokenize{codigos/FuzzyHandler:FuzzyHandler.rule_list_cambiar}}\pysiglinewithargsret{\sphinxcode{\sphinxupquote{FuzzyHandler.}}\sphinxbfcode{\sphinxupquote{rule\_list\_cambiar}}}{\emph{self}}{}
Funcion para modificar una regla

\end{fulllineitems}

\index{rule\_list\_eliminar() (en el módulo FuzzyHandler)@\spxentry{rule\_list\_eliminar()}\spxextra{en el módulo FuzzyHandler}}

\begin{fulllineitems}
\phantomsection\label{\detokenize{codigos/FuzzyHandler:FuzzyHandler.rule_list_eliminar}}\pysiglinewithargsret{\sphinxcode{\sphinxupquote{FuzzyHandler.}}\sphinxbfcode{\sphinxupquote{rule\_list\_eliminar}}}{\emph{self}}{}
Funcion para eliminar una regla

\end{fulllineitems}

\index{rule\_list\_visualizacion() (en el módulo FuzzyHandler)@\spxentry{rule\_list\_visualizacion()}\spxextra{en el módulo FuzzyHandler}}

\begin{fulllineitems}
\phantomsection\label{\detokenize{codigos/FuzzyHandler:FuzzyHandler.rule_list_visualizacion}}\pysiglinewithargsret{\sphinxcode{\sphinxupquote{FuzzyHandler.}}\sphinxbfcode{\sphinxupquote{rule\_list\_visualizacion}}}{\emph{self}}{}
Funcion para mostrar las reglas creadas para el controlador actual en un listWidget

\end{fulllineitems}

\index{seleccion\_entrada() (en el módulo FuzzyHandler)@\spxentry{seleccion\_entrada()}\spxextra{en el módulo FuzzyHandler}}

\begin{fulllineitems}
\phantomsection\label{\detokenize{codigos/FuzzyHandler:FuzzyHandler.seleccion_entrada}}\pysiglinewithargsret{\sphinxcode{\sphinxupquote{FuzzyHandler.}}\sphinxbfcode{\sphinxupquote{seleccion\_entrada}}}{\emph{self}}{}
Funcion para desplegar la informacion de la entrada seleccionada

\end{fulllineitems}

\index{seleccion\_etiqueta\_in() (en el módulo FuzzyHandler)@\spxentry{seleccion\_etiqueta\_in()}\spxextra{en el módulo FuzzyHandler}}

\begin{fulllineitems}
\phantomsection\label{\detokenize{codigos/FuzzyHandler:FuzzyHandler.seleccion_etiqueta_in}}\pysiglinewithargsret{\sphinxcode{\sphinxupquote{FuzzyHandler.}}\sphinxbfcode{\sphinxupquote{seleccion\_etiqueta\_in}}}{\emph{self}}{}
Funcion para desplegar la informacion de la etiqueta seleccionada de la entrada actual

\end{fulllineitems}

\index{seleccion\_etiqueta\_out() (en el módulo FuzzyHandler)@\spxentry{seleccion\_etiqueta\_out()}\spxextra{en el módulo FuzzyHandler}}

\begin{fulllineitems}
\phantomsection\label{\detokenize{codigos/FuzzyHandler:FuzzyHandler.seleccion_etiqueta_out}}\pysiglinewithargsret{\sphinxcode{\sphinxupquote{FuzzyHandler.}}\sphinxbfcode{\sphinxupquote{seleccion\_etiqueta\_out}}}{\emph{self}}{}
Funcion para desplegar la informacion de la etiqueta seleccionada de la salida actual

\end{fulllineitems}

\index{seleccion\_mf\_in() (en el módulo FuzzyHandler)@\spxentry{seleccion\_mf\_in()}\spxextra{en el módulo FuzzyHandler}}

\begin{fulllineitems}
\phantomsection\label{\detokenize{codigos/FuzzyHandler:FuzzyHandler.seleccion_mf_in}}\pysiglinewithargsret{\sphinxcode{\sphinxupquote{FuzzyHandler.}}\sphinxbfcode{\sphinxupquote{seleccion\_mf\_in}}}{\emph{self}}{}
Funcion para manejar el cambio de funcion de membresia para la etiqueta seleccionada

\end{fulllineitems}

\index{seleccion\_mf\_out() (en el módulo FuzzyHandler)@\spxentry{seleccion\_mf\_out()}\spxextra{en el módulo FuzzyHandler}}

\begin{fulllineitems}
\phantomsection\label{\detokenize{codigos/FuzzyHandler:FuzzyHandler.seleccion_mf_out}}\pysiglinewithargsret{\sphinxcode{\sphinxupquote{FuzzyHandler.}}\sphinxbfcode{\sphinxupquote{seleccion\_mf\_out}}}{\emph{self}}{}
Funcion para manejar el cambio de funcion de membresia para la etiqueta seleccionada

\end{fulllineitems}

\index{seleccion\_salida() (en el módulo FuzzyHandler)@\spxentry{seleccion\_salida()}\spxextra{en el módulo FuzzyHandler}}

\begin{fulllineitems}
\phantomsection\label{\detokenize{codigos/FuzzyHandler:FuzzyHandler.seleccion_salida}}\pysiglinewithargsret{\sphinxcode{\sphinxupquote{FuzzyHandler.}}\sphinxbfcode{\sphinxupquote{seleccion\_salida}}}{\emph{self}}{}
Funcion para desplegar la informacion de la salida seleccionada

\end{fulllineitems}

\index{seleccionar\_etiquetas() (en el módulo FuzzyHandler)@\spxentry{seleccionar\_etiquetas()}\spxextra{en el módulo FuzzyHandler}}

\begin{fulllineitems}
\phantomsection\label{\detokenize{codigos/FuzzyHandler:FuzzyHandler.seleccionar_etiquetas}}\pysiglinewithargsret{\sphinxcode{\sphinxupquote{FuzzyHandler.}}\sphinxbfcode{\sphinxupquote{seleccionar\_etiquetas}}}{\emph{self}}{}
Funcion para seleccionar las etiquetas correspodientes a cada entrada/salida de la regla seleccionada

\end{fulllineitems}

\index{show\_esquema() (en el módulo FuzzyHandler)@\spxentry{show\_esquema()}\spxextra{en el módulo FuzzyHandler}}

\begin{fulllineitems}
\phantomsection\label{\detokenize{codigos/FuzzyHandler:FuzzyHandler.show_esquema}}\pysiglinewithargsret{\sphinxcode{\sphinxupquote{FuzzyHandler.}}\sphinxbfcode{\sphinxupquote{show\_esquema}}}{\emph{self}}{}
Funcion para establecer la imagen del esquema de control seleccionado

\end{fulllineitems}



\section{Archivo Handler para la función de simulación de sistemas de control}
\label{\detokenize{codigos/simulacionHandler:archivo-handler-para-la-funcion-de-simulacion-de-sistemas-de-control}}\label{\detokenize{codigos/simulacionHandler::doc}}

\subsection{Archivo que contiene las rutinas de simulación y la clase SimpleThread (QtThread)}
\label{\detokenize{codigos/rutinas_simulacion:archivo-que-contiene-las-rutinas-de-simulacion-y-la-clase-simplethread-qtthread}}\label{\detokenize{codigos/rutinas_simulacion::doc}}\phantomsection\label{\detokenize{codigos/rutinas_simulacion:module-rutinas_simulacion}}\index{rutinas\_simulacion (módulo)@\spxentry{rutinas\_simulacion}\spxextra{módulo}}
Archivo que contiene la clase SimpleThread la cual ejecuta la simulacion de sistemas de control en hilo diferente al principal, esto se realiza de esta forma debido a que la simulacion puede tardar en algunos casos varios segundos, de ejecutarse en el hilo principal presentaria un comportamiento de bloqueo en la ventana principal
\index{SimpleThread (clase en rutinas\_simulacion)@\spxentry{SimpleThread}\spxextra{clase en rutinas\_simulacion}}

\begin{fulllineitems}
\phantomsection\label{\detokenize{codigos/rutinas_simulacion:rutinas_simulacion.SimpleThread}}\pysiglinewithargsret{\sphinxbfcode{\sphinxupquote{class }}\sphinxcode{\sphinxupquote{rutinas\_simulacion.}}\sphinxbfcode{\sphinxupquote{SimpleThread}}}{\emph{window}, \emph{regresar}, \emph{update\_bar}, \emph{error\_gui}, \emph{list\_info}, \emph{parent=None}}{}
Clase para realizar la simulacion de sistemas de control en un hilo diferente al principal
\begin{quote}\begin{description}
\item[{Parámetros}] \leavevmode
\sphinxstyleliteralstrong{\sphinxupquote{QThread}} (\sphinxstyleliteralemphasis{\sphinxupquote{ObjectType}}) \textendash{} Clase para crear un hilo paralelo al principal

\end{description}\end{quote}
\index{\_\_init\_\_() (método de rutinas\_simulacion.SimpleThread)@\spxentry{\_\_init\_\_()}\spxextra{método de rutinas\_simulacion.SimpleThread}}

\begin{fulllineitems}
\phantomsection\label{\detokenize{codigos/rutinas_simulacion:rutinas_simulacion.SimpleThread.__init__}}\pysiglinewithargsret{\sphinxbfcode{\sphinxupquote{\_\_init\_\_}}}{\emph{window}, \emph{regresar}, \emph{update\_bar}, \emph{error\_gui}, \emph{list\_info}, \emph{parent=None}}{}
Constructor para recibir las variables y funciones del hilo principal
\begin{quote}\begin{description}
\item[{Parámetros}] \leavevmode\begin{itemize}
\item {} 
\sphinxstyleliteralstrong{\sphinxupquote{window}} (\sphinxstyleliteralemphasis{\sphinxupquote{object}}) \textendash{} Objeto que contiene a la ventana principal

\item {} 
\sphinxstyleliteralstrong{\sphinxupquote{regresar}} (\sphinxstyleliteralemphasis{\sphinxupquote{function}}) \textendash{} Funcion a la que regresa una vez terminada la simulacion, plot\_final\_results de simulacionHandler.py

\item {} 
\sphinxstyleliteralstrong{\sphinxupquote{update\_bar}} (\sphinxstyleliteralemphasis{\sphinxupquote{function}}) \textendash{} Funcion para actualizar la barra de progreso, update\_progresBar\_function de simulacionHandler.py

\item {} 
\sphinxstyleliteralstrong{\sphinxupquote{error\_gui}} (\sphinxstyleliteralemphasis{\sphinxupquote{function}}) \textendash{} Funcion para mostrar los errores ocurridos durante la simulacion, error\_gui de simulacionHandler.py

\item {} 
\sphinxstyleliteralstrong{\sphinxupquote{list\_info}} (\sphinxstyleliteralemphasis{\sphinxupquote{list}}) \textendash{} Lista con toda la informacion necesaria

\item {} 
\sphinxstyleliteralstrong{\sphinxupquote{parent}} (\sphinxstyleliteralemphasis{\sphinxupquote{NoneType}}\sphinxstyleliteralemphasis{\sphinxupquote{, }}\sphinxstyleliteralemphasis{\sphinxupquote{optional}}) \textendash{} Sin efecto, defaults to None

\end{itemize}

\end{description}\end{quote}

\end{fulllineitems}

\index{run() (método de rutinas\_simulacion.SimpleThread)@\spxentry{run()}\spxextra{método de rutinas\_simulacion.SimpleThread}}

\begin{fulllineitems}
\phantomsection\label{\detokenize{codigos/rutinas_simulacion:rutinas_simulacion.SimpleThread.run}}\pysiglinewithargsret{\sphinxbfcode{\sphinxupquote{run}}}{}{}
Funcion a ejecutar cuando se hace el llamado a self.start()

\end{fulllineitems}

\index{run\_fuzzy() (método de rutinas\_simulacion.SimpleThread)@\spxentry{run\_fuzzy()}\spxextra{método de rutinas\_simulacion.SimpleThread}}

\begin{fulllineitems}
\phantomsection\label{\detokenize{codigos/rutinas_simulacion:rutinas_simulacion.SimpleThread.run_fuzzy}}\pysiglinewithargsret{\sphinxbfcode{\sphinxupquote{run\_fuzzy}}}{}{}
Funcion para realizar la simulacion de sistemas de control de esquemas difusos

\end{fulllineitems}

\index{run\_pid() (método de rutinas\_simulacion.SimpleThread)@\spxentry{run\_pid()}\spxextra{método de rutinas\_simulacion.SimpleThread}}

\begin{fulllineitems}
\phantomsection\label{\detokenize{codigos/rutinas_simulacion:rutinas_simulacion.SimpleThread.run_pid}}\pysiglinewithargsret{\sphinxbfcode{\sphinxupquote{run\_pid}}}{}{}
Funcion para realizar la simulacion de sistemas de control con controlador PID clasico

\end{fulllineitems}

\index{stop() (método de rutinas\_simulacion.SimpleThread)@\spxentry{stop()}\spxextra{método de rutinas\_simulacion.SimpleThread}}

\begin{fulllineitems}
\phantomsection\label{\detokenize{codigos/rutinas_simulacion:rutinas_simulacion.SimpleThread.stop}}\pysiglinewithargsret{\sphinxbfcode{\sphinxupquote{stop}}}{}{}
Funcion para detener el hilo

\end{fulllineitems}


\end{fulllineitems}

\index{controlador\_validator() (en el módulo rutinas\_simulacion)@\spxentry{controlador\_validator()}\spxextra{en el módulo rutinas\_simulacion}}

\begin{fulllineitems}
\phantomsection\label{\detokenize{codigos/rutinas_simulacion:rutinas_simulacion.controlador_validator}}\pysiglinewithargsret{\sphinxcode{\sphinxupquote{rutinas\_simulacion.}}\sphinxbfcode{\sphinxupquote{controlador\_validator}}}{\emph{self}, \emph{esquema}, \emph{InputList}, \emph{OutputList}, \emph{RuleEtiquetas}}{}
Funcion para validar los controladores difusos con respecto al esquema de control seleccionado
\begin{quote}\begin{description}
\item[{Parámetros}] \leavevmode\begin{itemize}
\item {} 
\sphinxstyleliteralstrong{\sphinxupquote{esquema}} (\sphinxstyleliteralemphasis{\sphinxupquote{int}}) \textendash{} Esquema de control seleccionado representado por un valor

\item {} 
\sphinxstyleliteralstrong{\sphinxupquote{InputList}} (\sphinxstyleliteralemphasis{\sphinxupquote{list}}) \textendash{} Lista de entradas

\item {} 
\sphinxstyleliteralstrong{\sphinxupquote{OutputList}} (\sphinxstyleliteralemphasis{\sphinxupquote{list}}) \textendash{} Lista de salidas

\item {} 
\sphinxstyleliteralstrong{\sphinxupquote{RuleEtiquetas}} (\sphinxstyleliteralemphasis{\sphinxupquote{list}}) \textendash{} Lista con set de reglas

\end{itemize}

\end{description}\end{quote}

\end{fulllineitems}

\index{system\_creator\_ss() (en el módulo rutinas\_simulacion)@\spxentry{system\_creator\_ss()}\spxextra{en el módulo rutinas\_simulacion}}

\begin{fulllineitems}
\phantomsection\label{\detokenize{codigos/rutinas_simulacion:rutinas_simulacion.system_creator_ss}}\pysiglinewithargsret{\sphinxcode{\sphinxupquote{rutinas\_simulacion.}}\sphinxbfcode{\sphinxupquote{system\_creator\_ss}}}{\emph{self}, \emph{A}, \emph{B}, \emph{C}, \emph{D}}{}
Funcion para la creacion del sistema a partir de la matriz de estado, matriz de entrada, matriz de salida y la matriz de transmision directa la ecuacion de espacio de estados
\begin{quote}\begin{description}
\item[{Parámetros}] \leavevmode\begin{itemize}
\item {} 
\sphinxstyleliteralstrong{\sphinxupquote{A}} (\sphinxstyleliteralemphasis{\sphinxupquote{list}}) \textendash{} Matriz de estados

\item {} 
\sphinxstyleliteralstrong{\sphinxupquote{B}} (\sphinxstyleliteralemphasis{\sphinxupquote{list}}) \textendash{} Matriz de entrada

\item {} 
\sphinxstyleliteralstrong{\sphinxupquote{C}} (\sphinxstyleliteralemphasis{\sphinxupquote{list}}) \textendash{} Matriz de salida

\item {} 
\sphinxstyleliteralstrong{\sphinxupquote{D}} (\sphinxstyleliteralemphasis{\sphinxupquote{list}}) \textendash{} Matriz de transmision directa

\end{itemize}

\end{description}\end{quote}

\end{fulllineitems}

\index{system\_creator\_tf() (en el módulo rutinas\_simulacion)@\spxentry{system\_creator\_tf()}\spxextra{en el módulo rutinas\_simulacion}}

\begin{fulllineitems}
\phantomsection\label{\detokenize{codigos/rutinas_simulacion:rutinas_simulacion.system_creator_tf}}\pysiglinewithargsret{\sphinxcode{\sphinxupquote{rutinas\_simulacion.}}\sphinxbfcode{\sphinxupquote{system\_creator\_tf}}}{\emph{self}, \emph{numerador}, \emph{denominador}}{}
Funcion para la creacion del sistema a partir de los coeficientes del numerador y del denominador de la funcion de transferencia
\begin{quote}\begin{description}
\item[{Parámetros}] \leavevmode\begin{itemize}
\item {} 
\sphinxstyleliteralstrong{\sphinxupquote{numerador}} (\sphinxstyleliteralemphasis{\sphinxupquote{list}}) \textendash{} Coeficientes del numerador

\item {} 
\sphinxstyleliteralstrong{\sphinxupquote{denominador}} (\sphinxstyleliteralemphasis{\sphinxupquote{list}}) \textendash{} Coeficientes del denominador

\end{itemize}

\end{description}\end{quote}

\end{fulllineitems}



\subsection{Archivo para definir los algoritmos de ajuste del tamaño de paso para los Runge\sphinxhyphen{}kutta explícitos y embebidos}
\label{\detokenize{codigos/rutinas_rk:archivo-para-definir-los-algoritmos-de-ajuste-del-tamano-de-paso-para-los-runge-kutta-explicitos-y-embebidos}}\label{\detokenize{codigos/rutinas_rk::doc}}\phantomsection\label{\detokenize{codigos/rutinas_rk:module-rutinas_rk}}\index{rutinas\_rk (módulo)@\spxentry{rutinas\_rk}\spxextra{módulo}}
Archivo para definir los algoritmos de ajuste del tamaño de paso para los Runge\sphinxhyphen{}kutta explícitos y embebidos, en el caso de los métodos explícitos se utiliza el método de doble paso
\index{rk\_doble\_paso\_adaptativo() (en el módulo rutinas\_rk)@\spxentry{rk\_doble\_paso\_adaptativo()}\spxextra{en el módulo rutinas\_rk}}

\begin{fulllineitems}
\phantomsection\label{\detokenize{codigos/rutinas_rk:rutinas_rk.rk_doble_paso_adaptativo}}\pysiglinewithargsret{\sphinxcode{\sphinxupquote{rutinas\_rk.}}\sphinxbfcode{\sphinxupquote{rk\_doble\_paso\_adaptativo}}}{\emph{systema}, \emph{h\_ant}, \emph{tiempo}, \emph{tbound}, \emph{xVectB}, \emph{entrada}, \emph{metodo}, \emph{ordenq}, \emph{rtol}, \emph{atol}, \emph{max\_step\_increase}, \emph{min\_step\_decrease}, \emph{safety\_factor}}{}
Función para definir y manejar el ajuste del tamaño de paso por el método de doble paso para Runge\sphinxhyphen{}kutta’s explícitos, la función está realizada de forma específica para trabajar con sistemas de control representados con ecuaciones de espacio de estados
\begin{quote}\begin{description}
\item[{Parámetros}] \leavevmode\begin{itemize}
\item {} 
\sphinxstyleliteralstrong{\sphinxupquote{systema}} (\sphinxstyleliteralemphasis{\sphinxupquote{LTI}}) \textendash{} Representación del sistema de control

\item {} 
\sphinxstyleliteralstrong{\sphinxupquote{h\_ant}} (\sphinxstyleliteralemphasis{\sphinxupquote{float}}) \textendash{} Tamaño de paso actual

\item {} 
\sphinxstyleliteralstrong{\sphinxupquote{tiempo}} (\sphinxstyleliteralemphasis{\sphinxupquote{float}}) \textendash{} Tiempo actual

\item {} 
\sphinxstyleliteralstrong{\sphinxupquote{tbound}} (\sphinxstyleliteralemphasis{\sphinxupquote{float}}) \textendash{} Tiempo máximo de simulación

\item {} 
\sphinxstyleliteralstrong{\sphinxupquote{xVectB}} (\sphinxstyleliteralemphasis{\sphinxupquote{numpyArray}}) \textendash{} Vector de estado

\item {} 
\sphinxstyleliteralstrong{\sphinxupquote{entrada}} (\sphinxstyleliteralemphasis{\sphinxupquote{float}}) \textendash{} Valor de entrada al sistema

\item {} 
\sphinxstyleliteralstrong{\sphinxupquote{metodo}} (\sphinxstyleliteralemphasis{\sphinxupquote{function}}) \textendash{} Runge\sphinxhyphen{}Kutta a utilizar: RK2, Rk3, etc.

\item {} 
\sphinxstyleliteralstrong{\sphinxupquote{ordenq}} (\sphinxstyleliteralemphasis{\sphinxupquote{int}}) \textendash{} Orden del método

\item {} 
\sphinxstyleliteralstrong{\sphinxupquote{rtol}} (\sphinxstyleliteralemphasis{\sphinxupquote{float}}) \textendash{} Tolerancia relativa

\item {} 
\sphinxstyleliteralstrong{\sphinxupquote{atol}} (\sphinxstyleliteralemphasis{\sphinxupquote{float}}) \textendash{} Tolerancia absoluta

\item {} 
\sphinxstyleliteralstrong{\sphinxupquote{max\_step\_increase}} (\sphinxstyleliteralemphasis{\sphinxupquote{float}}) \textendash{} Máximo incremento del tamaño de paso

\item {} 
\sphinxstyleliteralstrong{\sphinxupquote{min\_step\_decrease}} (\sphinxstyleliteralemphasis{\sphinxupquote{float}}) \textendash{} Mínimo decremento del tamaño de paso

\item {} 
\sphinxstyleliteralstrong{\sphinxupquote{safety\_factor}} (\sphinxstyleliteralemphasis{\sphinxupquote{float}}) \textendash{} Factor de seguridad

\end{itemize}

\end{description}\end{quote}

\end{fulllineitems}

\index{rk\_embebido\_adaptativo() (en el módulo rutinas\_rk)@\spxentry{rk\_embebido\_adaptativo()}\spxextra{en el módulo rutinas\_rk}}

\begin{fulllineitems}
\phantomsection\label{\detokenize{codigos/rutinas_rk:rutinas_rk.rk_embebido_adaptativo}}\pysiglinewithargsret{\sphinxcode{\sphinxupquote{rutinas\_rk.}}\sphinxbfcode{\sphinxupquote{rk\_embebido\_adaptativo}}}{\emph{systema}, \emph{h\_ant}, \emph{tiempo}, \emph{tbound}, \emph{xVectr}, \emph{entrada}, \emph{metodo}, \emph{ordenq}, \emph{rtol}, \emph{atol}, \emph{max\_step\_increase}, \emph{min\_step\_decrease}, \emph{safety\_factor}}{}
Función para definir y manejar el ajuste del tamaño de paso para Runge\sphinxhyphen{}kutta’s embebidos, la función esta realizada de forma específica para trabajar con sistemas de control representados con ecuaciones de espacio de estados
\begin{quote}\begin{description}
\item[{Parámetros}] \leavevmode\begin{itemize}
\item {} 
\sphinxstyleliteralstrong{\sphinxupquote{systema}} (\sphinxstyleliteralemphasis{\sphinxupquote{LTI}}) \textendash{} Representación del sistema de control

\item {} 
\sphinxstyleliteralstrong{\sphinxupquote{h\_ant}} (\sphinxstyleliteralemphasis{\sphinxupquote{float}}) \textendash{} Tamaño de paso actual

\item {} 
\sphinxstyleliteralstrong{\sphinxupquote{tiempo}} (\sphinxstyleliteralemphasis{\sphinxupquote{float}}) \textendash{} Tiempo actual

\item {} 
\sphinxstyleliteralstrong{\sphinxupquote{tbound}} (\sphinxstyleliteralemphasis{\sphinxupquote{float}}) \textendash{} Tiempo máximo de simulación

\item {} 
\sphinxstyleliteralstrong{\sphinxupquote{xVectB}} (\sphinxstyleliteralemphasis{\sphinxupquote{numpyArray}}) \textendash{} Vector de estado

\item {} 
\sphinxstyleliteralstrong{\sphinxupquote{entrada}} (\sphinxstyleliteralemphasis{\sphinxupquote{float}}) \textendash{} Valor de entrada al sistema

\item {} 
\sphinxstyleliteralstrong{\sphinxupquote{metodo}} (\sphinxstyleliteralemphasis{\sphinxupquote{function}}) \textendash{} Runge\sphinxhyphen{}Kutta a utilizar: DOPRI54, RKF45, etc.

\item {} 
\sphinxstyleliteralstrong{\sphinxupquote{ordenq}} (\sphinxstyleliteralemphasis{\sphinxupquote{int}}) \textendash{} Valor del método de menor orden

\item {} 
\sphinxstyleliteralstrong{\sphinxupquote{rtol}} (\sphinxstyleliteralemphasis{\sphinxupquote{float}}) \textendash{} Tolerancia relativa

\item {} 
\sphinxstyleliteralstrong{\sphinxupquote{atol}} (\sphinxstyleliteralemphasis{\sphinxupquote{float}}) \textendash{} Tolerancia absoluta

\item {} 
\sphinxstyleliteralstrong{\sphinxupquote{max\_step\_increase}} (\sphinxstyleliteralemphasis{\sphinxupquote{float}}) \textendash{} Máximo incremento del tamaño de paso

\item {} 
\sphinxstyleliteralstrong{\sphinxupquote{min\_step\_decrease}} (\sphinxstyleliteralemphasis{\sphinxupquote{float}}) \textendash{} Mínimo decremento del tamaño de paso

\item {} 
\sphinxstyleliteralstrong{\sphinxupquote{safety\_factor}} (\sphinxstyleliteralemphasis{\sphinxupquote{float}}) \textendash{} Factor de seguridad

\end{itemize}

\end{description}\end{quote}

\end{fulllineitems}



\subsection{Archivo para compilar los Runge\sphinxhyphen{}kutta explicitos y embebidos utilizando numba}
\label{\detokenize{codigos/rk_generator:archivo-para-compilar-los-runge-kutta-explicitos-y-embebidos-utilizando-numba}}\label{\detokenize{codigos/rk_generator::doc}}\phantomsection\label{\detokenize{codigos/rk_generator:module-rk_generator}}\index{rk\_generator (módulo)@\spxentry{rk\_generator}\spxextra{módulo}}
Archivo para compilar los Runge\sphinxhyphen{}kutta explicitos y embebidos utilizando numba, los metodos quedan guardados en el archivo: metodos\_RK.cp37\sphinxhyphen{}win32.pyd y pueden ser importados desde el archivo como una funcion de un modulo
\index{SSPRK3() (en el módulo rk\_generator)@\spxentry{SSPRK3()}\spxextra{en el módulo rk\_generator}}

\begin{fulllineitems}
\phantomsection\label{\detokenize{codigos/rk_generator:rk_generator.SSPRK3}}\pysiglinewithargsret{\sphinxcode{\sphinxupquote{rk\_generator.}}\sphinxbfcode{\sphinxupquote{SSPRK3}}}{\emph{A}, \emph{B}, \emph{C}, \emph{D}, \emph{x}, \emph{h}, \emph{inputValue}}{}
Runge\sphinxhyphen{}Kutta con preservado de estabilidad fuerte de orden 3, en el metodo se asumio entrada constante, por lo que se descarta t + h*cs
\begin{quote}\begin{description}
\item[{Parámetros}] \leavevmode\begin{itemize}
\item {} 
\sphinxstyleliteralstrong{\sphinxupquote{A}} (\sphinxstyleliteralemphasis{\sphinxupquote{float64}}\sphinxstyleliteralemphasis{\sphinxupquote{, }}\sphinxstyleliteralemphasis{\sphinxupquote{2d}}\sphinxstyleliteralemphasis{\sphinxupquote{, }}\sphinxstyleliteralemphasis{\sphinxupquote{F}}) \textendash{} Matriz de estados

\item {} 
\sphinxstyleliteralstrong{\sphinxupquote{B}} (\sphinxstyleliteralemphasis{\sphinxupquote{float64}}\sphinxstyleliteralemphasis{\sphinxupquote{, }}\sphinxstyleliteralemphasis{\sphinxupquote{2d}}\sphinxstyleliteralemphasis{\sphinxupquote{, }}\sphinxstyleliteralemphasis{\sphinxupquote{C}}) \textendash{} Matriz de entrada

\item {} 
\sphinxstyleliteralstrong{\sphinxupquote{C}} (\sphinxstyleliteralemphasis{\sphinxupquote{float64}}\sphinxstyleliteralemphasis{\sphinxupquote{, }}\sphinxstyleliteralemphasis{\sphinxupquote{2d}}\sphinxstyleliteralemphasis{\sphinxupquote{, }}\sphinxstyleliteralemphasis{\sphinxupquote{C}}) \textendash{} Matriz de salida

\item {} 
\sphinxstyleliteralstrong{\sphinxupquote{D}} (\sphinxstyleliteralemphasis{\sphinxupquote{float64}}\sphinxstyleliteralemphasis{\sphinxupquote{, }}\sphinxstyleliteralemphasis{\sphinxupquote{2d}}\sphinxstyleliteralemphasis{\sphinxupquote{, }}\sphinxstyleliteralemphasis{\sphinxupquote{C}}) \textendash{} {[}Matriz de transmision directa

\item {} 
\sphinxstyleliteralstrong{\sphinxupquote{x}} (\sphinxstyleliteralemphasis{\sphinxupquote{float64}}\sphinxstyleliteralemphasis{\sphinxupquote{, }}\sphinxstyleliteralemphasis{\sphinxupquote{2d}}\sphinxstyleliteralemphasis{\sphinxupquote{, }}\sphinxstyleliteralemphasis{\sphinxupquote{C}}) \textendash{} Vector de estado

\item {} 
\sphinxstyleliteralstrong{\sphinxupquote{h}} (\sphinxstyleliteralemphasis{\sphinxupquote{float64}}) \textendash{} Tamaño de paso

\item {} 
\sphinxstyleliteralstrong{\sphinxupquote{inputValue}} (\sphinxstyleliteralemphasis{\sphinxupquote{float64}}) \textendash{} Valor de entrada al sistema

\end{itemize}

\end{description}\end{quote}

\end{fulllineitems}

\index{bogacki\_shampine23() (en el módulo rk\_generator)@\spxentry{bogacki\_shampine23()}\spxextra{en el módulo rk\_generator}}

\begin{fulllineitems}
\phantomsection\label{\detokenize{codigos/rk_generator:rk_generator.bogacki_shampine23}}\pysiglinewithargsret{\sphinxcode{\sphinxupquote{rk\_generator.}}\sphinxbfcode{\sphinxupquote{bogacki\_shampine23}}}{\emph{A}, \emph{B}, \emph{C}, \emph{D}, \emph{x}, \emph{h}, \emph{inputValue}}{}
Runge\sphinxhyphen{}Kutta embebido de Bogacki\sphinxhyphen{}Shampine 3(2), la integracion se continua con la salida de orden 3, en el metodo se asumio entrada constante, por lo que se descarta t + h*cs
\begin{quote}\begin{description}
\item[{Parámetros}] \leavevmode\begin{itemize}
\item {} 
\sphinxstyleliteralstrong{\sphinxupquote{A}} (\sphinxstyleliteralemphasis{\sphinxupquote{float64}}\sphinxstyleliteralemphasis{\sphinxupquote{, }}\sphinxstyleliteralemphasis{\sphinxupquote{2d}}\sphinxstyleliteralemphasis{\sphinxupquote{, }}\sphinxstyleliteralemphasis{\sphinxupquote{F}}) \textendash{} Matriz de estados

\item {} 
\sphinxstyleliteralstrong{\sphinxupquote{B}} (\sphinxstyleliteralemphasis{\sphinxupquote{float64}}\sphinxstyleliteralemphasis{\sphinxupquote{, }}\sphinxstyleliteralemphasis{\sphinxupquote{2d}}\sphinxstyleliteralemphasis{\sphinxupquote{, }}\sphinxstyleliteralemphasis{\sphinxupquote{C}}) \textendash{} Matriz de entrada

\item {} 
\sphinxstyleliteralstrong{\sphinxupquote{C}} (\sphinxstyleliteralemphasis{\sphinxupquote{float64}}\sphinxstyleliteralemphasis{\sphinxupquote{, }}\sphinxstyleliteralemphasis{\sphinxupquote{2d}}\sphinxstyleliteralemphasis{\sphinxupquote{, }}\sphinxstyleliteralemphasis{\sphinxupquote{C}}) \textendash{} Matriz de salida

\item {} 
\sphinxstyleliteralstrong{\sphinxupquote{D}} (\sphinxstyleliteralemphasis{\sphinxupquote{float64}}\sphinxstyleliteralemphasis{\sphinxupquote{, }}\sphinxstyleliteralemphasis{\sphinxupquote{2d}}\sphinxstyleliteralemphasis{\sphinxupquote{, }}\sphinxstyleliteralemphasis{\sphinxupquote{C}}) \textendash{} {[}Matriz de transmision directa

\item {} 
\sphinxstyleliteralstrong{\sphinxupquote{x}} (\sphinxstyleliteralemphasis{\sphinxupquote{float64}}\sphinxstyleliteralemphasis{\sphinxupquote{, }}\sphinxstyleliteralemphasis{\sphinxupquote{2d}}\sphinxstyleliteralemphasis{\sphinxupquote{, }}\sphinxstyleliteralemphasis{\sphinxupquote{C}}) \textendash{} Vector de estado

\item {} 
\sphinxstyleliteralstrong{\sphinxupquote{h}} (\sphinxstyleliteralemphasis{\sphinxupquote{float64}}) \textendash{} Tamaño de paso

\item {} 
\sphinxstyleliteralstrong{\sphinxupquote{inputValue}} (\sphinxstyleliteralemphasis{\sphinxupquote{float64}}) \textendash{} Valor de entrada al sistema

\end{itemize}

\end{description}\end{quote}

\end{fulllineitems}

\index{cash\_karp45() (en el módulo rk\_generator)@\spxentry{cash\_karp45()}\spxextra{en el módulo rk\_generator}}

\begin{fulllineitems}
\phantomsection\label{\detokenize{codigos/rk_generator:rk_generator.cash_karp45}}\pysiglinewithargsret{\sphinxcode{\sphinxupquote{rk\_generator.}}\sphinxbfcode{\sphinxupquote{cash\_karp45}}}{\emph{A}, \emph{B}, \emph{C}, \emph{D}, \emph{x}, \emph{h}, \emph{inputValue}}{}
Runge\sphinxhyphen{}Kutta embebido de Cash\sphinxhyphen{}Karp 4(5), la integracion se continua con la salida de orden 4, en el metodo se asumio entrada constante, por lo que se descarta t + h*cs
\begin{quote}\begin{description}
\item[{Parámetros}] \leavevmode\begin{itemize}
\item {} 
\sphinxstyleliteralstrong{\sphinxupquote{A}} (\sphinxstyleliteralemphasis{\sphinxupquote{float64}}\sphinxstyleliteralemphasis{\sphinxupquote{, }}\sphinxstyleliteralemphasis{\sphinxupquote{2d}}\sphinxstyleliteralemphasis{\sphinxupquote{, }}\sphinxstyleliteralemphasis{\sphinxupquote{F}}) \textendash{} Matriz de estados

\item {} 
\sphinxstyleliteralstrong{\sphinxupquote{B}} (\sphinxstyleliteralemphasis{\sphinxupquote{float64}}\sphinxstyleliteralemphasis{\sphinxupquote{, }}\sphinxstyleliteralemphasis{\sphinxupquote{2d}}\sphinxstyleliteralemphasis{\sphinxupquote{, }}\sphinxstyleliteralemphasis{\sphinxupquote{C}}) \textendash{} Matriz de entrada

\item {} 
\sphinxstyleliteralstrong{\sphinxupquote{C}} (\sphinxstyleliteralemphasis{\sphinxupquote{float64}}\sphinxstyleliteralemphasis{\sphinxupquote{, }}\sphinxstyleliteralemphasis{\sphinxupquote{2d}}\sphinxstyleliteralemphasis{\sphinxupquote{, }}\sphinxstyleliteralemphasis{\sphinxupquote{C}}) \textendash{} Matriz de salida

\item {} 
\sphinxstyleliteralstrong{\sphinxupquote{D}} (\sphinxstyleliteralemphasis{\sphinxupquote{float64}}\sphinxstyleliteralemphasis{\sphinxupquote{, }}\sphinxstyleliteralemphasis{\sphinxupquote{2d}}\sphinxstyleliteralemphasis{\sphinxupquote{, }}\sphinxstyleliteralemphasis{\sphinxupquote{C}}) \textendash{} {[}Matriz de transmision directa

\item {} 
\sphinxstyleliteralstrong{\sphinxupquote{x}} (\sphinxstyleliteralemphasis{\sphinxupquote{float64}}\sphinxstyleliteralemphasis{\sphinxupquote{, }}\sphinxstyleliteralemphasis{\sphinxupquote{2d}}\sphinxstyleliteralemphasis{\sphinxupquote{, }}\sphinxstyleliteralemphasis{\sphinxupquote{C}}) \textendash{} Vector de estado

\item {} 
\sphinxstyleliteralstrong{\sphinxupquote{h}} (\sphinxstyleliteralemphasis{\sphinxupquote{float64}}) \textendash{} Tamaño de paso

\item {} 
\sphinxstyleliteralstrong{\sphinxupquote{inputValue}} (\sphinxstyleliteralemphasis{\sphinxupquote{float64}}) \textendash{} Valor de entrada al sistema

\end{itemize}

\end{description}\end{quote}

\end{fulllineitems}

\index{dopri54() (en el módulo rk\_generator)@\spxentry{dopri54()}\spxextra{en el módulo rk\_generator}}

\begin{fulllineitems}
\phantomsection\label{\detokenize{codigos/rk_generator:rk_generator.dopri54}}\pysiglinewithargsret{\sphinxcode{\sphinxupquote{rk\_generator.}}\sphinxbfcode{\sphinxupquote{dopri54}}}{\emph{A}, \emph{B}, \emph{C}, \emph{D}, \emph{x}, \emph{h}, \emph{inputValue}}{}
Runge\sphinxhyphen{}Kutta embebido de Dormand\sphinxhyphen{}Prince 5(4), la integracion se continua con la salida de orden 5, en el metodo se asumio entrada constante, por lo que se descarta t + h*cs
\begin{quote}\begin{description}
\item[{Parámetros}] \leavevmode\begin{itemize}
\item {} 
\sphinxstyleliteralstrong{\sphinxupquote{A}} (\sphinxstyleliteralemphasis{\sphinxupquote{float64}}\sphinxstyleliteralemphasis{\sphinxupquote{, }}\sphinxstyleliteralemphasis{\sphinxupquote{2d}}\sphinxstyleliteralemphasis{\sphinxupquote{, }}\sphinxstyleliteralemphasis{\sphinxupquote{F}}) \textendash{} Matriz de estados

\item {} 
\sphinxstyleliteralstrong{\sphinxupquote{B}} (\sphinxstyleliteralemphasis{\sphinxupquote{float64}}\sphinxstyleliteralemphasis{\sphinxupquote{, }}\sphinxstyleliteralemphasis{\sphinxupquote{2d}}\sphinxstyleliteralemphasis{\sphinxupquote{, }}\sphinxstyleliteralemphasis{\sphinxupquote{C}}) \textendash{} Matriz de entrada

\item {} 
\sphinxstyleliteralstrong{\sphinxupquote{C}} (\sphinxstyleliteralemphasis{\sphinxupquote{float64}}\sphinxstyleliteralemphasis{\sphinxupquote{, }}\sphinxstyleliteralemphasis{\sphinxupquote{2d}}\sphinxstyleliteralemphasis{\sphinxupquote{, }}\sphinxstyleliteralemphasis{\sphinxupquote{C}}) \textendash{} Matriz de salida

\item {} 
\sphinxstyleliteralstrong{\sphinxupquote{D}} (\sphinxstyleliteralemphasis{\sphinxupquote{float64}}\sphinxstyleliteralemphasis{\sphinxupquote{, }}\sphinxstyleliteralemphasis{\sphinxupquote{2d}}\sphinxstyleliteralemphasis{\sphinxupquote{, }}\sphinxstyleliteralemphasis{\sphinxupquote{C}}) \textendash{} {[}Matriz de transmision directa

\item {} 
\sphinxstyleliteralstrong{\sphinxupquote{x}} (\sphinxstyleliteralemphasis{\sphinxupquote{float64}}\sphinxstyleliteralemphasis{\sphinxupquote{, }}\sphinxstyleliteralemphasis{\sphinxupquote{2d}}\sphinxstyleliteralemphasis{\sphinxupquote{, }}\sphinxstyleliteralemphasis{\sphinxupquote{C}}) \textendash{} Vector de estado

\item {} 
\sphinxstyleliteralstrong{\sphinxupquote{h}} (\sphinxstyleliteralemphasis{\sphinxupquote{float64}}) \textendash{} Tamaño de paso

\item {} 
\sphinxstyleliteralstrong{\sphinxupquote{inputValue}} (\sphinxstyleliteralemphasis{\sphinxupquote{float64}}) \textendash{} Valor de entrada al sistema

\end{itemize}

\end{description}\end{quote}

\end{fulllineitems}

\index{fehlberg45() (en el módulo rk\_generator)@\spxentry{fehlberg45()}\spxextra{en el módulo rk\_generator}}

\begin{fulllineitems}
\phantomsection\label{\detokenize{codigos/rk_generator:rk_generator.fehlberg45}}\pysiglinewithargsret{\sphinxcode{\sphinxupquote{rk\_generator.}}\sphinxbfcode{\sphinxupquote{fehlberg45}}}{\emph{A}, \emph{B}, \emph{C}, \emph{D}, \emph{x}, \emph{h}, \emph{inputValue}}{}
Runge\sphinxhyphen{}Kutta embebido de Fehlberg 4(5), la integracion se continua con la salida de orden 4, en el metodo se asumio entrada constante, por lo que se descarta t + h*cs
\begin{quote}\begin{description}
\item[{Parámetros}] \leavevmode\begin{itemize}
\item {} 
\sphinxstyleliteralstrong{\sphinxupquote{A}} (\sphinxstyleliteralemphasis{\sphinxupquote{float64}}\sphinxstyleliteralemphasis{\sphinxupquote{, }}\sphinxstyleliteralemphasis{\sphinxupquote{2d}}\sphinxstyleliteralemphasis{\sphinxupquote{, }}\sphinxstyleliteralemphasis{\sphinxupquote{F}}) \textendash{} Matriz de estados

\item {} 
\sphinxstyleliteralstrong{\sphinxupquote{B}} (\sphinxstyleliteralemphasis{\sphinxupquote{float64}}\sphinxstyleliteralemphasis{\sphinxupquote{, }}\sphinxstyleliteralemphasis{\sphinxupquote{2d}}\sphinxstyleliteralemphasis{\sphinxupquote{, }}\sphinxstyleliteralemphasis{\sphinxupquote{C}}) \textendash{} Matriz de entrada

\item {} 
\sphinxstyleliteralstrong{\sphinxupquote{C}} (\sphinxstyleliteralemphasis{\sphinxupquote{float64}}\sphinxstyleliteralemphasis{\sphinxupquote{, }}\sphinxstyleliteralemphasis{\sphinxupquote{2d}}\sphinxstyleliteralemphasis{\sphinxupquote{, }}\sphinxstyleliteralemphasis{\sphinxupquote{C}}) \textendash{} Matriz de salida

\item {} 
\sphinxstyleliteralstrong{\sphinxupquote{D}} (\sphinxstyleliteralemphasis{\sphinxupquote{float64}}\sphinxstyleliteralemphasis{\sphinxupquote{, }}\sphinxstyleliteralemphasis{\sphinxupquote{2d}}\sphinxstyleliteralemphasis{\sphinxupquote{, }}\sphinxstyleliteralemphasis{\sphinxupquote{C}}) \textendash{} {[}Matriz de transmision directa

\item {} 
\sphinxstyleliteralstrong{\sphinxupquote{x}} (\sphinxstyleliteralemphasis{\sphinxupquote{float64}}\sphinxstyleliteralemphasis{\sphinxupquote{, }}\sphinxstyleliteralemphasis{\sphinxupquote{2d}}\sphinxstyleliteralemphasis{\sphinxupquote{, }}\sphinxstyleliteralemphasis{\sphinxupquote{C}}) \textendash{} Vector de estado

\item {} 
\sphinxstyleliteralstrong{\sphinxupquote{h}} (\sphinxstyleliteralemphasis{\sphinxupquote{float64}}) \textendash{} Tamaño de paso

\item {} 
\sphinxstyleliteralstrong{\sphinxupquote{inputValue}} (\sphinxstyleliteralemphasis{\sphinxupquote{float64}}) \textendash{} Valor de entrada al sistema

\end{itemize}

\end{description}\end{quote}

\end{fulllineitems}

\index{heun3() (en el módulo rk\_generator)@\spxentry{heun3()}\spxextra{en el módulo rk\_generator}}

\begin{fulllineitems}
\phantomsection\label{\detokenize{codigos/rk_generator:rk_generator.heun3}}\pysiglinewithargsret{\sphinxcode{\sphinxupquote{rk\_generator.}}\sphinxbfcode{\sphinxupquote{heun3}}}{\emph{A}, \emph{B}, \emph{C}, \emph{D}, \emph{x}, \emph{h}, \emph{inputValue}}{}
Runge\sphinxhyphen{}Kutta Heun de orden 3, en el metodo se asumio entrada constante, por lo que se descarta t + h*cs
\begin{quote}\begin{description}
\item[{Parámetros}] \leavevmode\begin{itemize}
\item {} 
\sphinxstyleliteralstrong{\sphinxupquote{A}} (\sphinxstyleliteralemphasis{\sphinxupquote{float64}}\sphinxstyleliteralemphasis{\sphinxupquote{, }}\sphinxstyleliteralemphasis{\sphinxupquote{2d}}\sphinxstyleliteralemphasis{\sphinxupquote{, }}\sphinxstyleliteralemphasis{\sphinxupquote{F}}) \textendash{} Matriz de estados

\item {} 
\sphinxstyleliteralstrong{\sphinxupquote{B}} (\sphinxstyleliteralemphasis{\sphinxupquote{float64}}\sphinxstyleliteralemphasis{\sphinxupquote{, }}\sphinxstyleliteralemphasis{\sphinxupquote{2d}}\sphinxstyleliteralemphasis{\sphinxupquote{, }}\sphinxstyleliteralemphasis{\sphinxupquote{C}}) \textendash{} Matriz de entrada

\item {} 
\sphinxstyleliteralstrong{\sphinxupquote{C}} (\sphinxstyleliteralemphasis{\sphinxupquote{float64}}\sphinxstyleliteralemphasis{\sphinxupquote{, }}\sphinxstyleliteralemphasis{\sphinxupquote{2d}}\sphinxstyleliteralemphasis{\sphinxupquote{, }}\sphinxstyleliteralemphasis{\sphinxupquote{C}}) \textendash{} Matriz de salida

\item {} 
\sphinxstyleliteralstrong{\sphinxupquote{D}} (\sphinxstyleliteralemphasis{\sphinxupquote{float64}}\sphinxstyleliteralemphasis{\sphinxupquote{, }}\sphinxstyleliteralemphasis{\sphinxupquote{2d}}\sphinxstyleliteralemphasis{\sphinxupquote{, }}\sphinxstyleliteralemphasis{\sphinxupquote{C}}) \textendash{} {[}Matriz de transmision directa

\item {} 
\sphinxstyleliteralstrong{\sphinxupquote{x}} (\sphinxstyleliteralemphasis{\sphinxupquote{float64}}\sphinxstyleliteralemphasis{\sphinxupquote{, }}\sphinxstyleliteralemphasis{\sphinxupquote{2d}}\sphinxstyleliteralemphasis{\sphinxupquote{, }}\sphinxstyleliteralemphasis{\sphinxupquote{C}}) \textendash{} Vector de estado

\item {} 
\sphinxstyleliteralstrong{\sphinxupquote{h}} (\sphinxstyleliteralemphasis{\sphinxupquote{float64}}) \textendash{} Tamaño de paso

\item {} 
\sphinxstyleliteralstrong{\sphinxupquote{inputValue}} (\sphinxstyleliteralemphasis{\sphinxupquote{float64}}) \textendash{} Valor de entrada al sistema

\end{itemize}

\end{description}\end{quote}

\end{fulllineitems}

\index{norm() (en el módulo rk\_generator)@\spxentry{norm()}\spxextra{en el módulo rk\_generator}}

\begin{fulllineitems}
\phantomsection\label{\detokenize{codigos/rk_generator:rk_generator.norm}}\pysiglinewithargsret{\sphinxcode{\sphinxupquote{rk\_generator.}}\sphinxbfcode{\sphinxupquote{norm}}}{\emph{x}}{}
Función para calcular la norma RMS de un vector. Función tomada de SciPy
\begin{quote}\begin{description}
\item[{Parámetros}] \leavevmode
\sphinxstyleliteralstrong{\sphinxupquote{x}} (\sphinxstyleliteralemphasis{\sphinxupquote{numpyArray}}) \textendash{} Vector

\end{description}\end{quote}

\end{fulllineitems}

\index{ralston3() (en el módulo rk\_generator)@\spxentry{ralston3()}\spxextra{en el módulo rk\_generator}}

\begin{fulllineitems}
\phantomsection\label{\detokenize{codigos/rk_generator:rk_generator.ralston3}}\pysiglinewithargsret{\sphinxcode{\sphinxupquote{rk\_generator.}}\sphinxbfcode{\sphinxupquote{ralston3}}}{\emph{A}, \emph{B}, \emph{C}, \emph{D}, \emph{x}, \emph{h}, \emph{inputValue}}{}
Runge\sphinxhyphen{}Kutta Ralston de orden 3, en el metodo se asumio entrada constante, por lo que se descarta t + h*cs
\begin{quote}\begin{description}
\item[{Parámetros}] \leavevmode\begin{itemize}
\item {} 
\sphinxstyleliteralstrong{\sphinxupquote{A}} (\sphinxstyleliteralemphasis{\sphinxupquote{float64}}\sphinxstyleliteralemphasis{\sphinxupquote{, }}\sphinxstyleliteralemphasis{\sphinxupquote{2d}}\sphinxstyleliteralemphasis{\sphinxupquote{, }}\sphinxstyleliteralemphasis{\sphinxupquote{F}}) \textendash{} Matriz de estados

\item {} 
\sphinxstyleliteralstrong{\sphinxupquote{B}} (\sphinxstyleliteralemphasis{\sphinxupquote{float64}}\sphinxstyleliteralemphasis{\sphinxupquote{, }}\sphinxstyleliteralemphasis{\sphinxupquote{2d}}\sphinxstyleliteralemphasis{\sphinxupquote{, }}\sphinxstyleliteralemphasis{\sphinxupquote{C}}) \textendash{} Matriz de entrada

\item {} 
\sphinxstyleliteralstrong{\sphinxupquote{C}} (\sphinxstyleliteralemphasis{\sphinxupquote{float64}}\sphinxstyleliteralemphasis{\sphinxupquote{, }}\sphinxstyleliteralemphasis{\sphinxupquote{2d}}\sphinxstyleliteralemphasis{\sphinxupquote{, }}\sphinxstyleliteralemphasis{\sphinxupquote{C}}) \textendash{} Matriz de salida

\item {} 
\sphinxstyleliteralstrong{\sphinxupquote{D}} (\sphinxstyleliteralemphasis{\sphinxupquote{float64}}\sphinxstyleliteralemphasis{\sphinxupquote{, }}\sphinxstyleliteralemphasis{\sphinxupquote{2d}}\sphinxstyleliteralemphasis{\sphinxupquote{, }}\sphinxstyleliteralemphasis{\sphinxupquote{C}}) \textendash{} {[}Matriz de transmision directa

\item {} 
\sphinxstyleliteralstrong{\sphinxupquote{x}} (\sphinxstyleliteralemphasis{\sphinxupquote{float64}}\sphinxstyleliteralemphasis{\sphinxupquote{, }}\sphinxstyleliteralemphasis{\sphinxupquote{2d}}\sphinxstyleliteralemphasis{\sphinxupquote{, }}\sphinxstyleliteralemphasis{\sphinxupquote{C}}) \textendash{} Vector de estado

\item {} 
\sphinxstyleliteralstrong{\sphinxupquote{h}} (\sphinxstyleliteralemphasis{\sphinxupquote{float64}}) \textendash{} Tamaño de paso

\item {} 
\sphinxstyleliteralstrong{\sphinxupquote{inputValue}} (\sphinxstyleliteralemphasis{\sphinxupquote{float64}}) \textendash{} Valor de entrada al sistema

\end{itemize}

\end{description}\end{quote}

\end{fulllineitems}

\index{ralston4() (en el módulo rk\_generator)@\spxentry{ralston4()}\spxextra{en el módulo rk\_generator}}

\begin{fulllineitems}
\phantomsection\label{\detokenize{codigos/rk_generator:rk_generator.ralston4}}\pysiglinewithargsret{\sphinxcode{\sphinxupquote{rk\_generator.}}\sphinxbfcode{\sphinxupquote{ralston4}}}{\emph{A}, \emph{B}, \emph{C}, \emph{D}, \emph{x}, \emph{h}, \emph{inputValue}}{}
Runge\sphinxhyphen{}Kutta Ralston con minimo error de truncamiento de orden 4, en el metodo se asumio entrada constante, por lo que se descarta t + h*cs
\begin{quote}\begin{description}
\item[{Parámetros}] \leavevmode\begin{itemize}
\item {} 
\sphinxstyleliteralstrong{\sphinxupquote{A}} (\sphinxstyleliteralemphasis{\sphinxupquote{float64}}\sphinxstyleliteralemphasis{\sphinxupquote{, }}\sphinxstyleliteralemphasis{\sphinxupquote{2d}}\sphinxstyleliteralemphasis{\sphinxupquote{, }}\sphinxstyleliteralemphasis{\sphinxupquote{F}}) \textendash{} Matriz de estados

\item {} 
\sphinxstyleliteralstrong{\sphinxupquote{B}} (\sphinxstyleliteralemphasis{\sphinxupquote{float64}}\sphinxstyleliteralemphasis{\sphinxupquote{, }}\sphinxstyleliteralemphasis{\sphinxupquote{2d}}\sphinxstyleliteralemphasis{\sphinxupquote{, }}\sphinxstyleliteralemphasis{\sphinxupquote{C}}) \textendash{} Matriz de entrada

\item {} 
\sphinxstyleliteralstrong{\sphinxupquote{C}} (\sphinxstyleliteralemphasis{\sphinxupquote{float64}}\sphinxstyleliteralemphasis{\sphinxupquote{, }}\sphinxstyleliteralemphasis{\sphinxupquote{2d}}\sphinxstyleliteralemphasis{\sphinxupquote{, }}\sphinxstyleliteralemphasis{\sphinxupquote{C}}) \textendash{} Matriz de salida

\item {} 
\sphinxstyleliteralstrong{\sphinxupquote{D}} (\sphinxstyleliteralemphasis{\sphinxupquote{float64}}\sphinxstyleliteralemphasis{\sphinxupquote{, }}\sphinxstyleliteralemphasis{\sphinxupquote{2d}}\sphinxstyleliteralemphasis{\sphinxupquote{, }}\sphinxstyleliteralemphasis{\sphinxupquote{C}}) \textendash{} {[}Matriz de transmision directa

\item {} 
\sphinxstyleliteralstrong{\sphinxupquote{x}} (\sphinxstyleliteralemphasis{\sphinxupquote{float64}}\sphinxstyleliteralemphasis{\sphinxupquote{, }}\sphinxstyleliteralemphasis{\sphinxupquote{2d}}\sphinxstyleliteralemphasis{\sphinxupquote{, }}\sphinxstyleliteralemphasis{\sphinxupquote{C}}) \textendash{} Vector de estado

\item {} 
\sphinxstyleliteralstrong{\sphinxupquote{h}} (\sphinxstyleliteralemphasis{\sphinxupquote{float64}}) \textendash{} Tamaño de paso

\item {} 
\sphinxstyleliteralstrong{\sphinxupquote{inputValue}} (\sphinxstyleliteralemphasis{\sphinxupquote{float64}}) \textendash{} Valor de entrada al sistema

\end{itemize}

\end{description}\end{quote}

\end{fulllineitems}

\index{runge\_kutta2() (en el módulo rk\_generator)@\spxentry{runge\_kutta2()}\spxextra{en el módulo rk\_generator}}

\begin{fulllineitems}
\phantomsection\label{\detokenize{codigos/rk_generator:rk_generator.runge_kutta2}}\pysiglinewithargsret{\sphinxcode{\sphinxupquote{rk\_generator.}}\sphinxbfcode{\sphinxupquote{runge\_kutta2}}}{\emph{A}, \emph{B}, \emph{C}, \emph{D}, \emph{x}, \emph{h}, \emph{inputValue}}{}
Runge\sphinxhyphen{}Kutta de orden 2, en el metodo se asumio entrada constante, por lo que se descarta t + h*cs
\begin{quote}\begin{description}
\item[{Parámetros}] \leavevmode\begin{itemize}
\item {} 
\sphinxstyleliteralstrong{\sphinxupquote{A}} (\sphinxstyleliteralemphasis{\sphinxupquote{float64}}\sphinxstyleliteralemphasis{\sphinxupquote{, }}\sphinxstyleliteralemphasis{\sphinxupquote{2d}}\sphinxstyleliteralemphasis{\sphinxupquote{, }}\sphinxstyleliteralemphasis{\sphinxupquote{F}}) \textendash{} Matriz de estados

\item {} 
\sphinxstyleliteralstrong{\sphinxupquote{B}} (\sphinxstyleliteralemphasis{\sphinxupquote{float64}}\sphinxstyleliteralemphasis{\sphinxupquote{, }}\sphinxstyleliteralemphasis{\sphinxupquote{2d}}\sphinxstyleliteralemphasis{\sphinxupquote{, }}\sphinxstyleliteralemphasis{\sphinxupquote{C}}) \textendash{} Matriz de entrada

\item {} 
\sphinxstyleliteralstrong{\sphinxupquote{C}} (\sphinxstyleliteralemphasis{\sphinxupquote{float64}}\sphinxstyleliteralemphasis{\sphinxupquote{, }}\sphinxstyleliteralemphasis{\sphinxupquote{2d}}\sphinxstyleliteralemphasis{\sphinxupquote{, }}\sphinxstyleliteralemphasis{\sphinxupquote{C}}) \textendash{} Matriz de salida

\item {} 
\sphinxstyleliteralstrong{\sphinxupquote{D}} (\sphinxstyleliteralemphasis{\sphinxupquote{float64}}\sphinxstyleliteralemphasis{\sphinxupquote{, }}\sphinxstyleliteralemphasis{\sphinxupquote{2d}}\sphinxstyleliteralemphasis{\sphinxupquote{, }}\sphinxstyleliteralemphasis{\sphinxupquote{C}}) \textendash{} {[}Matriz de transmision directa

\item {} 
\sphinxstyleliteralstrong{\sphinxupquote{x}} (\sphinxstyleliteralemphasis{\sphinxupquote{float64}}\sphinxstyleliteralemphasis{\sphinxupquote{, }}\sphinxstyleliteralemphasis{\sphinxupquote{2d}}\sphinxstyleliteralemphasis{\sphinxupquote{, }}\sphinxstyleliteralemphasis{\sphinxupquote{C}}) \textendash{} Vector de estado

\item {} 
\sphinxstyleliteralstrong{\sphinxupquote{h}} (\sphinxstyleliteralemphasis{\sphinxupquote{float64}}) \textendash{} Tamaño de paso

\item {} 
\sphinxstyleliteralstrong{\sphinxupquote{inputValue}} (\sphinxstyleliteralemphasis{\sphinxupquote{float64}}) \textendash{} Valor de entrada al sistema

\end{itemize}

\end{description}\end{quote}

\end{fulllineitems}

\index{runge\_kutta3() (en el módulo rk\_generator)@\spxentry{runge\_kutta3()}\spxextra{en el módulo rk\_generator}}

\begin{fulllineitems}
\phantomsection\label{\detokenize{codigos/rk_generator:rk_generator.runge_kutta3}}\pysiglinewithargsret{\sphinxcode{\sphinxupquote{rk\_generator.}}\sphinxbfcode{\sphinxupquote{runge\_kutta3}}}{\emph{A}, \emph{B}, \emph{C}, \emph{D}, \emph{x}, \emph{h}, \emph{inputValue}}{}
Runge\sphinxhyphen{}Kutta de orden 3, en el metodo se asumio entrada constante, por lo que se descarta t + h*cs
\begin{quote}\begin{description}
\item[{Parámetros}] \leavevmode\begin{itemize}
\item {} 
\sphinxstyleliteralstrong{\sphinxupquote{A}} (\sphinxstyleliteralemphasis{\sphinxupquote{float64}}\sphinxstyleliteralemphasis{\sphinxupquote{, }}\sphinxstyleliteralemphasis{\sphinxupquote{2d}}\sphinxstyleliteralemphasis{\sphinxupquote{, }}\sphinxstyleliteralemphasis{\sphinxupquote{F}}) \textendash{} Matriz de estados

\item {} 
\sphinxstyleliteralstrong{\sphinxupquote{B}} (\sphinxstyleliteralemphasis{\sphinxupquote{float64}}\sphinxstyleliteralemphasis{\sphinxupquote{, }}\sphinxstyleliteralemphasis{\sphinxupquote{2d}}\sphinxstyleliteralemphasis{\sphinxupquote{, }}\sphinxstyleliteralemphasis{\sphinxupquote{C}}) \textendash{} Matriz de entrada

\item {} 
\sphinxstyleliteralstrong{\sphinxupquote{C}} (\sphinxstyleliteralemphasis{\sphinxupquote{float64}}\sphinxstyleliteralemphasis{\sphinxupquote{, }}\sphinxstyleliteralemphasis{\sphinxupquote{2d}}\sphinxstyleliteralemphasis{\sphinxupquote{, }}\sphinxstyleliteralemphasis{\sphinxupquote{C}}) \textendash{} Matriz de salida

\item {} 
\sphinxstyleliteralstrong{\sphinxupquote{D}} (\sphinxstyleliteralemphasis{\sphinxupquote{float64}}\sphinxstyleliteralemphasis{\sphinxupquote{, }}\sphinxstyleliteralemphasis{\sphinxupquote{2d}}\sphinxstyleliteralemphasis{\sphinxupquote{, }}\sphinxstyleliteralemphasis{\sphinxupquote{C}}) \textendash{} {[}Matriz de transmision directa

\item {} 
\sphinxstyleliteralstrong{\sphinxupquote{x}} (\sphinxstyleliteralemphasis{\sphinxupquote{float64}}\sphinxstyleliteralemphasis{\sphinxupquote{, }}\sphinxstyleliteralemphasis{\sphinxupquote{2d}}\sphinxstyleliteralemphasis{\sphinxupquote{, }}\sphinxstyleliteralemphasis{\sphinxupquote{C}}) \textendash{} Vector de estado

\item {} 
\sphinxstyleliteralstrong{\sphinxupquote{h}} (\sphinxstyleliteralemphasis{\sphinxupquote{float64}}) \textendash{} Tamaño de paso

\item {} 
\sphinxstyleliteralstrong{\sphinxupquote{inputValue}} (\sphinxstyleliteralemphasis{\sphinxupquote{float64}}) \textendash{} Valor de entrada al sistema

\end{itemize}

\end{description}\end{quote}

\end{fulllineitems}

\index{runge\_kutta4() (en el módulo rk\_generator)@\spxentry{runge\_kutta4()}\spxextra{en el módulo rk\_generator}}

\begin{fulllineitems}
\phantomsection\label{\detokenize{codigos/rk_generator:rk_generator.runge_kutta4}}\pysiglinewithargsret{\sphinxcode{\sphinxupquote{rk\_generator.}}\sphinxbfcode{\sphinxupquote{runge\_kutta4}}}{\emph{A}, \emph{B}, \emph{C}, \emph{D}, \emph{x}, \emph{h}, \emph{inputValue}}{}
Runge\sphinxhyphen{}Kutta de orden 4, en el metodo se asumio entrada constante, por lo que se descarta t + h*cs
\begin{quote}\begin{description}
\item[{Parámetros}] \leavevmode\begin{itemize}
\item {} 
\sphinxstyleliteralstrong{\sphinxupquote{A}} (\sphinxstyleliteralemphasis{\sphinxupquote{float64}}\sphinxstyleliteralemphasis{\sphinxupquote{, }}\sphinxstyleliteralemphasis{\sphinxupquote{2d}}\sphinxstyleliteralemphasis{\sphinxupquote{, }}\sphinxstyleliteralemphasis{\sphinxupquote{F}}) \textendash{} Matriz de estados

\item {} 
\sphinxstyleliteralstrong{\sphinxupquote{B}} (\sphinxstyleliteralemphasis{\sphinxupquote{float64}}\sphinxstyleliteralemphasis{\sphinxupquote{, }}\sphinxstyleliteralemphasis{\sphinxupquote{2d}}\sphinxstyleliteralemphasis{\sphinxupquote{, }}\sphinxstyleliteralemphasis{\sphinxupquote{C}}) \textendash{} Matriz de entrada

\item {} 
\sphinxstyleliteralstrong{\sphinxupquote{C}} (\sphinxstyleliteralemphasis{\sphinxupquote{float64}}\sphinxstyleliteralemphasis{\sphinxupquote{, }}\sphinxstyleliteralemphasis{\sphinxupquote{2d}}\sphinxstyleliteralemphasis{\sphinxupquote{, }}\sphinxstyleliteralemphasis{\sphinxupquote{C}}) \textendash{} Matriz de salida

\item {} 
\sphinxstyleliteralstrong{\sphinxupquote{D}} (\sphinxstyleliteralemphasis{\sphinxupquote{float64}}\sphinxstyleliteralemphasis{\sphinxupquote{, }}\sphinxstyleliteralemphasis{\sphinxupquote{2d}}\sphinxstyleliteralemphasis{\sphinxupquote{, }}\sphinxstyleliteralemphasis{\sphinxupquote{C}}) \textendash{} {[}Matriz de transmision directa

\item {} 
\sphinxstyleliteralstrong{\sphinxupquote{x}} (\sphinxstyleliteralemphasis{\sphinxupquote{float64}}\sphinxstyleliteralemphasis{\sphinxupquote{, }}\sphinxstyleliteralemphasis{\sphinxupquote{2d}}\sphinxstyleliteralemphasis{\sphinxupquote{, }}\sphinxstyleliteralemphasis{\sphinxupquote{C}}) \textendash{} Vector de estado

\item {} 
\sphinxstyleliteralstrong{\sphinxupquote{h}} (\sphinxstyleliteralemphasis{\sphinxupquote{float64}}) \textendash{} Tamaño de paso

\item {} 
\sphinxstyleliteralstrong{\sphinxupquote{inputValue}} (\sphinxstyleliteralemphasis{\sphinxupquote{float64}}) \textendash{} Valor de entrada al sistema

\end{itemize}

\end{description}\end{quote}

\end{fulllineitems}

\index{runge\_kutta5() (en el módulo rk\_generator)@\spxentry{runge\_kutta5()}\spxextra{en el módulo rk\_generator}}

\begin{fulllineitems}
\phantomsection\label{\detokenize{codigos/rk_generator:rk_generator.runge_kutta5}}\pysiglinewithargsret{\sphinxcode{\sphinxupquote{rk\_generator.}}\sphinxbfcode{\sphinxupquote{runge\_kutta5}}}{\emph{A}, \emph{B}, \emph{C}, \emph{D}, \emph{x}, \emph{h}, \emph{inputValue}}{}
Runge\sphinxhyphen{}Kutta de orden 5, en el metodo se asumio entrada constante, por lo que se descarta t + h*cs
\begin{quote}\begin{description}
\item[{Parámetros}] \leavevmode\begin{itemize}
\item {} 
\sphinxstyleliteralstrong{\sphinxupquote{A}} (\sphinxstyleliteralemphasis{\sphinxupquote{float64}}\sphinxstyleliteralemphasis{\sphinxupquote{, }}\sphinxstyleliteralemphasis{\sphinxupquote{2d}}\sphinxstyleliteralemphasis{\sphinxupquote{, }}\sphinxstyleliteralemphasis{\sphinxupquote{F}}) \textendash{} Matriz de estados

\item {} 
\sphinxstyleliteralstrong{\sphinxupquote{B}} (\sphinxstyleliteralemphasis{\sphinxupquote{float64}}\sphinxstyleliteralemphasis{\sphinxupquote{, }}\sphinxstyleliteralemphasis{\sphinxupquote{2d}}\sphinxstyleliteralemphasis{\sphinxupquote{, }}\sphinxstyleliteralemphasis{\sphinxupquote{C}}) \textendash{} Matriz de entrada

\item {} 
\sphinxstyleliteralstrong{\sphinxupquote{C}} (\sphinxstyleliteralemphasis{\sphinxupquote{float64}}\sphinxstyleliteralemphasis{\sphinxupquote{, }}\sphinxstyleliteralemphasis{\sphinxupquote{2d}}\sphinxstyleliteralemphasis{\sphinxupquote{, }}\sphinxstyleliteralemphasis{\sphinxupquote{C}}) \textendash{} Matriz de salida

\item {} 
\sphinxstyleliteralstrong{\sphinxupquote{D}} (\sphinxstyleliteralemphasis{\sphinxupquote{float64}}\sphinxstyleliteralemphasis{\sphinxupquote{, }}\sphinxstyleliteralemphasis{\sphinxupquote{2d}}\sphinxstyleliteralemphasis{\sphinxupquote{, }}\sphinxstyleliteralemphasis{\sphinxupquote{C}}) \textendash{} {[}Matriz de transmision directa

\item {} 
\sphinxstyleliteralstrong{\sphinxupquote{x}} (\sphinxstyleliteralemphasis{\sphinxupquote{float64}}\sphinxstyleliteralemphasis{\sphinxupquote{, }}\sphinxstyleliteralemphasis{\sphinxupquote{2d}}\sphinxstyleliteralemphasis{\sphinxupquote{, }}\sphinxstyleliteralemphasis{\sphinxupquote{C}}) \textendash{} Vector de estado

\item {} 
\sphinxstyleliteralstrong{\sphinxupquote{h}} (\sphinxstyleliteralemphasis{\sphinxupquote{float64}}) \textendash{} Tamaño de paso

\item {} 
\sphinxstyleliteralstrong{\sphinxupquote{inputValue}} (\sphinxstyleliteralemphasis{\sphinxupquote{float64}}) \textendash{} Valor de entrada al sistema

\end{itemize}

\end{description}\end{quote}

\end{fulllineitems}

\index{tres\_octavos4() (en el módulo rk\_generator)@\spxentry{tres\_octavos4()}\spxextra{en el módulo rk\_generator}}

\begin{fulllineitems}
\phantomsection\label{\detokenize{codigos/rk_generator:rk_generator.tres_octavos4}}\pysiglinewithargsret{\sphinxcode{\sphinxupquote{rk\_generator.}}\sphinxbfcode{\sphinxupquote{tres\_octavos4}}}{\emph{A}, \emph{B}, \emph{C}, \emph{D}, \emph{x}, \emph{h}, \emph{inputValue}}{}
Runge\sphinxhyphen{}Kutta 3/8 de orden 4, en el metodo se asumio entrada constante, por lo que se descarta t + h*cs
\begin{quote}\begin{description}
\item[{Parámetros}] \leavevmode\begin{itemize}
\item {} 
\sphinxstyleliteralstrong{\sphinxupquote{A}} (\sphinxstyleliteralemphasis{\sphinxupquote{float64}}\sphinxstyleliteralemphasis{\sphinxupquote{, }}\sphinxstyleliteralemphasis{\sphinxupquote{2d}}\sphinxstyleliteralemphasis{\sphinxupquote{, }}\sphinxstyleliteralemphasis{\sphinxupquote{F}}) \textendash{} Matriz de estados

\item {} 
\sphinxstyleliteralstrong{\sphinxupquote{B}} (\sphinxstyleliteralemphasis{\sphinxupquote{float64}}\sphinxstyleliteralemphasis{\sphinxupquote{, }}\sphinxstyleliteralemphasis{\sphinxupquote{2d}}\sphinxstyleliteralemphasis{\sphinxupquote{, }}\sphinxstyleliteralemphasis{\sphinxupquote{C}}) \textendash{} Matriz de entrada

\item {} 
\sphinxstyleliteralstrong{\sphinxupquote{C}} (\sphinxstyleliteralemphasis{\sphinxupquote{float64}}\sphinxstyleliteralemphasis{\sphinxupquote{, }}\sphinxstyleliteralemphasis{\sphinxupquote{2d}}\sphinxstyleliteralemphasis{\sphinxupquote{, }}\sphinxstyleliteralemphasis{\sphinxupquote{C}}) \textendash{} Matriz de salida

\item {} 
\sphinxstyleliteralstrong{\sphinxupquote{D}} (\sphinxstyleliteralemphasis{\sphinxupquote{float64}}\sphinxstyleliteralemphasis{\sphinxupquote{, }}\sphinxstyleliteralemphasis{\sphinxupquote{2d}}\sphinxstyleliteralemphasis{\sphinxupquote{, }}\sphinxstyleliteralemphasis{\sphinxupquote{C}}) \textendash{} {[}Matriz de transmision directa

\item {} 
\sphinxstyleliteralstrong{\sphinxupquote{x}} (\sphinxstyleliteralemphasis{\sphinxupquote{float64}}\sphinxstyleliteralemphasis{\sphinxupquote{, }}\sphinxstyleliteralemphasis{\sphinxupquote{2d}}\sphinxstyleliteralemphasis{\sphinxupquote{, }}\sphinxstyleliteralemphasis{\sphinxupquote{C}}) \textendash{} Vector de estado

\item {} 
\sphinxstyleliteralstrong{\sphinxupquote{h}} (\sphinxstyleliteralemphasis{\sphinxupquote{float64}}) \textendash{} Tamaño de paso

\item {} 
\sphinxstyleliteralstrong{\sphinxupquote{inputValue}} (\sphinxstyleliteralemphasis{\sphinxupquote{float64}}) \textendash{} Valor de entrada al sistema

\end{itemize}

\end{description}\end{quote}

\end{fulllineitems}



\subsection{Archivo para compilar las funciones encargadas de la simulación en tiempo discreto utilizando numba}
\label{\detokenize{codigos/discreto_generator:archivo-para-compilar-las-funciones-encargadas-de-la-simulacion-en-tiempo-discreto-utilizando-numba}}\label{\detokenize{codigos/discreto_generator::doc}}\phantomsection\label{\detokenize{codigos/discreto_generator:module-discreto_generator}}\index{discreto\_generator (módulo)@\spxentry{discreto\_generator}\spxextra{módulo}}
Archivo para compilar las funciones encargadas de la simulacion en tiempo discreto utilizando numba, las funciones quedan guardadas en el archivo: discreto\_sim.cp37\sphinxhyphen{}win32.pyd y pueden ser importadas desde el archivo como una funcion de un modulo
\index{PID\_discreto() (en el módulo discreto\_generator)@\spxentry{PID\_discreto()}\spxextra{en el módulo discreto\_generator}}

\begin{fulllineitems}
\phantomsection\label{\detokenize{codigos/discreto_generator:discreto_generator.PID_discreto}}\pysiglinewithargsret{\sphinxcode{\sphinxupquote{discreto\_generator.}}\sphinxbfcode{\sphinxupquote{PID\_discreto}}}{\emph{error}, \emph{ts}, \emph{s\_integral}, \emph{error\_anterior}, \emph{kp}, \emph{ki}, \emph{kd}}{}
Funcion para calcular el PID en forma discreta
\begin{quote}\begin{description}
\item[{Parámetros}] \leavevmode\begin{itemize}
\item {} 
\sphinxstyleliteralstrong{\sphinxupquote{error}} (\sphinxstyleliteralemphasis{\sphinxupquote{float}}) \textendash{} Señal de error

\item {} 
\sphinxstyleliteralstrong{\sphinxupquote{ts}} (\sphinxstyleliteralemphasis{\sphinxupquote{float}}) \textendash{} Periodo de muestreo

\item {} 
\sphinxstyleliteralstrong{\sphinxupquote{s\_integral}} (\sphinxstyleliteralemphasis{\sphinxupquote{float}}) \textendash{} Acumulador de la señal integral

\item {} 
\sphinxstyleliteralstrong{\sphinxupquote{error\_anterior}} (\sphinxstyleliteralemphasis{\sphinxupquote{deque Object}}) \textendash{} deque con el error anterior

\item {} 
\sphinxstyleliteralstrong{\sphinxupquote{kp}} (\sphinxstyleliteralemphasis{\sphinxupquote{float}}) \textendash{} Ganancia proporcional

\item {} 
\sphinxstyleliteralstrong{\sphinxupquote{ki}} (\sphinxstyleliteralemphasis{\sphinxupquote{float}}) \textendash{} Ganancia integral

\item {} 
\sphinxstyleliteralstrong{\sphinxupquote{kd}} (\sphinxstyleliteralemphasis{\sphinxupquote{float}}) \textendash{} Ganancia derivativa

\end{itemize}

\end{description}\end{quote}

\end{fulllineitems}

\index{derivadas\_discretas() (en el módulo discreto\_generator)@\spxentry{derivadas\_discretas()}\spxextra{en el módulo discreto\_generator}}

\begin{fulllineitems}
\phantomsection\label{\detokenize{codigos/discreto_generator:discreto_generator.derivadas_discretas}}\pysiglinewithargsret{\sphinxcode{\sphinxupquote{discreto\_generator.}}\sphinxbfcode{\sphinxupquote{derivadas\_discretas}}}{\emph{error}, \emph{ts}, \emph{error\_anterior}}{}
Funcion para calcular la derivada del error y la segunda derivada del error
\begin{quote}\begin{description}
\item[{Parámetros}] \leavevmode\begin{itemize}
\item {} 
\sphinxstyleliteralstrong{\sphinxupquote{error}} (\sphinxstyleliteralemphasis{\sphinxupquote{float}}) \textendash{} Señal de error

\item {} 
\sphinxstyleliteralstrong{\sphinxupquote{ts}} (\sphinxstyleliteralemphasis{\sphinxupquote{float}}) \textendash{} Periodo de muestreo

\item {} 
\sphinxstyleliteralstrong{\sphinxupquote{error\_anterior}} (\sphinxstyleliteralemphasis{\sphinxupquote{deque Object}}) \textendash{} deque con el error anterior

\end{itemize}

\end{description}\end{quote}

\end{fulllineitems}

\index{ss\_discreta() (en el módulo discreto\_generator)@\spxentry{ss\_discreta()}\spxextra{en el módulo discreto\_generator}}

\begin{fulllineitems}
\phantomsection\label{\detokenize{codigos/discreto_generator:discreto_generator.ss_discreta}}\pysiglinewithargsret{\sphinxcode{\sphinxupquote{discreto\_generator.}}\sphinxbfcode{\sphinxupquote{ss\_discreta}}}{\emph{A}, \emph{B}, \emph{C}, \emph{D}, \emph{x}, \emph{\_}, \emph{inputValue}}{}
Funcion para calcular la respuesta del sistema por medio de la representacion discreta de las ecuaciones de espacio de estados
\begin{quote}\begin{description}
\item[{Parámetros}] \leavevmode\begin{itemize}
\item {} 
\sphinxstyleliteralstrong{\sphinxupquote{ss}} (\sphinxstyleliteralemphasis{\sphinxupquote{LTI}}) \textendash{} Representacion del sistema

\item {} 
\sphinxstyleliteralstrong{\sphinxupquote{x}} (\sphinxstyleliteralemphasis{\sphinxupquote{numpyArray}}) \textendash{} Vector de estado

\item {} 
\sphinxstyleliteralstrong{\sphinxupquote{\_}} (\sphinxstyleliteralemphasis{\sphinxupquote{float}}) \textendash{} No importa

\item {} 
\sphinxstyleliteralstrong{\sphinxupquote{inputValue}} (\sphinxstyleliteralemphasis{\sphinxupquote{float}}) \textendash{} Valor de entrada al sistema

\end{itemize}

\end{description}\end{quote}

\end{fulllineitems}

\phantomsection\label{\detokenize{codigos/simulacionHandler:module-simulacionHandler}}\index{simulacionHandler (módulo)@\spxentry{simulacionHandler}\spxextra{módulo}}
Archivo para el manejo de la funcion de simulacion de sistemas de control, sirve de intermediario entre la interfaz grafica y la clase creada para manejar la simulacion en una hilo distinto, esto es debido al tiempo que puede llegar a tomar cada simulacion
\index{N\_validator() (en el módulo simulacionHandler)@\spxentry{N\_validator()}\spxextra{en el módulo simulacionHandler}}

\begin{fulllineitems}
\phantomsection\label{\detokenize{codigos/simulacionHandler:simulacionHandler.N_validator}}\pysiglinewithargsret{\sphinxcode{\sphinxupquote{simulacionHandler.}}\sphinxbfcode{\sphinxupquote{N\_validator}}}{\emph{self}}{}
Validacion del valor N

\end{fulllineitems}

\index{SimulacionHandler() (en el módulo simulacionHandler)@\spxentry{SimulacionHandler()}\spxextra{en el módulo simulacionHandler}}

\begin{fulllineitems}
\phantomsection\label{\detokenize{codigos/simulacionHandler:simulacionHandler.SimulacionHandler}}\pysiglinewithargsret{\sphinxcode{\sphinxupquote{simulacionHandler.}}\sphinxbfcode{\sphinxupquote{SimulacionHandler}}}{\emph{self}}{}
Funcion principal para el manejo de la funcionalida de simulacion de sistemas de control, se crean las señales a ejecutar cuando se interactua con los widgets incluyendo las validaciones de entradas

\end{fulllineitems}

\index{accion\_esquema\_selector() (en el módulo simulacionHandler)@\spxentry{accion\_esquema\_selector()}\spxextra{en el módulo simulacionHandler}}

\begin{fulllineitems}
\phantomsection\label{\detokenize{codigos/simulacionHandler:simulacionHandler.accion_esquema_selector}}\pysiglinewithargsret{\sphinxcode{\sphinxupquote{simulacionHandler.}}\sphinxbfcode{\sphinxupquote{accion\_esquema\_selector}}}{\emph{self}}{}
Funcion para mostrar los widgets indicados en funcion del esquema seleccionado

\end{fulllineitems}

\index{accionadordem\_validator() (en el módulo simulacionHandler)@\spxentry{accionadordem\_validator()}\spxextra{en el módulo simulacionHandler}}

\begin{fulllineitems}
\phantomsection\label{\detokenize{codigos/simulacionHandler:simulacionHandler.accionadordem_validator}}\pysiglinewithargsret{\sphinxcode{\sphinxupquote{simulacionHandler.}}\sphinxbfcode{\sphinxupquote{accionadordem\_validator}}}{\emph{self}}{}
Validacion del denominador de la funcion de transferencia correspondiente al accionador

\end{fulllineitems}

\index{accionadornum\_validator() (en el módulo simulacionHandler)@\spxentry{accionadornum\_validator()}\spxextra{en el módulo simulacionHandler}}

\begin{fulllineitems}
\phantomsection\label{\detokenize{codigos/simulacionHandler:simulacionHandler.accionadornum_validator}}\pysiglinewithargsret{\sphinxcode{\sphinxupquote{simulacionHandler.}}\sphinxbfcode{\sphinxupquote{accionadornum\_validator}}}{\emph{self}}{}
Validacion del numerador de la funcion de transferencia correspondiente al accionador

\end{fulllineitems}

\index{atol\_validator() (en el módulo simulacionHandler)@\spxentry{atol\_validator()}\spxextra{en el módulo simulacionHandler}}

\begin{fulllineitems}
\phantomsection\label{\detokenize{codigos/simulacionHandler:simulacionHandler.atol_validator}}\pysiglinewithargsret{\sphinxcode{\sphinxupquote{simulacionHandler.}}\sphinxbfcode{\sphinxupquote{atol\_validator}}}{\emph{self}}{}
Validacion de la tolerancia absoluta

\end{fulllineitems}

\index{calcular\_simulacion() (en el módulo simulacionHandler)@\spxentry{calcular\_simulacion()}\spxextra{en el módulo simulacionHandler}}

\begin{fulllineitems}
\phantomsection\label{\detokenize{codigos/simulacionHandler:simulacionHandler.calcular_simulacion}}\pysiglinewithargsret{\sphinxcode{\sphinxupquote{simulacionHandler.}}\sphinxbfcode{\sphinxupquote{calcular\_simulacion}}}{\emph{self}}{}
Funcion para inicializar el QThread y realizar los calculos de la simulacion

\end{fulllineitems}

\index{configuration\_data() (en el módulo simulacionHandler)@\spxentry{configuration\_data()}\spxextra{en el módulo simulacionHandler}}

\begin{fulllineitems}
\phantomsection\label{\detokenize{codigos/simulacionHandler:simulacionHandler.configuration_data}}\pysiglinewithargsret{\sphinxcode{\sphinxupquote{simulacionHandler.}}\sphinxbfcode{\sphinxupquote{configuration\_data}}}{\emph{self}}{}
Funcion para cambiar la configuracion del solver a utilizar

\end{fulllineitems}

\index{error\_gui() (en el módulo simulacionHandler)@\spxentry{error\_gui()}\spxextra{en el módulo simulacionHandler}}

\begin{fulllineitems}
\phantomsection\label{\detokenize{codigos/simulacionHandler:simulacionHandler.error_gui}}\pysiglinewithargsret{\sphinxcode{\sphinxupquote{simulacionHandler.}}\sphinxbfcode{\sphinxupquote{error\_gui}}}{\emph{self}, \emph{error}}{}
Funcion para mostrar los errores que pudiesen ocurrir durante la simulacion, esta funcion es utilizada por el QThread
\begin{quote}\begin{description}
\item[{Parámetros}] \leavevmode
\sphinxstyleliteralstrong{\sphinxupquote{error}} (\sphinxstyleliteralemphasis{\sphinxupquote{int}}) \textendash{} Indicador del error

\end{description}\end{quote}

\end{fulllineitems}

\index{escalonAvanzado\_validator() (en el módulo simulacionHandler)@\spxentry{escalonAvanzado\_validator()}\spxextra{en el módulo simulacionHandler}}

\begin{fulllineitems}
\phantomsection\label{\detokenize{codigos/simulacionHandler:simulacionHandler.escalonAvanzado_validator}}\pysiglinewithargsret{\sphinxcode{\sphinxupquote{simulacionHandler.}}\sphinxbfcode{\sphinxupquote{escalonAvanzado\_validator}}}{\emph{self}}{}
Validacion del escalon avanzado

\end{fulllineitems}

\index{escalon\_validator() (en el módulo simulacionHandler)@\spxentry{escalon\_validator()}\spxextra{en el módulo simulacionHandler}}

\begin{fulllineitems}
\phantomsection\label{\detokenize{codigos/simulacionHandler:simulacionHandler.escalon_validator}}\pysiglinewithargsret{\sphinxcode{\sphinxupquote{simulacionHandler.}}\sphinxbfcode{\sphinxupquote{escalon\_validator}}}{\emph{self}}{}
Validacion del escalon simple

\end{fulllineitems}

\index{get\_pathcontroller1() (en el módulo simulacionHandler)@\spxentry{get\_pathcontroller1()}\spxextra{en el módulo simulacionHandler}}

\begin{fulllineitems}
\phantomsection\label{\detokenize{codigos/simulacionHandler:simulacionHandler.get_pathcontroller1}}\pysiglinewithargsret{\sphinxcode{\sphinxupquote{simulacionHandler.}}\sphinxbfcode{\sphinxupquote{get\_pathcontroller1}}}{\emph{self}}{}
Funcion para obtener la direccion al archivo del controlador difuso

\end{fulllineitems}

\index{get\_pathcontroller2() (en el módulo simulacionHandler)@\spxentry{get\_pathcontroller2()}\spxextra{en el módulo simulacionHandler}}

\begin{fulllineitems}
\phantomsection\label{\detokenize{codigos/simulacionHandler:simulacionHandler.get_pathcontroller2}}\pysiglinewithargsret{\sphinxcode{\sphinxupquote{simulacionHandler.}}\sphinxbfcode{\sphinxupquote{get\_pathcontroller2}}}{\emph{self}}{}
Funcion para obtener la direccion al archivo del controlador difuso 2 (PD)

\end{fulllineitems}

\index{inferiorSaturador\_validator() (en el módulo simulacionHandler)@\spxentry{inferiorSaturador\_validator()}\spxextra{en el módulo simulacionHandler}}

\begin{fulllineitems}
\phantomsection\label{\detokenize{codigos/simulacionHandler:simulacionHandler.inferiorSaturador_validator}}\pysiglinewithargsret{\sphinxcode{\sphinxupquote{simulacionHandler.}}\sphinxbfcode{\sphinxupquote{inferiorSaturador\_validator}}}{\emph{self}}{}
Validacion del limite inferior del saturador

\end{fulllineitems}

\index{kd\_validator() (en el módulo simulacionHandler)@\spxentry{kd\_validator()}\spxextra{en el módulo simulacionHandler}}

\begin{fulllineitems}
\phantomsection\label{\detokenize{codigos/simulacionHandler:simulacionHandler.kd_validator}}\pysiglinewithargsret{\sphinxcode{\sphinxupquote{simulacionHandler.}}\sphinxbfcode{\sphinxupquote{kd\_validator}}}{\emph{self}}{}
Validacion de la ganancia derivativa

\end{fulllineitems}

\index{ki\_validator() (en el módulo simulacionHandler)@\spxentry{ki\_validator()}\spxextra{en el módulo simulacionHandler}}

\begin{fulllineitems}
\phantomsection\label{\detokenize{codigos/simulacionHandler:simulacionHandler.ki_validator}}\pysiglinewithargsret{\sphinxcode{\sphinxupquote{simulacionHandler.}}\sphinxbfcode{\sphinxupquote{ki\_validator}}}{\emph{self}}{}
Validacion de la ganancia integral

\end{fulllineitems}

\index{kp\_validator() (en el módulo simulacionHandler)@\spxentry{kp\_validator()}\spxextra{en el módulo simulacionHandler}}

\begin{fulllineitems}
\phantomsection\label{\detokenize{codigos/simulacionHandler:simulacionHandler.kp_validator}}\pysiglinewithargsret{\sphinxcode{\sphinxupquote{simulacionHandler.}}\sphinxbfcode{\sphinxupquote{kp\_validator}}}{\emph{self}}{}
Validacion de la ganancia proporcional

\end{fulllineitems}

\index{maxstep\_validator() (en el módulo simulacionHandler)@\spxentry{maxstep\_validator()}\spxextra{en el módulo simulacionHandler}}

\begin{fulllineitems}
\phantomsection\label{\detokenize{codigos/simulacionHandler:simulacionHandler.maxstep_validator}}\pysiglinewithargsret{\sphinxcode{\sphinxupquote{simulacionHandler.}}\sphinxbfcode{\sphinxupquote{maxstep\_validator}}}{\emph{self}}{}
Validacion del incremento maximo de paso

\end{fulllineitems}

\index{minstep\_validator() (en el módulo simulacionHandler)@\spxentry{minstep\_validator()}\spxextra{en el módulo simulacionHandler}}

\begin{fulllineitems}
\phantomsection\label{\detokenize{codigos/simulacionHandler:simulacionHandler.minstep_validator}}\pysiglinewithargsret{\sphinxcode{\sphinxupquote{simulacionHandler.}}\sphinxbfcode{\sphinxupquote{minstep\_validator}}}{\emph{self}}{}
Validacion del decremento minimo de paso

\end{fulllineitems}

\index{pade\_validator() (en el módulo simulacionHandler)@\spxentry{pade\_validator()}\spxextra{en el módulo simulacionHandler}}

\begin{fulllineitems}
\phantomsection\label{\detokenize{codigos/simulacionHandler:simulacionHandler.pade_validator}}\pysiglinewithargsret{\sphinxcode{\sphinxupquote{simulacionHandler.}}\sphinxbfcode{\sphinxupquote{pade\_validator}}}{\emph{self}}{}
Validacion del orden del pade

\end{fulllineitems}

\index{plot\_final\_results() (en el módulo simulacionHandler)@\spxentry{plot\_final\_results()}\spxextra{en el módulo simulacionHandler}}

\begin{fulllineitems}
\phantomsection\label{\detokenize{codigos/simulacionHandler:simulacionHandler.plot_final_results}}\pysiglinewithargsret{\sphinxcode{\sphinxupquote{simulacionHandler.}}\sphinxbfcode{\sphinxupquote{plot\_final\_results}}}{\emph{self}, \emph{result}}{}
Funcion para graficar los resultados finales de la simulacion
\begin{quote}\begin{description}
\item[{Parámetros}] \leavevmode
\sphinxstyleliteralstrong{\sphinxupquote{result}} (\sphinxstyleliteralemphasis{\sphinxupquote{list}}) \textendash{} Lista con los resultados obtenidos

\end{description}\end{quote}

\end{fulllineitems}

\index{restablecer\_configuracion() (en el módulo simulacionHandler)@\spxentry{restablecer\_configuracion()}\spxextra{en el módulo simulacionHandler}}

\begin{fulllineitems}
\phantomsection\label{\detokenize{codigos/simulacionHandler:simulacionHandler.restablecer_configuracion}}\pysiglinewithargsret{\sphinxcode{\sphinxupquote{simulacionHandler.}}\sphinxbfcode{\sphinxupquote{restablecer\_configuracion}}}{\emph{self}}{}
Funcion para restablecer la configuracion avanzada por defecto

\end{fulllineitems}

\index{rtol\_validator() (en el módulo simulacionHandler)@\spxentry{rtol\_validator()}\spxextra{en el módulo simulacionHandler}}

\begin{fulllineitems}
\phantomsection\label{\detokenize{codigos/simulacionHandler:simulacionHandler.rtol_validator}}\pysiglinewithargsret{\sphinxcode{\sphinxupquote{simulacionHandler.}}\sphinxbfcode{\sphinxupquote{rtol\_validator}}}{\emph{self}}{}
Validacion de la tolerancia relativa

\end{fulllineitems}

\index{safetyFactor\_validator() (en el módulo simulacionHandler)@\spxentry{safetyFactor\_validator()}\spxextra{en el módulo simulacionHandler}}

\begin{fulllineitems}
\phantomsection\label{\detokenize{codigos/simulacionHandler:simulacionHandler.safetyFactor_validator}}\pysiglinewithargsret{\sphinxcode{\sphinxupquote{simulacionHandler.}}\sphinxbfcode{\sphinxupquote{safetyFactor\_validator}}}{\emph{self}}{}
Validacion del factor de seguridad

\end{fulllineitems}

\index{sensordem\_validator() (en el módulo simulacionHandler)@\spxentry{sensordem\_validator()}\spxextra{en el módulo simulacionHandler}}

\begin{fulllineitems}
\phantomsection\label{\detokenize{codigos/simulacionHandler:simulacionHandler.sensordem_validator}}\pysiglinewithargsret{\sphinxcode{\sphinxupquote{simulacionHandler.}}\sphinxbfcode{\sphinxupquote{sensordem\_validator}}}{\emph{self}}{}
Validacion del denominador de la funcion de transferencia correspondiente al sensor

\end{fulllineitems}

\index{sensornum\_validator() (en el módulo simulacionHandler)@\spxentry{sensornum\_validator()}\spxextra{en el módulo simulacionHandler}}

\begin{fulllineitems}
\phantomsection\label{\detokenize{codigos/simulacionHandler:simulacionHandler.sensornum_validator}}\pysiglinewithargsret{\sphinxcode{\sphinxupquote{simulacionHandler.}}\sphinxbfcode{\sphinxupquote{sensornum\_validator}}}{\emph{self}}{}
Validacion del numerador de la funcion de transferencia correspondiente al sensor

\end{fulllineitems}

\index{simulacion\_stacked\_to\_ss() (en el módulo simulacionHandler)@\spxentry{simulacion\_stacked\_to\_ss()}\spxextra{en el módulo simulacionHandler}}

\begin{fulllineitems}
\phantomsection\label{\detokenize{codigos/simulacionHandler:simulacionHandler.simulacion_stacked_to_ss}}\pysiglinewithargsret{\sphinxcode{\sphinxupquote{simulacionHandler.}}\sphinxbfcode{\sphinxupquote{simulacion\_stacked\_to\_ss}}}{\emph{self}}{}
Funcion para cambiar de funcion de transferencia a ecuacion de espacio de estados

\end{fulllineitems}

\index{simulacion\_stacked\_to\_tf() (en el módulo simulacionHandler)@\spxentry{simulacion\_stacked\_to\_tf()}\spxextra{en el módulo simulacionHandler}}

\begin{fulllineitems}
\phantomsection\label{\detokenize{codigos/simulacionHandler:simulacionHandler.simulacion_stacked_to_tf}}\pysiglinewithargsret{\sphinxcode{\sphinxupquote{simulacionHandler.}}\sphinxbfcode{\sphinxupquote{simulacion\_stacked\_to\_tf}}}{\emph{self}}{}
Funcion para cambiar de ecuacion de espacio de estados a funcion de transferencia

\end{fulllineitems}

\index{ssA\_validator() (en el módulo simulacionHandler)@\spxentry{ssA\_validator()}\spxextra{en el módulo simulacionHandler}}

\begin{fulllineitems}
\phantomsection\label{\detokenize{codigos/simulacionHandler:simulacionHandler.ssA_validator}}\pysiglinewithargsret{\sphinxcode{\sphinxupquote{simulacionHandler.}}\sphinxbfcode{\sphinxupquote{ssA\_validator}}}{\emph{self}}{}
Validacion de la matriz de estados de la ecuacion de espacio de estados

\end{fulllineitems}

\index{ssB\_validator() (en el módulo simulacionHandler)@\spxentry{ssB\_validator()}\spxextra{en el módulo simulacionHandler}}

\begin{fulllineitems}
\phantomsection\label{\detokenize{codigos/simulacionHandler:simulacionHandler.ssB_validator}}\pysiglinewithargsret{\sphinxcode{\sphinxupquote{simulacionHandler.}}\sphinxbfcode{\sphinxupquote{ssB\_validator}}}{\emph{self}}{}
Validacion de la matriz de entrada de la ecuacion de espacio de estados

\end{fulllineitems}

\index{ssC\_validator() (en el módulo simulacionHandler)@\spxentry{ssC\_validator()}\spxextra{en el módulo simulacionHandler}}

\begin{fulllineitems}
\phantomsection\label{\detokenize{codigos/simulacionHandler:simulacionHandler.ssC_validator}}\pysiglinewithargsret{\sphinxcode{\sphinxupquote{simulacionHandler.}}\sphinxbfcode{\sphinxupquote{ssC\_validator}}}{\emph{self}}{}
Validacion de la matriz de salida de la ecuacion de espacio de estados

\end{fulllineitems}

\index{ssD\_validator() (en el módulo simulacionHandler)@\spxentry{ssD\_validator()}\spxextra{en el módulo simulacionHandler}}

\begin{fulllineitems}
\phantomsection\label{\detokenize{codigos/simulacionHandler:simulacionHandler.ssD_validator}}\pysiglinewithargsret{\sphinxcode{\sphinxupquote{simulacionHandler.}}\sphinxbfcode{\sphinxupquote{ssD\_validator}}}{\emph{self}}{}
Validacion de la matriz de transmision directa de la ecuacion de espacio de estados

\end{fulllineitems}

\index{ssdelay\_validator() (en el módulo simulacionHandler)@\spxentry{ssdelay\_validator()}\spxextra{en el módulo simulacionHandler}}

\begin{fulllineitems}
\phantomsection\label{\detokenize{codigos/simulacionHandler:simulacionHandler.ssdelay_validator}}\pysiglinewithargsret{\sphinxcode{\sphinxupquote{simulacionHandler.}}\sphinxbfcode{\sphinxupquote{ssdelay\_validator}}}{\emph{self}}{}
Validacion del delay de la ecuacion de espacio de estados

\end{fulllineitems}

\index{ssperiodo\_validator() (en el módulo simulacionHandler)@\spxentry{ssperiodo\_validator()}\spxextra{en el módulo simulacionHandler}}

\begin{fulllineitems}
\phantomsection\label{\detokenize{codigos/simulacionHandler:simulacionHandler.ssperiodo_validator}}\pysiglinewithargsret{\sphinxcode{\sphinxupquote{simulacionHandler.}}\sphinxbfcode{\sphinxupquote{ssperiodo\_validator}}}{\emph{self}}{}
Validacion del periodo de muestreo de la ecuacion de espacio de estados

\end{fulllineitems}

\index{superiorSaturador\_validator() (en el módulo simulacionHandler)@\spxentry{superiorSaturador\_validator()}\spxextra{en el módulo simulacionHandler}}

\begin{fulllineitems}
\phantomsection\label{\detokenize{codigos/simulacionHandler:simulacionHandler.superiorSaturador_validator}}\pysiglinewithargsret{\sphinxcode{\sphinxupquote{simulacionHandler.}}\sphinxbfcode{\sphinxupquote{superiorSaturador\_validator}}}{\emph{self}}{}
Validacion del limite superior del saturador

\end{fulllineitems}

\index{tfdelay\_validator() (en el módulo simulacionHandler)@\spxentry{tfdelay\_validator()}\spxextra{en el módulo simulacionHandler}}

\begin{fulllineitems}
\phantomsection\label{\detokenize{codigos/simulacionHandler:simulacionHandler.tfdelay_validator}}\pysiglinewithargsret{\sphinxcode{\sphinxupquote{simulacionHandler.}}\sphinxbfcode{\sphinxupquote{tfdelay\_validator}}}{\emph{self}}{}
Validacion del delay de la funcion de transferencia

\end{fulllineitems}

\index{tfdem\_validator() (en el módulo simulacionHandler)@\spxentry{tfdem\_validator()}\spxextra{en el módulo simulacionHandler}}

\begin{fulllineitems}
\phantomsection\label{\detokenize{codigos/simulacionHandler:simulacionHandler.tfdem_validator}}\pysiglinewithargsret{\sphinxcode{\sphinxupquote{simulacionHandler.}}\sphinxbfcode{\sphinxupquote{tfdem\_validator}}}{\emph{self}}{}
Validacion del denominador de la funcion de transferencia

\end{fulllineitems}

\index{tfnum\_validator() (en el módulo simulacionHandler)@\spxentry{tfnum\_validator()}\spxextra{en el módulo simulacionHandler}}

\begin{fulllineitems}
\phantomsection\label{\detokenize{codigos/simulacionHandler:simulacionHandler.tfnum_validator}}\pysiglinewithargsret{\sphinxcode{\sphinxupquote{simulacionHandler.}}\sphinxbfcode{\sphinxupquote{tfnum\_validator}}}{\emph{self}}{}
Validacion del numerador de la funcion de transferencia

\end{fulllineitems}

\index{tfperiodo\_validator() (en el módulo simulacionHandler)@\spxentry{tfperiodo\_validator()}\spxextra{en el módulo simulacionHandler}}

\begin{fulllineitems}
\phantomsection\label{\detokenize{codigos/simulacionHandler:simulacionHandler.tfperiodo_validator}}\pysiglinewithargsret{\sphinxcode{\sphinxupquote{simulacionHandler.}}\sphinxbfcode{\sphinxupquote{tfperiodo\_validator}}}{\emph{self}}{}
Validacion del periodo de muestreo de la funcion de transferencia

\end{fulllineitems}

\index{tiempo\_validator() (en el módulo simulacionHandler)@\spxentry{tiempo\_validator()}\spxextra{en el módulo simulacionHandler}}

\begin{fulllineitems}
\phantomsection\label{\detokenize{codigos/simulacionHandler:simulacionHandler.tiempo_validator}}\pysiglinewithargsret{\sphinxcode{\sphinxupquote{simulacionHandler.}}\sphinxbfcode{\sphinxupquote{tiempo\_validator}}}{\emph{self}}{}
Validacion del tiempo de simulacion

\end{fulllineitems}

\index{update\_progresBar\_function() (en el módulo simulacionHandler)@\spxentry{update\_progresBar\_function()}\spxextra{en el módulo simulacionHandler}}

\begin{fulllineitems}
\phantomsection\label{\detokenize{codigos/simulacionHandler:simulacionHandler.update_progresBar_function}}\pysiglinewithargsret{\sphinxcode{\sphinxupquote{simulacionHandler.}}\sphinxbfcode{\sphinxupquote{update\_progresBar\_function}}}{\emph{self}, \emph{value}}{}
Funcion para actualizar la barra de progreso de la simulacion, esta funcion es utilizada por el QThread
\begin{quote}\begin{description}
\item[{Parámetros}] \leavevmode
\sphinxstyleliteralstrong{\sphinxupquote{value}} (\sphinxstyleliteralemphasis{\sphinxupquote{float}}) \textendash{} Valor en porcentaje del progreso

\end{description}\end{quote}

\end{fulllineitems}

\phantomsection\label{\detokenize{codigos/Principal:module-main}}\index{main (módulo)@\spxentry{main}\spxextra{módulo}}
Archivo principal, en orden de ejecutar la aplicacion, este es el archivo a ejecutar
\index{MainWindow (clase en main)@\spxentry{MainWindow}\spxextra{clase en main}}

\begin{fulllineitems}
\phantomsection\label{\detokenize{codigos/Principal:main.MainWindow}}\pysiglinewithargsret{\sphinxbfcode{\sphinxupquote{class }}\sphinxcode{\sphinxupquote{main.}}\sphinxbfcode{\sphinxupquote{MainWindow}}}{\emph{parent=None}}{}
Clase principal del programa, esta clase hereda de QMainWindow y Ui\_MainWindow, la primera es la clase base de ventanas que ofrece Qt mientras que la segunda es la clase que se crea a partir de qtdesigner y quien posee toda la definicion de toda la interfaz grafica. Desde aca se ejecutan los archivos Handler, quienes representan los enlaces entre las rutinas y la interfaz grafica de cada una de las funciones del laboratorio virtual, estos Handlers se tratan como si fueran una extension de esta clase, por tanto, se les envia self y se recibe self y se sigue tratando como si fuera parte de la clase.
\begin{quote}\begin{description}
\item[{Parámetros}] \leavevmode\begin{itemize}
\item {} 
\sphinxstyleliteralstrong{\sphinxupquote{QtWidgets}} (\sphinxstyleliteralemphasis{\sphinxupquote{ObjectType}}) \textendash{} Clase base de ventana ofrecida por Qt

\item {} 
\sphinxstyleliteralstrong{\sphinxupquote{Ui\_MainWindow}} (\sphinxstyleliteralemphasis{\sphinxupquote{ObjectType}}) \textendash{} Clase con la interfaz grafica autogenerada con qtdesigner

\end{itemize}

\end{description}\end{quote}
\index{\_\_init\_\_() (método de main.MainWindow)@\spxentry{\_\_init\_\_()}\spxextra{método de main.MainWindow}}

\begin{fulllineitems}
\phantomsection\label{\detokenize{codigos/Principal:main.MainWindow.__init__}}\pysiglinewithargsret{\sphinxbfcode{\sphinxupquote{\_\_init\_\_}}}{\emph{parent=None}}{}
Constructor de la clase, aca se inicializan los objetos de las clases heredadas y se hacen los llamados a los Handlers
\begin{quote}\begin{description}
\item[{Parámetros}] \leavevmode
\sphinxstyleliteralstrong{\sphinxupquote{parent}} (\sphinxstyleliteralemphasis{\sphinxupquote{NoneType}}\sphinxstyleliteralemphasis{\sphinxupquote{, }}\sphinxstyleliteralemphasis{\sphinxupquote{optional}}) \textendash{} Sin efecto, defaults to None

\end{description}\end{quote}

\end{fulllineitems}

\index{closeEvent() (método de main.MainWindow)@\spxentry{closeEvent()}\spxextra{método de main.MainWindow}}

\begin{fulllineitems}
\phantomsection\label{\detokenize{codigos/Principal:main.MainWindow.closeEvent}}\pysiglinewithargsret{\sphinxbfcode{\sphinxupquote{closeEvent}}}{\emph{event}}{}
Evento pera el cerrado de la ventana

\end{fulllineitems}

\index{resource\_path() (método de main.MainWindow)@\spxentry{resource\_path()}\spxextra{método de main.MainWindow}}

\begin{fulllineitems}
\phantomsection\label{\detokenize{codigos/Principal:main.MainWindow.resource_path}}\pysiglinewithargsret{\sphinxbfcode{\sphinxupquote{resource\_path}}}{\emph{relative\_path}}{}
Funcion para generar direcciones absolutas a partir de direcciones relativas
\begin{quote}\begin{description}
\item[{Parámetros}] \leavevmode
\sphinxstyleliteralstrong{\sphinxupquote{relative\_path}} (\sphinxstyleliteralemphasis{\sphinxupquote{str}}) \textendash{} direccion relativa

\end{description}\end{quote}

\end{fulllineitems}


\end{fulllineitems}



\chapter{Indices y tablas}
\label{\detokenize{index:indices-y-tablas}}\begin{itemize}
\item {} 
\DUrole{xref,std,std-ref}{genindex}

\item {} 
\DUrole{xref,std,std-ref}{modindex}

\item {} 
\DUrole{xref,std,std-ref}{search}

\end{itemize}


\renewcommand{\indexname}{Índice de Módulos Python}
\begin{sphinxtheindex}
\let\bigletter\sphinxstyleindexlettergroup
\bigletter{a}
\item\relax\sphinxstyleindexentry{analisisHandler}\sphinxstyleindexpageref{codigos/analisisHandler:\detokenize{module-analisisHandler}}
\indexspace
\bigletter{d}
\item\relax\sphinxstyleindexentry{discreto\_generator}\sphinxstyleindexpageref{codigos/discreto_generator:\detokenize{module-discreto_generator}}
\indexspace
\bigletter{f}
\item\relax\sphinxstyleindexentry{focusLineEdit}\sphinxstyleindexpageref{codigos/Promociones:\detokenize{module-focusLineEdit}}
\item\relax\sphinxstyleindexentry{FuzzyHandler}\sphinxstyleindexpageref{codigos/FuzzyHandler:\detokenize{module-FuzzyHandler}}
\indexspace
\bigletter{m}
\item\relax\sphinxstyleindexentry{main}\sphinxstyleindexpageref{codigos/Principal:\detokenize{module-main}}
\item\relax\sphinxstyleindexentry{mlpwidget}\sphinxstyleindexpageref{codigos/Promociones:\detokenize{module-mlpwidget}}
\item\relax\sphinxstyleindexentry{modificadorMf}\sphinxstyleindexpageref{codigos/modificadorMF:\detokenize{module-modificadorMf}}
\indexspace
\bigletter{p}
\item\relax\sphinxstyleindexentry{pyqtgraphWidget}\sphinxstyleindexpageref{codigos/Promociones:\detokenize{module-pyqtgraphWidget}}
\indexspace
\bigletter{r}
\item\relax\sphinxstyleindexentry{rk\_generator}\sphinxstyleindexpageref{codigos/rk_generator:\detokenize{module-rk_generator}}
\item\relax\sphinxstyleindexentry{rutinas\_analisis}\sphinxstyleindexpageref{codigos/rutinas_analisis:\detokenize{module-rutinas_analisis}}
\item\relax\sphinxstyleindexentry{rutinas\_CSV}\sphinxstyleindexpageref{codigos/rutinas_CSV:\detokenize{module-rutinas_CSV}}
\item\relax\sphinxstyleindexentry{rutinas\_fuzzy}\sphinxstyleindexpageref{codigos/rutinas_fuzzy:\detokenize{module-rutinas_fuzzy}}
\item\relax\sphinxstyleindexentry{rutinas\_PID}\sphinxstyleindexpageref{codigos/rutinas_PID:\detokenize{module-rutinas_PID}}
\item\relax\sphinxstyleindexentry{rutinas\_rk}\sphinxstyleindexpageref{codigos/rutinas_rk:\detokenize{module-rutinas_rk}}
\item\relax\sphinxstyleindexentry{rutinas\_simulacion}\sphinxstyleindexpageref{codigos/rutinas_simulacion:\detokenize{module-rutinas_simulacion}}
\indexspace
\bigletter{s}
\item\relax\sphinxstyleindexentry{simulacionHandler}\sphinxstyleindexpageref{codigos/simulacionHandler:\detokenize{module-simulacionHandler}}
\indexspace
\bigletter{t}
\item\relax\sphinxstyleindexentry{TuningHandler}\sphinxstyleindexpageref{codigos/TuningHandler:\detokenize{module-TuningHandler}}
\end{sphinxtheindex}

\renewcommand{\indexname}{Índice}
\printindex
\end{document}