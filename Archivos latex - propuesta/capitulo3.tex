En este capítulo se abarcará la metodología a emplear en el desarrollo de esta investigación, se definirá el tipo, diseño y modalidad de la investigación asi como las fases de la misma.

\section{Tipo de investigación}
	
	Tomando en cuenta los objetivos de la investigación y las bases teóricas que la componen se considera que esta investigación es de tipo proyectiva, esto es debido a que se pretende realizar una propuesta concreta para solventar una problemática, \textcite{jacquelin2010guia} afirma que:	
	
	\blockquote[p.$\,$133]{La investigación proyectiva tiene como objetivo diseñar o crear propuestas dirigidas a resolver determinadas situaciones. Los proyectos de arquitectura e ingeniería, el diseño de maquinarias, la creación de programas de intervención social, el diseño de programas de estudio, los inventos, la elaboración de programas informáticos, entre otros, siempre que estén sustentados en un proceso de investigación, son ejemplos de investigación proyectiva.}

\section{Diseño de la investigación}

	El diseño de la investigación es no experimental y de tipo transeccional descriptivo, esto es debido a que se describirán los métodos de análisis y diseño de sistemas de control clásicos y difusos, \textcite{sampieri1998metodologia} definen  la investigación no experimental como: \blockquote[p.150]{la investigación que se realiza sin manipular deliberadamente variables. Es decir,
	se trata de estudios donde no hacemos variar en forma intencional las variables independientes para
	ver su efecto sobre otras variables. Lo que hacemos en la investigación no experimental
	es observar fenómenos tal como se dan en su contexto natural, para posteriormente
	analizarlos.}

\section{Modalidad}

	Esta investigación se encuentra enmarcada en la modalidad de un proyecto factible, debido a que tiene objetivos para atender una necesidad por medio de unas acciones claramente definidas, \textcite{renie2002factible} afirma que:
	
	\blockquote[pp.$\,$6-7]{un proyecto factible consiste en un conjunto de actividades vinculadas entre sí, cuya ejecución permitirá el logro de objetivos previamente definidos en atención a las necesidades que pueda tener una institución o un grupo social en un momento determinado. Es decir, la finalidad del proyecto factible radica en el diseño de una propuesta de acción dirigida a resolver un problema o necesidad previamente detectada en el medio.}

\section{Fases de la investigación}
	
	\paragraph{Fase 1: Estudio de los sistemas de control clásicos y difusos}
		
		En esta fase se procederá a realizar los estudios necesarios en el área de los sistemas de control, esto con la idea de abarcarlos en profundidad y tener un entendimiento claro de su funcionamiento y de la matemática implicada, además, se realizará de forma similar un estudio de controladores difusos con estructura Mamdani y de los esquemas de control difuso.
		
	\paragraph{Fase 2: Codificación de rutinas}
		
		Para esta fase con los conocimientos adquiridos de la fase 1, se determinarán que rutinas pueden ser ejecutadas solo con las bibliotecas de python y cuales se deberán codificar de cero, además, se codificaran todas las rutinas necesarias para el funcionamiento del laboratorio virtual, para esto, se hará uso de las bibliotecas externas de cálculo numérico, control, diseño de controladores difusos y salidas gráficas junto con las que se consideren necesarias.
		
	\paragraph{Fase 3: Interfaz gráfica y enlace con rutinas}
		
		En esta fase se realizará la interfaz gráfica para el usuario final, esta interfaz gráfica deberá conectarse y adaptarse a las rutinas previamente codificadas en la fase 2 para su funcionamiento adecuado, en orden de tener diseño acorde se tomarán en cuenta los antecedentes presentados en esta propuesta.
		
	\paragraph{Fase 4: Comparación de resultados}
		
		En esta última fase y con el laboratorio de sistemas de control ya en funcionamiento se procederá a analizar los resultados obtenidos y a compararlos con otras herramientas, la evaluación se realizará en función del resultado esperado, facilidad de implementación, velocidad de ejecución, las ventajas y desventajas de cada una de las herramientas.

\section{Factibilidad de la propuesta}

	Esta investigación es claramente factible al estar enfocada en la utilización de software libre para la creación del laboratorio virtual, asi lo demuestran los antecedentes expuestos en esta propuesta, los cuales han hecho uso de software para le creación de aplicaciones con usos en el área de los sistemas de control, otros han utilizado específicamente Python para el desarrollo de controladores difusos, ademas, al tratarse de software libre no existe la necesidad de financiación externa, facilitando asi su desarrollo. Para realizar esta investigación solo hará falta el uso de una computadora con acceso a internet con el fin de obtener el software necesario y realizar las investigaciones pertinentes en cada tema, ademas, se espera hacer uso de la biblioteca de la universidad, nuevamente, para realizar las investigaciones pertinentes.
	
\section{Cronograma de ejecución}

	Las actividades a realizar fueron determinadas en función de las fases, los requerimientos y objetivos de esta investigación, en la \cref{tab:Cronograma} se puede observar el cronograma de ejecución, en donde se denotan las fases y actividades de la misma junto con el numero de semanas que se estima emplear en cada una de las actividades. La estimación de semanas para cada una de las actividades se tomo considerando la complejidad del tema, a su vez, la complejidad fue evaluada tomando en cuenta los antecedentes expuestos en esta propuesta y teniendo en consideración las experiencias similares previas en cada uno de los temas a abarcar.

% \vspace{0.5cm}

% \begin{table}[h]
% 	\caption{Cronograma de ejecución}
% 	\label{tab:Cronograma}
% 	\resizebox{\textwidth}{!}{
% 	\begin{tabular}{llc}
% 	\toprule
% 	\multicolumn{1}{c}{Fases} & \multicolumn{1}{c}{Actividades} & \multicolumn{1}{l}{Semanas} \\ \midrule
% 	\multirow{3}{*}{Fase 1: Estudio de los sistemas de control clásicos y difusos} & Estudio de los métodos de análisis de sistemas de control & 1 \\
% 		& Estudio de controladores PID & 1 \\
% 		& Estudio de controladores difusos tipo Mamdani & 1 \\
% 		&  & \multicolumn{1}{l}{} \\
% 	\multirow{4}{*}{Fase 2: Codificación de rutinas} & Codificar las rutinas de análisis de sistemas de control & 2 \\
% 		& Codificar las rutinas para la entonación de controladores PID & 2 \\
% 		& Codificar rutinas para la creación de controladores difusos & 3 \\
% 		& Codificar rutinas para la simulación de sistemas de control & 2 \\
% 		&  & \multicolumn{1}{l}{} \\
% 	\multirow{2}{*}{Fase 3: Interfaz gráfica y enlace con rutinas} & Crear la interfaz gráfica para el laboratorio virtual & 1 \\
% 		& Acoplar las rutinas con la interfaz gráfica & 2 \\
% 		&  & \multicolumn{1}{l}{} \\
% 	\multirow{2}{*}{Fase 4: Comparación de resultados} & Análisis de resultados obtenidos & 1 \\
% 		& Comparación con otras herramientas & 1 \\ \midrule
% 		& \multicolumn{1}{r}{Total semanas:} & 17 \\ \bottomrule
% 	\end{tabular}%
% 	}
% \end{table}

	Adicionalmente se realizo un diagrama de Gantt para tener una mejor representación de los tiempos a utilizar en cada actividad durante el desarrollo de la investigación. El diagrama se puede observar en la \cref{fig:gantt}, a continuación se listan las actividades a realizar:

\vspace{10pt}

\begin{enumerate}[label=\bfseries Actividad \arabic*:, wide=0pt, leftmargin=*]
	\item Estudio de los métodos de análisis de sistemas de control. 
	\item Estudio de controladores PID.
	\item Estudio de controladores difusos tipo Mamdani.
	\item Codificar las rutinas de análisis de sistemas de control.
	\item Codificar las rutinas para la entonación de controladores PID.
	\item Codificar rutinas para la creación de controladores difusos.
	\item Codificar rutinas para la simulación de sistemas de control.
	\item Crear la interfaz gráfica para el laboratorio virtual.
	\item Acoplar las rutinas con la interfaz gráfica.
	\item Análisis de resultados obtenidos.
	\item Comparación con otras herramientas.
\end{enumerate}

\afterpage{
\begin{landscape}
\begin{table}
    \caption{Cronograma de ejecución}
    \label{tab:Cronograma}
    \small
    \begin{tabular}{@{\extracolsep{\fill}}llc}
    \toprule
    \multicolumn{1}{c}{Fases} & \multicolumn{1}{c}{Actividades} & \multicolumn{1}{l}{Semanas} \\ \midrule
    \multirow{3}{*}{Fase 1: Estudio de los sistemas de control clásicos y difusos} & Estudio de los métodos de análisis de sistemas de control & 1 \\
     & Estudio de controladores PID & 1 \\
     & Estudio de controladores difusos tipo Mamdani & 1 \\
     &  & \multicolumn{1}{l}{} \\
    \multirow{4}{*}{Fase 2: Codificación de rutinas} & Codificar las rutinas de análisis de sistemas de control & 2 \\
     & Codificar las rutinas para la entonación de controladores PID & 2 \\
     & Codificar rutinas para la creación de controladores difusos & 3 \\
     & Codificar rutinas para la simulación de sistemas de control & 2 \\
     &  & \multicolumn{1}{l}{} \\
    \multirow{2}{*}{Fase 3: Interfaz gráfica y enlace con rutinas} & Crear la interfaz gráfica para el laboratorio virtual & 1 \\
     & Acoplar las rutinas con la interfaz gráfica & 2 \\
     &  & \multicolumn{1}{l}{} \\
    \multirow{2}{*}{Fase 4: Comparación de resultados} & Análisis de resultados obtenidos & 1 \\
     & Comparación con otras herramientas & 1 \\ \midrule
     & \multicolumn{1}{r}{Total semanas:} & 17 \\ \bottomrule
    \end{tabular}%
\end{table}
\end{landscape}
}

\afterpage{
\begin{landscape}
	\begin{figure}
	\begin{ganttchart}[vgrid={draw=none,draw=none},%
				%today=15,%
				%today offset=.5,%
				%today label=Heute,%
				%progress=today,%
				x unit=4.2pt,
				y unit chart=0.6cm,
				newline shortcut=true,
				bar/.append style={gray!33},
				canvas/.append style={fill=none},
				group/.style={fill=gray}
				]{0}{120}
	
	\gantttitle{2019}{121} \\
	\gantttitle{julio}{31}
	\gantttitle{agosto}{31}
	\gantttitle{septiembre}{30}
	\gantttitle{octubre}{29}\\
	\gantttitle{}{1}
	\gantttitle{W1}{7}
	\gantttitle{W2}{7}
	\gantttitle{W3}{7}
	\gantttitle{W4}{7}
	\gantttitle{W5}{7}
	\gantttitle{W6}{7}
	\gantttitle{W7}{7} 
	\gantttitle{W8}{7}
	\gantttitle{W9}{7}
	\gantttitle{W10}{7}
	\gantttitle{W11}{7}
	\gantttitle{W12}{7}
	\gantttitle{W13}{7}
	\gantttitle{W14}{7}
	\gantttitle{W15}{7}
	\gantttitle{W16}{7}
	\gantttitle{W17}{7}
	\gantttitle{}{1} \\

	\ganttgroup{Duracion Total}{1}{119} \\
	
	\ganttgroup{Fase 1}{1}{21} \\
	\ganttbar{Actividad 1}{1}{7} \\
	\ganttbar{Actividad 2}{7}{14} \\
	\ganttbar{Actividad 3}{14}{21} \\
	
	\ganttgroup{Fase 2}{21}{84}\\
	\ganttbar{Actividad 4}{21}{35} \\
	\ganttbar{Actividad 5}{35}{49} \\
	\ganttbar{Actividad 6}{49}{70} \\
	\ganttbar{Actividad 7}{70}{84} \\
	
	\ganttgroup{Fase 3}{84}{105} \\
	\ganttbar{Actividad 8}{84}{91} \\
	\ganttlinkedbar[link bulge=3]{Actividad 9}{91}{105} \\

	\ganttgroup{Fase 4}{105}{119} \\
	\ganttbar{Actividad 10}{105}{112} \\
	\ganttlinkedbar[link bulge=3]{Actividad 11}{112}{119} \\
	
	\begin{scope}[on background layer]
	\ganttlink[link bulge=3]{elem2}{elem6}
	\ganttlink[link bulge=3]{elem3}{elem7}
	\ganttlink[link bulge=3]{elem4}{elem8}
	
	\ganttlink[link bulge=3]{elem2}{elem9}
	\ganttlink[link bulge=3]{elem3}{elem9}
	\ganttlink[link bulge=3]{elem4}{elem9}

	\ganttlink[link bulge=3]{elem6}{elem12}
	\ganttlink[link bulge=3]{elem7}{elem12}
	\ganttlink[link bulge=3]{elem8}{elem12}
	\ganttlink[link bulge=3]{elem9}{elem12}
	\ganttlink[link bulge=3]{elem12}{elem14}
	\end{scope}
	\end{ganttchart}
	\caption[Diagrama de Gantt]{\textbf{Diagrama de Gantt}. Fuente: Elaboración propia}
	\label{fig:gantt}
	\end{figure}
	\end{landscape}
}

