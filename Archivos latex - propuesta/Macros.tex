% Formato general
\crefname{equation}{}{}
\crefname{table}{\spanishtablename}{\spanishtablename}

\setcounter{tocdepth}{5}
\setcounter{secnumdepth}{5}

\newcommand\punto[1]{#1.$\;$}
\titleformat{\section}[block]{\LARGE\bfseries}{\thesection}{1em}{}
\titleformat{\subsection}[block]{\hspace{0.9cm}\large\bfseries}{\thesubsection}{1em}{\punto}
\titleformat{\subsubsection}[block]{\hspace{1.92cm}\large\itshape\bfseries}{\thesubsubsection}{1em}{\punto}

\titlespacing\section{0pt}{10pt plus 4pt minus 4pt}{10pt plus 4pt minus 4pt}
\titlespacing\subsection{0pt}{0pt plus 4pt minus 4pt}{0pt plus 4pt minus 4pt}
\titlespacing\subsubsection{0pt}{0pt plus 4pt minus 4pt}{0pt plus 4pt minus 4pt}

\titleformat*{\paragraph}{\large\bfseries}
\titleformat*{\subparagraph}{\large\bfseries}


\titlecontents{chapter}% <section-type>
	[0pt]% <left>
	{\addvspace{1em}}% <above-code>
	{\bfseries\thecontentslabel\quad}% <numbered-entry-format>
	{\bfseries}% <numberless-entry-format>
	{\bfseries\hfill\contentspage}
	
%\let\svsubsubsection\subsubsection %separacion del bloque de texto por seccion
%\def\subsubsection{\leftskip 1.25cm\svsubsubsection} 

\setlength{\textfloatsep}{20pt plus 4.0pt minus 4.0pt}
\setlength{\floatsep}{20pt plus 4.0pt minus 4.0pt}
\setlength{\intextsep}{20pt plus 4.0pt minus 4.0pt}

% Ecuaciones 
\setlength{\abovedisplayskip}{0pt plus 4pt minus 4pt}
\setlength{\belowdisplayskip}{0pt plus 4pt minus 4pt}
\setlength{\abovedisplayshortskip}{8pt plus 4pt minus 4pt}
\setlength{\belowdisplayshortskip}{20pt plus 4pt minus 4pt}

% incio de macro para cambiar chapter al centro
\newcommand{\Centerformat}{
	\titleformat{\chapter}[display]
	{\normalfont\Large\bfseries\filcenter}{\chaptertitlename\ \thechapter}{5pt}{\LARGE}[\vspace{-5pt}\leavevmode]
	\titlespacing{\chapter}
	{0pt}{30pt}{20pt}
}
% --------------------------------------------------------------------------

% incio de macro para cambiar chapter a la izquierda
\newcommand{\CenterformatBack}{
	\titleformat{\chapter}[display]
	{\normalfont\huge\bfseries}{\chaptertitlename\ \thechapter}{20pt}{\Huge}[\vspace{-30pt}\leavevmode]
	\titlespacing{\chapter}
	{0pt}{30pt}{20pt}
}
% --------------------------------------------------------------------------

\setlist[itemize]{leftmargin=1.25cm}

%Inicio del macro para el sortdlist ( ordenar un itemize ) -----------------
\newcommand{\sortitem}[2][\relax]{%
	\DTLnewrow{list}% Create a new entry
	\ifx#1\relax
	\DTLnewdbentry{list}{sortlabel}{#2}% Add entry sortlabel (no optional argument)
	\else
	\DTLnewdbentry{list}{sortlabel}{#1}% Add entry sortlabel (optional argument)
	\fi%
	\DTLnewdbentry{list}{description}{#2}% Add entry description
}
\newenvironment{sortedlist}{%
	\DTLifdbexists{list}{\DTLcleardb{list}}{\DTLnewdb{list}}% Create new/discard old list
}{%
	\DTLsort{sortlabel}{list}% Sort list
	\begin{itemize}%
		\DTLforeach*{list}{\theDesc=description}{%
			\item \theDesc}% Print each item
	\end{itemize}%
}
% --------------------------------------------------------------------------

% para modificar el ambiente " qoute "
\AtBeginEnvironment{quote}{\addtolength\leftmargini{1.25cm}}

% Para modificar el Y en las referencias y bibliografia
\DeclareDelimFormat*{finalnamedelim}
{\ifnum\value{liststop}>2 \finalandcomma\fi\addspace y \space}

% the bibliography also needs another conditional, so we can't wrap
% everything up with just the two lines above
\DeclareDelimFormat[bib,biblist]{finalnamedelim}{%
	\ifthenelse{\value{listcount}>\maxprtauth}
	{}
	{\ifthenelse{\value{liststop}>2}
		{\finalandcomma\addspace y \space}
		{\addspace y \space}}}

% this is a special delimiter to solve the bugs reported in
% https://tex.stackexchange.com/q/417648/35864
\DeclareDelimFormat*{finalnamedelim:apa:family-given}{%
	\ifthenelse{\value{listcount}>\maxprtauth}
	{}
	{\finalandcomma\addspace y\space}}

\pagestyle{plain}
\newcommand{\MainStyle}{}
\newcommand{\FrontBackStyle}{}
\newcommand{\CenterformatFront}{}

% para modificar caption

%\DeclareCaptionLabelSeparator*{spaced}{.\\}
%\DeclareCaptionLabelFormat{numero}{\textbf{#1 #2}}

%\captionsetup[table]{textfont=it, format=plain, justification=justified, singlelinecheck=false, labelsep=newline, skip=0pt}
%\captionsetup[figure]{labelformat=numero,labelsep=period, labelfont=bf, textfont=bf, justification=centering, singlelinecheck=false}