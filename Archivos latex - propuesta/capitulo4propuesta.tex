\section{Factibilidad de la propuesta}

    Esta investigación es claramente factible al estar enfocada en la utilización de software libre para la creación del laboratorio virtual, asi lo demuestran los antecedentes expuestos en esta propuesta, los cuales han hecho uso de software para le creación de aplicaciones con usos en el área de los sistemas de control, otros han utilizado específicamente Python para el desarrollo de controladores difusos, ademas, al tratarse de software libre no existe la necesidad de financiación externa, facilitando asi su desarrollo. Para realizar esta investigación solo hará falta el uso de una computadora con acceso a internet con el fin de obtener el software necesario y realizar las investigaciones pertinentes en cada tema, ademas, se espera hacer uso de la biblioteca de la universidad, nuevamente, para realizar las investigaciones pertinentes.

\section{Cronograma de ejecución}

    Las actividades a realizar fueron determinadas en función de las fases, los requerimientos y objetivos de esta investigación, en la \cref{tab:Cronograma} se puede observar el cronograma de ejecución, en donde se denotan las fases y actividades de la misma junto con el numero de semanas que se estima emplear en cada una de las actividades. La estimación de semanas para cada una de las actividades se tomo considerando la complejidad del tema, a su vez, la complejidad fue evaluada tomando en cuenta los antecedentes expuestos en esta propuesta y teniendo en consideración las experiencias similares previas en cada uno de los temas a abarcar.

\afterpage{
\begin{landscape}
\begin{table}
    \caption{Cronograma de ejecución}
    \label{tab:Cronograma}
    \small
    \begin{tabular}{@{\extracolsep{\fill}}llc}
    \toprule
    \multicolumn{1}{c}{Fases} & \multicolumn{1}{c}{Actividades} & \multicolumn{1}{l}{Semanas} \\ \midrule
    \multirow{3}{*}{Fase 1: Estudio de los sistemas de control clásicos y difusos} & Estudio de los métodos de análisis de sistemas de control & 1 \\
     & Estudio de controladores PID & 1 \\
     & Estudio de controladores difusos tipo Mamdani & 1 \\
     &  & \multicolumn{1}{l}{} \\
    \multirow{4}{*}{Fase 2: Codificación de rutinas} & Codificar las rutinas de análisis de sistemas de control & 2 \\
     & Codificar las rutinas para la entonación de controladores PID & 2 \\
     & Codificar rutinas para la creación de controladores difusos & 3 \\
     & Codificar rutinas para la simulación de sistemas de control & 2 \\
     &  & \multicolumn{1}{l}{} \\
    \multirow{2}{*}{Fase 3: Interfaz gráfica y enlace con rutinas} & Crear la interfaz gráfica para el laboratorio virtual & 1 \\
     & Acoplar las rutinas con la interfaz gráfica & 2 \\
     &  & \multicolumn{1}{l}{} \\
    \multirow{2}{*}{Fase 4: Comparación de resultados} & Análisis de resultados obtenidos & 1 \\
     & Comparación con otras herramientas & 1 \\ \midrule
     & \multicolumn{1}{r}{Total semanas:} & 17 \\ \bottomrule
    \end{tabular}%
\end{table}
\end{landscape}
}