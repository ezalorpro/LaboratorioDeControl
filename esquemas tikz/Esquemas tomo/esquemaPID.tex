\documentclass[tikz, border=0mm]{standalone}

\usetikzlibrary{shapes,arrows,positioning, calc}

\begin{document}

\tikzset{%
    block1/.style={
        draw,
        fill=white, 
        rectangle, 
        minimum height=2em, 
        minimum width=3em,
        node distance=0.9cm
    },
    input/.style={
        inner sep=0pt, 
        node distance=0.6cm
    },      
    output/.style={
        inner sep=0pt, 
        node distance=0.6cm
    },      
    sum/.style = {
        draw, 
        fill=white, 
        circle, 
        minimum size=2mm, 
        inner sep=0pt,
        node distance=0.6cm
    },
    pinstyle/.style = {
        pin edge={to-,thin,black}
    },
    branch/.style={
        fill,
        circle,
        minimum size=3pt,
        outer sep=-1pt,
        inner sep=0pt, 
        node distance=0.6cm
    },
    EmptyBlock/.style={
        circle,
        minimum size=0pt,
        outer sep=-1pt,
        inner sep=-1pt, 
        node distance=2cm
    }
}

\begin{tikzpicture}[auto, >=latex']

    \node[input] (input) {};
    \node [sum, right = of input] (sum) {};
    \node [block1, right = of sum] (controlador) {$PID$};
    \node [block1, right = of controlador] (proceso) {$Proceso$};
    \node [branch, right = of proceso] (yjoint) {};
    \node [output, right = of yjoint] (output) {};
    \coordinate[node distance=0.6cm, below = of proceso] (empty1) {};    

    \draw [->] (input) node[above right] {$sp$} -- (sum);
    \draw [->] (sum) -- (controlador) node[pos=0.9, above left] {$e(t)$};
    \draw [->] (controlador) -- (proceso) node[pos=1, above left] {$sc(t)$};
    \draw [->] (proceso) -- (output) node [name=y, above left] {$Vp$};
    \draw [-] (yjoint.south) |- (empty1);
    \draw [->] (empty1) -| node[pos=0.97, right] {\tiny{$-$}} (sum);

\end{tikzpicture}
\end{document}