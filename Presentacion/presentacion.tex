\documentclass[usenames,dvipsnames]{beamer}
\usepackage[utf8]{inputenc}
\usepackage{verbatim}
\usetheme{uniud}
\usepackage{xparse}
\usepackage{subcaption}
\ExplSyntaxOn
\NewDocumentCommand{\convertto}{mm}
% #1 = em or ex (or any other unit)
% #2 = dimen to convert
{
	\texttt{#2~=~\fp_to_decimal:n { (#2)/(1#1) }#1}
}
\ExplSyntaxOff

%%% Some useful commands
% pdf-friendly newline in links
\newcommand{\pdfnewline}{\texorpdfstring{\newline}{ }} 
% Fill the vertical space in a slide (to put text at the bottom)
\newcommand{\framefill}{\vskip0pt plus 1filll}


\title[Universidad Nacional Experimental del Tachira]{Laboratorio virtual de sistemas de control clasicos y difusos utilizando software libre}
\date[Marzo 2020]{Marzo 05, 2020}
\author[Proyecto Especial de Grado]{
  \textbf{Autor}: \hfill \\ Br. Kleiver J. Carrasco M. \\ \vspace{10pt} \textbf{Tutor}: \\ MSc. Ing. Juan R. Vizcaya R.
}

\institute{
	Universidad Nacional Experimental del Tachira

	Vicerrectorado Academico
	
	Decanato de Docencia
	
Departamento de Electronica
}

\begin{document}

\begin{frame}
	\titlepage
\end{frame}

\begin{frame}
	\frametitle{INTRODUCCIÓN}
	\vspace{30pt}
	\begin{itemize}
		\Large
		\setlength\itemsep{1em} 
		\item Planteamiento del problema
		\begin{itemize}
			\large
			\vspace{10pt} 
			\item[--] ¿Es posible realizar un laboratorio para el análisis de sistemas de control con software libre?
			\item[--] ¿Cumpliría con los requisitos para analizar, diseñar y simular sistemas de control?
			\item[--] ¿Cómo se desempeñaría en comparación con otras herramientas?
		\end{itemize}
		\item ¿Por qué ``Laboratorio Virtual''?
		%\item Importancia
	\end{itemize}
	
	
\begin{figure}
	\centering
	\begin{subfigure}[b]{0.12\linewidth}
		\includegraphics[width=\linewidth]{imagenes/logoMATLAB}
	\end{subfigure}
	\hspace{0.8cm}
	\begin{subfigure}[b]{0.17\linewidth}
		\includegraphics[width=\linewidth]{imagenes/logoSciLab}
	\end{subfigure}
	
\end{figure}
\end{frame}

\begin{frame}
	\frametitle{OBJETIVOS}
	\vspace{20pt}
	\begin{block}{Objetivo General}
		Desarrollar un laboratorio virtual de sistemas de control clásicos y difusos utilizando software libre.
	\end{block}
	
	\begin{block}{Objetivos específicos}
		\begin{enumerate} 
			\item Estudiar los sistemas de control clásicos.
			\item Estudiar el diseño de controladores difusos tipo Mamdani.
			\item Codificar las rutinas de análisis, diseño y simulación de sistemas de control necesarias.
			\item Realizar la interfaz gráfica de un laboratorio de sistemas de control virtual.
			\item Comparar los resultados obtenidos con dos herramientas de corte similar.
		\end{enumerate}
	\end{block}
	
\end{frame}

\end{document}