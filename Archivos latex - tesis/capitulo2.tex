\section{Antecedentes}
	
	Para esta investigación se buscaron antecedentes de alcance nacional e internacional que ayuden a guiar, sustentar y validar esta investigación.
	
	\textcite{casallas2005desarrollo}, escribieron para la revista Acción Pedagógica de la Universidad de los Andes un artículo titulado \enquote{Desarrollo básico de un Laboratorio Virtual de Control de Procesos basado en Internet}, fue desarrollado conjuntamente entre la UNET y la ULA para ser usado a través de Internet y cuyo objetivo fue desarrollar un laboratorio virtual de control de procesos para la enseñanza a distancia, permitiendo ser utilizado solo por usuarios registrados y en determinados horarios. Aunque la finalidad de la investigación acá propuesta difiere del objetivo de \citeauthor{casallas2005desarrollo}, esta servirá como ayuda para determinar las funciones más utilizadas en el área de control, así como las necesidades típicas de un laboratorio de control virtual.
	
	Adicionalmente, \textcite{salazar2019diseno}, realizo en la Universidad Técnica del Norte, Ecuador, una tesis titulada \enquote{Diseño de un sistema de riego inteligente para cultivos de hortalizas basado en Fuzzy Logic en la granja la pradera de la Universidad Técnica del Norte.} para la implementación del controlador basado en la lógica difusa utilizo, entre otros equipos, un Rasberry PI para la ejecución de Python, específicamente, hizo uso de la biblioteca Scikit-Fuzzy, la cual le permitió definir las funciones de membresía, establecer las reglas difusas, realizar los procesos de fuzzificación y defuzzificación, todo esto para una arquitectura de controlador difuso tipo Mamdani, esta tesis servirá como referencia para establecer el diseño de controladores difusos utilizando la biblioteca Scikit-Fuzzy.

	Por otro lado, \textcite{congo2018aplicaciones} realizo en la Universidad Tecnológica Israel, Ecuador, una tesis titulada \enquote{Aplicaciones del software libre Python para prácticas de laboratorio aplicado a la asignatura de tratamiento digital de señales de la Universidad Tecnológica Israel} cuyo objetivo fue desarrollar por medio del software Python y sus diferentes librerías científicas, la realización de 3 Prácticas de Laboratorio para la asignatura Procesamiento Digital de Señales de la Universidad Tecnológica Israel, al igual que la intención de este trabajo, se realizó una interfaz gráfica con el objetivo de facilitar su uso, en este caso como herramienta para el procesamiento digital de señales. Este trabajo sustenta la viabilidad de realizar una interfaz gráfica con enfoque similar en el campo de la electrónica y se utilizara como guía parcial para establecer su uso en sistemas de control.
	
	Finalmente, \textcite{cadavid2009toolbox} redacto para la revista Educación en Ingeniería de Colombia, un artículo titulado \enquote{Toolbox didáctico para el diseño y análisis de sistemas de control lineal}, cuyo objetivo es describir un toolbox realizado en MATLAB para el análisis de sistemas de control lineales, este toolbox consiste de una interfaz gráfica que permite realizar pruebas a sistemas de control como respuesta escalón, respuesta en frecuencia, análisis de estabilidad, diseño de controladores, entre otras funciones. Este articulo será de utilidad para establecer la interfaz gráfica que se pretende realizar, además, sirvió como punto de partida para determinar las funciones que debería tener un laboratorio virtual de sistemas de control, como punto adicional se puede resaltar que reafirma la utilidad de esta investigación. 
	
\section{Bases teóricas}
	
    Para poder realizar el laboratorio virtual de sistemas de control clásicos y difusos será necesario abarcar conocimientos de análisis de sistemas de control, diseño de controladores PID y controladores difusos, a continuación, se presentan los conceptos necesarios para el desarrollo de esta investigación.
    
    \subsection{Procesos}
		
		Es importante partir de la base, es por ellos que empezaremos con los procesos, \textcite{sanchez2003control} define los procesos como \enquote{[...] un bloque que se identifica porque tiene una o más variables de salida de las cuales es importante conocer y mantener sus valores}(p.$\,$153). Así, un proceso se caracteriza por tener una entrada y una salida (para procesos SISO), dicha salida deberá mantenerse alrededor de un punto de referencia dado, es decir, se debe controlar su salida.
	
	\subsection{Modelado de procesos en tiempo continuo}
	
		Un proceso puede ser representado por ecuaciones diferenciales que determinen su comportamiento dinámico en el dominio del tiempo, no obstante, trabajar con ecuaciones diferenciales es tedioso a nivel matemático, por tanto, se prefiere modelar los procesos utilizando la transformada de Laplace para llevar de una representación dinámica a una algebraica, esta ecuación algebraica se operara para llevar a una función de transferencia que represente al proceso en el dominio de la frecuencia compleja \Parencite{smith1985principles}.
	
		\subsubsection{Transformada de Laplace}
		
			La transformada de Laplace de una función dependiente del tiempo f(t) viene dada por la ecuación:
			
			\begin{equation}\label{eq:Laplace}
				F(s) = \mathcal{L}\left[f(t) \right] = \int_{0}^{\infty} f(t)e^{-st}dt
			\end{equation}
			
			Si suponemos que f(t) es un proceso cuya representación dinámica viene dada por una ecuación diferencial de la forma:
			
			\begin{equation}\label{eq:Diffeq}
				a_{2}\frac{d^{2}y(t)}{dt^{2}} + a_{1}\frac{dy(t)}{dt} + a_{0}y(t) = b_{0}x(t)
			\end{equation}
			
			Aplicando \cref{eq:Laplace} a \cref{eq:Diffeq}  y despejando la relación $Y(s)/X(s)$ para el proceso con condiciones iniciales igual a cero (debido a que el proceso debe ser lineal) se obtiene la correspondiente función de transferencia del proceso $H(S)$:
			
			 \begin{equation}\label{eq:TransferFunction}
			 	H(s) =	\frac{Y(s)}{X(s)} = \frac{b_{0}}{a_{2}s^{2} + a_{1}s + a_{0}}
			 \end{equation}
			 
			 La demostración completa \Parencite[pp.$\,$21-22]{smith1985principles} no es de interés para este trabajo, pero si su resultado, la función de transferencia puede ser analizada para determinar las características del proceso, como su ganancia, constante de tiempo, tiempo muerto, entre otras.
			 
		 \subsubsection{Ecuaciones de espacio de estado}
		 
		 	Las ecuaciones de espacio de estado son un método más moderno para modelar todo tipo de sistemas, no solo físicos, sino también biológicos, económicos, sociales y otros. Así como una ecuación diferencial puede ser representada como una función de transferencia también se puede representar como una ecuación de espacio de estados, por tanto, una función de transferencia también puede ser representada en el espacio de estados y viceversa. Las ecuaciones linealizadas alrededor de un estado de operación \Parencite[p.$\,$31]{ogata2003ingenieria} son:
		 
			\begin{align}
                \dot{x}(t) &= Ax(t) + Bu(t) \label{eq:SSrepresentationX} \\
				y(t) &= Cx(t) + Du(t) \label{eq:SSrepresentationY}
			\end{align}
			
			\begin{spacing}{1.5}
				con: 
				
				$A$: como matriz de estado.
				
				$B$: como matriz de entrada.
				
				$C$: como matriz de salida.
				
				$D$: como matriz de transmisión directa.
				
            \end{spacing}
            
            Normalmente la notación presentada en las ecuaciones \cref{eq:SSrepresentationX,eq:SSrepresentationY} es preferida dado que son compactas y fáciles de entender, no obstante, es al expandir las ecuaciones en su forma matricial que podemos ver claramente que la ecuación \cref{eq:SSrepresentationX} es un sistema de ecuaciones diferenciales ordinarias:
            
            \vspace{20pt}
            \begin{align}
                \begin{bmatrix}
                    \dot{x}_{1}\\
                    \dot{x}_{2}\\
                    \vdots\\
                    \dot{x}_{n}\\
                    \end{bmatrix}&=
                    \begin{bmatrix}
                    a_{11} & a_{12} & \cdots & a_{1n}\\
                    a_{21} & a_{22} & \cdots & a_{2n}\\
                    \vdots & \vdots & \ddots & \vdots\\
                    a_{n1} & a_{n2} & \cdots & a_{nn}\\
                    \end{bmatrix}
                    \begin{bmatrix}
                    x_{1}\\
                    x_{2}\\
                    \vdots\\
                    x_{n}\\
                    \end{bmatrix}+
                    \begin{bmatrix}
                    b_{11} & b_{12} & \cdots & b_{1m}\\
                    b_{21} & b_{22} & \cdots & b_{2m}\\
                    \vdots & \vdots & \ddots & \vdots\\
                    b_{n1} & b_{n2} & \cdots & b_{nm}\\
                    \end{bmatrix}.
                    \begin{bmatrix}
                    u_{1}\\
                    u_{2}\\
                    \vdots\\
                    u_{m}\\
                    \end{bmatrix}\\
                    \begin{bmatrix}
                    y_{1}\\
                    y_{2}\\
                    \vdots\\
                    y_{n}\\
                    \end{bmatrix}&=
                    \begin{bmatrix}
                    c_{11} & c_{12} & \cdots & c_{1n}\\
                    c_{21} & c_{22} & \cdots & c_{2n}\\
                    \vdots & \vdots & \ddots & \vdots\\
                    c_{r1} & c_{r2} & \cdots & c_{rn}\\
                    \end{bmatrix}
                    \begin{bmatrix}
                    x_{1}\\
                    x_{2}\\
                    \vdots\\
                    x_{n}\\
                    \end{bmatrix}+
                    \begin{bmatrix}
                    d_{11} & d_{12} & \cdots & d_{1m}\\
                    d_{21} & d_{22} & \cdots & d_{2m}\\
                    \vdots & \vdots & \ddots & \vdots\\
                    d_{r1} & d_{r2} & \cdots & d_{rm}\\
                    \end{bmatrix}.
                    \begin{bmatrix}
                    u_{1}\\
                    u_{2}\\
                    \vdots\\
                    u_{m}\\
                    \end{bmatrix}
            \end{align}
            \vspace{20pt}

            \begin{spacing}{1.5}
				con: 
				
				$n$: como el numero de variables de estado
				
				$m$: como el numero de entradas
				
				$r$: como el numero de salidas
				
            \end{spacing}

            Para resolver estos sistemas de ecuaciones diferenciales podemos utilizar metodos iterativos como los metodos de Runge-Kuuta.

    \subsection{Modelado de procesos en tiempo discreto}

        Así como para tiempo continuo se utilizan ecuaciones diferenciales, en el tiempo discreto se hace uso de ecuaciones en diferencias, las ecuaciones en diferencias se pueden utilizar para aproximar a las ecuaciones diferenciales, estas primeras se suelen utilizar porque son más fáciles de programar \Parencite{kuo1996sistemas}, debido a esto los procesos pueden ser representados de la siguiente forma:
        
        \begin{equation}\label{eq:EqEnDiferencias}
            y(k+n) + a_{n-1}y(k+n-1) + \cdots + a_1 y(k+1) + a_0 y(k) = f(k) 
        \end{equation}
        
        Asi mismo debido a que la ecuacion \cref{eq:SSrepresentationX} esta compuesta de ecuaciones diferenciales, esta puede ser llevada a una ecuacion en diferencias y ser representada de forma discreta como sigue:

        \begin{align}\label{eq:SSdiscreto}
            \dot{x}(k+1) &= Ax(k) + Bu(k) \\
            y(k) &= Cx(k) + Du(k)
        \end{align}

        \subsubsection{Transformada z}
		
			En tiempo discreto se puede modelar un proceso utilizando la transformada z sobre la ecuación en diferencias del proceso para obtener la función de transferencia discreta $H(z)$, la ecuación general para la transformada z es:
			
			\begin{equation}\label{eq:Ztransform}
				F(z)= \sum\limits_{k=0}^{\infty}f(k)z^{-k}
			\end{equation}
