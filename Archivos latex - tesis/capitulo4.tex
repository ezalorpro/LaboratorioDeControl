\section{Estructura general}

    El laboratorio virtual de sistemas de control clásicos y difusos fue pensado de modo que cada una de sus funcionalidades principales sean independientes la una de la otra, no obstante, ciertas rutinas son comunes entre ellas. Por otro lado, se organizo la aplicación de modo que cada funcionalidad principal posea un archivo \enquote{handler}, el cual hará de intermediario entre la interfaz gráfica y las rutinas de calculo que se encuentran en un archivo de rutinas. La Figura 4 sirve como referencia de la estructura general utilizada.


\section{Función de análisis de sistemas de control}
    
    Con esta funcion se quiso dar la posibilidad de realizar los análisis en el tiempo y en frecuencia tipicos para un proceso ingresado por el usuario, estos analisis tipicos son: respuesta al escalon, respuesta al impulso, diagrama de bode, diagrama de Nyquist, lugar de las raices y diagrama de Nichols.

    \subsection{Definición del proceso}

        Un proceso deberá ser ingresado para poder realizar el análisis, la entonación o la simulación del mismo. El sistema puede ser definido con los coeficientes del numerador y el denominador de la función de transferencia o puede ser definido ingresando las matrices A, B, C y D de la ecuación de espacio de estados. En ambos casos se deben cumplir con los principios matemáticos que correspondan, i.g., las función de transferencia no puede ser impropia.

    \subsection{Tiempo discreto}

        La opción de discretizacion permite llevar el proceso ingresado en tiempo continuo a una aproximación en tiempo discreto con solo ingresar el periodo de muestreo y seleccionar el método a utilizar. Se implementaron los siguientes métodos de discretizacion:

        \begin{enumerate}[leftmargin=\parindent]
            \item Retensor de orden cero  (ZOH)
            \item Retensor de primer orden (FOH)
            \item Euler hacia atrás (backward\_diff)
            \item Euler hacia adelante (Euler)
            \item Transformacion trapezoidal (tustin)
            \item Transformacion matched (matched)
            \item Transformacion de impulso (impulse)
        \end{enumerate}

    \subsection{Proceso con atraso o Delay}
        
        