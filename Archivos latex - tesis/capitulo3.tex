En este capítulo se abarcará la metodología empleada en el desarrollo de esta investigación, se definirá el tipo, diseño y modalidad de la investigación, así como las fases de la misma.

\section{Tipo de investigación}
	
	Tomando en cuenta los objetivos de la investigación y las bases teóricas que la componen se considera que esta investigación es de tipo proyectiva, esto es debido a que se realizó una propuesta concreta para solventar una problemática, \textcite{jacquelin2010guia} afirma que:	
	
	\blockquote[p.$\,$133]{La investigación proyectiva tiene como objetivo diseñar o crear propuestas dirigidas a resolver determinadas situaciones. Los proyectos de arquitectura e ingeniería, el diseño de maquinarias, la creación de programas de intervención social, el diseño de programas de estudio, los inventos, la elaboración de programas informáticos, entre otros, siempre que estén sustentados en un proceso de investigación, son ejemplos de investigación proyectiva.}

\section{Diseño de la investigación}

	El diseño de la investigación es no experimental y de tipo transeccional descriptivo, esto es debido a que se describieron los métodos de análisis y diseño de sistemas de control clásicos y difusos, \textcite{sampieri1998metodologia} definen la investigación no experimental como: \blockquote[p.150]{la investigación que se realiza sin manipular deliberadamente variables. Es decir,
	se trata de estudios donde no hacemos variar en forma intencional las variables independientes para
	ver su efecto sobre otras variables. Lo que hacemos en la investigación no experimental
	es observar fenómenos tal como se dan en su contexto natural, para posteriormente
	analizarlos.}

\section{Modalidad}

	Esta investigación se encuentra enmarcada en la modalidad de un proyecto factible, debido a que tiene objetivos para atender una necesidad por medio de unas acciones claramente definidas, \textcite{renie2002factible} afirma que:
	
	\blockquote[pp.$\,$6-7]{un proyecto factible consiste en un conjunto de actividades vinculadas entre sí, cuya ejecución permitirá el logro de objetivos previamente definidos en atención a las necesidades que pueda tener una institución o un grupo social en un momento determinado. Es decir, la finalidad del proyecto factible radica en el diseño de una propuesta de acción dirigida a resolver un problema o necesidad previamente detectada en el medio.}

\section{Fases de la investigación}
	
	\paragraph{Fase 1: Estudio de los sistemas de control clásicos y difusos}
		
		En esta fase se procedió a realizar los estudios necesarios en el área de los sistemas de control, esto con la idea de abarcarlos en profundidad y tener un entendimiento claro de su funcionamiento y de la matemática implicada, además, se realizó de forma similar un estudio de controladores difusos con estructura Mamdani y de los esquemas de control difuso.
		
	\paragraph{Fase 2: Codificación de rutinas}
		
		Para esta fase con los conocimientos adquiridos de la fase 1, se determinaron que rutinas podían ser ejecutadas solo con las bibliotecas de Python y cuales debían ser codificadas de cero, además, se codificaron todas las rutinas necesarias para el funcionamiento del Laboratorio Virtual, para esto, se hizo uso de las bibliotecas externas de cálculo numérico, control, diseño de controladores difusos y salidas gráficas junto con las que se consideraron necesarias.
		
	\paragraph{Fase 3: Interfaz gráfica y enlace con rutinas}
		
		En esta fase se realizó la interfaz gráfica para el usuario final, esta interfaz gráfica permite conectarse y adaptarse a las rutinas previamente codificadas en la fase 2 para su funcionamiento adecuado, en orden de tener un diseño acorde se tomaron en cuenta los antecedentes presentados en el capítulo 2.
		
	\paragraph{Fase 4: Comparación de resultados}
		
		En esta última fase se procedió a analizar los resultados obtenidos y a compararlos con otras herramientas, la evaluación se realizó en función del resultado esperado, facilidad de implementación, velocidad de ejecución, las ventajas y desventajas de cada una de las herramientas.
