% Formato tesis UNET y numerado
% Autor: Kleiver Carrasco
% Correo: kleiver615@gmail.com

\addto\captionsspanish{\renewcommand{\contentsname}%
	{Índice}%
}

\addto\captionsspanish{\renewcommand{\listtablename}%
	{Índice de tablas}%
}

\addto\captionsspanish{\renewcommand{\listfigurename}%
	{Índice de figuras}%
}

\addto\captionsspanish{\renewcommand{\listlistingname}
	{Índice de códigos}%
}


% Algunas configuraciones generales --------------------------------------
%\renewcommand{\chaptername}{Etapa}		% para cambiar de nombre los capítulos
\renewcommand{\thefootnote}{(\arabic{footnote})}
\renewcommand{\labelitemi}{$\bullet$}
\setlength{\parindent}{1.5cm}
\parskip=0pt plus 1pt
\quotingsetup{leftmargin=\parindent, rightmargin=\parindent}

\crefname{equation}{}{}
\crefname{listing}{Código}{Códigos}
\SetupFloatingEnvironment{listing}{name=Código}
\crefname{table}{\spanishtablename}{\spanishtablename}
\newcommand{\addperiod}[1]{#1.$\enspace$}
% ------------------------------------------------------------------------


% --------------------------------------------
% FORMATO UNET -------------------------------

\newcommand{\UnetFormat}{
	
	\counterwithout{figure}{chapter}
	\counterwithout{table}{chapter}
	\counterwithout{equation}{chapter}
	\counterwithout{listing}{chapter}
	\setcounter{tocdepth}{1}
	\setcounter{secnumdepth}{0}
	
	\renewcommand{\thechapter}{\Roman{chapter}}
	\titleformat{\section}[block]{\centering\normalsize\bfseries}{\thesection}{1em}{}
	\titleformat{\subsection}[block]{\centering\normalsize\bfseries\itshape}{\thesubsection}{1em}{}
	\titleformat{\subsubsection}[block]{\normalsize\bfseries\itshape}{\thesubsubsection}{1em}{}
	\titleformat{\paragraph}[runin]{\normalsize\bfseries\itshape }{\theparagraph}{1em}{\addperiod}
	
	\titlespacing\section{0pt}{20pt plus 4pt minus 4pt}{20pt plus 4pt minus 4pt}
	\titlespacing\subsection{0pt}{10pt plus 4pt minus 4pt}{10pt plus 4pt minus 4pt}
	\titlespacing\subsubsection{0pt}{10pt plus 4pt minus 4pt}{10pt plus 4pt minus 4pt}
	\titlespacing{\paragraph}{1.5cm}{10pt plus 0pt minus 4pt}{0pt plus 0pt minus 0pt}
	
	% setup de caption y ambiente de tablas
	\captionsetup[table]{textfont=bf, format=plain, justification=justified, singlelinecheck=false, labelsep=spaced, belowskip=0pt, aboveskip=1pt}
		
	% setup de caption y ambiente de figuras
	\captionsetup[figure]{labelformat=simple,labelsep=period, labelfont=bf, textfont=footnotesize, justification=justified, singlelinecheck=false, belowskip=-10pt}
	
	\captionsetup[subfigure]{labelsep=period, labelfont=bf, textfont=normal, justification=centering, singlelinecheck=false, belowskip=0pt}
	
	\captionsetup[listing]{format=plain, justification=justified, singlelinecheck=false, labelsep=period, textfont=footnotesize}

	\setlength{\textfloatsep}{15pt plus 1.0pt minus 2.0pt}
	\setlength{\floatsep}{15pt plus 1.0pt minus 2.0pt}
	\setlength{\intextsep}{15pt plus 1.0pt minus 2.0pt} 
	

	% Configuracion del TOC	
	\titlecontents{section}% <section-type>
		[1.5cm]% <left>
		{\addvspace{-2pt}}% <above-code>
		{\makebox[0.78cm][l]{\thecontentslabel}\hspace{1.7em}}% <numbered-entry-format>
		{}% <numberless-entry-format>
		{\titlerule*[8pt]{.}\contentspage}[\addvspace{-2pt}]

	\titlecontents{subsection}% <section-type>
		[3cm]% <left>
		{\addvspace{-2pt}}% <above-code>
		{\makebox[0.63cm][l]{\thecontentslabel}\hspace{2.1em}}% <numbered-entry-format>
		{}% <numberless-entry-format>
		{\titlerule*[8pt]{.}\contentspage}[\addvspace{-2pt}]

	\titlecontents{subsubsection}% <section-type>
		[4.5cm]% <left>
		{\addvspace{-2pt}}% <above-code>
		{\makebox[20pt][l]{\thecontentslabel}\hspace{3.1em}}% <numbered-entry-format>
		{}% <numberless-entry-format>
		{\titlerule*[8pt]{.}\contentspage}[\addvspace{-2pt}]

	\titlecontents{paragraph}% <section-type>
		[6.52cm]% <left>
		{\addvspace{0pt}}% <above-code>
		{\makebox[20pt][l]{\thecontentslabel}\qquad\qquad}% <numbered-entry-format>
		{}% <numberless-entry-format>
		{\titlerule*[8pt]{.}\contentspage}[\addvspace{0pt}]

	% Comando para agregar anexos en formato UNET
	\newcounter{LAnexos} % contador
	\newcounter{contaAnexos} % contador
	\renewcommand{\theLAnexos}{\Alph{LAnexos}}
	\renewcommand{\thecontaAnexos}{[Anexo~\Alph{contaAnexos}]}

	\newcommand{\AgregarAnexo}[2]{%
		\clearpage%
		\newpage%
		\begin{center}%
		\stepcounter{LAnexos}%
		\refstepcounter{contaAnexos}%
		\thecontaAnexos\label{##2}\par%
		[##1]%
		\vspace{10pt}%
		\leavevmode
		\end{center}
		
	}

	% Configuracion de capitulos -----------------
	\newcommand{\Centerformat}{
		\setcounter{secnumdepth}{0}
		\titleformat{\chapter}[display]
		{\large\bfseries\centering}{\MakeUppercase{\chaptertitlename\ \thechapter}}{0pt}{\large\MakeUppercase}
		\titlespacing{\chapter}
		{0pt}{-5pt}{20pt}
		
		\titlecontents{chapter}% <section-type>
			[3.22cm]% <left>
			{\addvspace{7pt}}% <above-code>
			{\hspace{-3.22cm}\MakeUppercase{\bfseries\chaptername}\hspace{0.1em}
			\makebox[10pt][l]{\bfseries\thecontentslabel}\bfseries\quad\uppercase}% <numbered-entry-format>
			{\hspace{-3.22cm}\bfseries\uppercase}% <numberless-entry-format>
			{\hfill\contentspage}[\addvspace{7pt}]
	}

	\newcommand{\CenterformatBack}{
		\titleformat{\chapter}[display]
		{\large\bfseries\centering}{\MakeUppercase{\chaptertitlename\ \thechapter}}{0pt}{\large\MakeUppercase}
		\titlespacing{\chapter}
		{0pt}{-15pt}{30pt}
		
		\titlecontents{chapter}% <section-type>
			[0pt]% <left>
			{\addvspace{0pt}}% <above-code>
			{}% <numbered-entry-format>
			{\bfseries\normalsize\uppercase}% <numberless-entry-format>
			{\titlerule*[8pt]{.}\contentspage}[\addvspace{0pt}]
	}

	\newcommand{\CenterformatFront}{
		\setcounter{secnumdepth}{-1}
		\titleformat{\chapter}[display]
		{\large\bfseries\centering}{\MakeUppercase{\chaptertitlename\ \thechapter}}{0pt}{\large\MakeUppercase}
		\titlespacing{\chapter}
		{0pt}{-25pt}{10pt}
		
		\titlecontents{chapter}% <section-type>
			[0pt]% <left>
			{\addvspace{0pt}}% <above-code>
			{}% <numbered-entry-format>
			{\bfseries\normalsize\uppercase}% <numberless-entry-format>
			{\titlerule*[8pt]{.}\contentspage}[\addvspace{0pt}]
	}
}



% FORMATO UNET FIN ---------------------------
% --------------------------------------------

% --------------------------------------------
% FORMATO NUMERADO ---------------------------

\newcommand{\NumeradoFormat}{
	
	\setcounter{tocdepth}{4}
	\setcounter{secnumdepth}{4}
	\renewcommand{\thechapter}{\Roman{chapter}}
	
	\renewcommand{\thesection}{\arabic{chapter}.\arabic{section}}
	\renewcommand{\thesubsection}{\arabic{chapter}.\arabic{section}.\arabic{subsection}}
	\renewcommand{\thesubsubsection}{\arabic{chapter}.\arabic{section}.\arabic{subsection}.\arabic{subsubsection}}
	\renewcommand{\theparagraph}{\arabic{chapter}.\arabic{section}.\arabic{subsection}.\arabic{subsubsection}.\arabic{paragraph}}
	\renewcommand{\thefigure}{\arabic{chapter}.\arabic{figure}}
	\renewcommand{\thetable}{\arabic{chapter}.\arabic{table}}
	\renewcommand{\theequation}{\arabic{chapter}.\arabic{equation}}
	\renewcommand{\thelisting}{\arabic{chapter}.\arabic{listing}}
	
	\titleformat{\section}[block]{\normalsize\bfseries}{\thesection}{1em}{}
	\titleformat{\subsection}[block]{\normalsize\bfseries}{\thesubsection}{1em}{}
	\titleformat{\subsubsection}[block]{\normalsize\bfseries\itshape}{\thesubsubsection}{1em}{}
	\titleformat{\paragraph}[runin]{\normalsize\bfseries\itshape }{\theparagraph}{1em}{\addperiod}
	
	\titlespacing\section{0pt}{20pt plus 4pt minus 4pt}{20pt plus 4pt minus 4pt}
	\titlespacing\subsection{1.5cm}{10pt plus 4pt minus 4pt}{10pt plus 4pt minus 4pt}
	\titlespacing\subsubsection{1.5cm}{10pt plus 4pt minus 4pt}{10pt plus 4pt minus 4pt}
	\titlespacing{\paragraph}{1.5cm}{10pt plus 0pt minus 4pt}{0pt plus 4pt minus 4pt}
	
	% setup de caption y ambiente de tablas
	\captionsetup[table]{textfont=bf, format=plain, justification=justified, singlelinecheck=false, labelsep=spaced, belowskip=0pt, aboveskip=1pt}
	
	% setup de caption y ambiente de figuras
	\captionsetup[figure]{labelformat=simple,labelsep=period, labelfont=bf, textfont=footnotesize, justification=justified, singlelinecheck=false, belowskip=-10pt}
	
	\captionsetup[subfigure]{labelsep=period, labelfont=bf, textfont=normal, justification=centering, singlelinecheck=false, belowskip=0pt}
	
	\captionsetup[listing]{format=plain, justification=justified, singlelinecheck=false, labelsep=period, textfont=footnotesize}

	\setlength{\textfloatsep}{15pt plus 1.0pt minus 2.0pt}
	\setlength{\floatsep}{15pt plus 1.0pt minus 2.0pt}
	\setlength{\intextsep}{15pt plus 1.0pt minus 2.0pt}
	
	% Configuracion del TOC
	\newlength\sectionlenght
	\setlength{\sectionlenght}{-0.5cm - \widthof{\thesection}}

	\newlength\subsectionlenght
	\setlength{\subsectionlenght}{-0.5cm - \widthof{\thesubsection}}

	\newlength\subsubsectionlenght
	\setlength{\subsubsectionlenght}{-0.5cm - \widthof{\thesubsubsection}}

	\newlength\paragraphlenght
	\setlength{\paragraphlenght}{-0.5cm - \widthof{\theparagraph}}

	\titlecontents{section}% <section-type>
		[0.5cm + \widthof{\thesection}]% <left>
		{\addvspace{-2pt}}% <above-code>
		{\hspace{\sectionlenght}\makebox[\widthof{\thecontentslabel}][l]{\thecontentslabel}\hspace{0.5cm}}% <numbered-entry-format>
		{}% <numberless-entry-format>
		{\titlerule*[8pt]{.}\contentspage}[\addvspace{-2pt}]
	
	\titlecontents{subsection}% <section-type>
		[0.5cm + \widthof{\thesection} + 0.5cm + \widthof{\thesubsection}]% <left>
		{\addvspace{-2pt}}% <above-code>
		{\hspace{\subsectionlenght}\makebox[\widthof{\thecontentslabel}][l]{\thecontentslabel}\hspace{0.5cm}}% <numbered-entry-format>
		{}% <numberless-entry-format>
		{\titlerule*[8pt]{.}\contentspage}[\addvspace{-2pt}]
		
	\titlecontents{subsubsection}% <section-type>
		[1cm + \widthof{\thesection} + \widthof{\thesubsection} + 0.5cm + \widthof{\thesubsubsection}]% <left>
		{\addvspace{-2pt}}% <above-code>
		{\hspace{\subsubsectionlenght}\makebox[\widthof{\thecontentslabel}][l]{\thecontentslabel}\hspace{0.5cm}}% <numbered-entry-format>
		{}% <numberless-entry-format>
		{\titlerule*[8pt]{.}\contentspage}[\addvspace{-2pt}]
		
	\titlecontents{paragraph}% <section-type>
		[1.5cm + \widthof{\thesection} + \widthof{\thesubsection} + \widthof{\thesubsubsection} + 0.5cm + \widthof{\theparagraph}]% <left>
		{\addvspace{0pt}}% <above-code>
		{\hspace{\paragraphlenght}\makebox[\widthof{\thecontentslabel}][l]{\thecontentslabel}\hspace{0.5cm}}% <numbered-entry-format>
		{}% <numberless-entry-format>
		{\titlerule*[8pt]{.}\contentspage}[\addvspace{0pt}]

	% Comando para agregar anexos en formato UNET
	\newcounter{LAnexos} % contador
	\newcounter{contaAnexos} % contador
	\renewcommand{\theLAnexos}{\Alph{LAnexos}}
	\renewcommand{\thecontaAnexos}{[Anexo~\Alph{contaAnexos}]}
	
	\newcommand{\AgregarAnexo}[2]{%
		\clearpage%
		\newpage%
		\begin{center}%
		\stepcounter{LAnexos}%
		\refstepcounter{contaAnexos}%
		\thecontaAnexos\label{##2}\par%
		[##1]%
		\vspace{10pt}%
		\leavevmode
		\end{center}
		
	}

	% Configuracion de capitulos -----------------
	\newcommand{\Centerformat}{
		\setcounter{secnumdepth}{4}
		\titleformat{\chapter}[display]
		{\large\bfseries\centering}{\MakeUppercase{\chaptertitlename\ \thechapter}}{0pt}{\large\MakeUppercase}
		\titlespacing{\chapter}
		{0pt}{-5pt}{20pt}
		
		\titlecontents{chapter}% <section-type>
			[3.22cm]% <left>
			{\addvspace{7pt}}% <above-code>
			{\hspace{-3.22cm}\MakeUppercase{\bfseries\chaptername}\hspace{0.1em}
			\makebox[10pt][l]{\bfseries\thecontentslabel}\bfseries\quad\uppercase}% <numbered-entry-format>
			{\hspace{-3.22cm}\bfseries\uppercase}% <numberless-entry-format>
			{\hfill\contentspage}[\addvspace{7pt}]
	}

	\newcommand{\CenterformatBack}{
		\titleformat{\chapter}[display]
		{\large\bfseries\centering}{\MakeUppercase{\chaptertitlename\ \thechapter}}{0pt}{\large\MakeUppercase}
		\titlespacing{\chapter}
		{0pt}{-15pt}{30pt}
		
		\titlecontents{chapter}% <section-type>
			[0pt]% <left>
			{\addvspace{0pt}}% <above-code>
			{}% <numbered-entry-format>
			{\bfseries\normalsize\uppercase}% <numberless-entry-format>
			{\titlerule*[8pt]{.}\contentspage}[\addvspace{0pt}]
	}

	\newcommand{\CenterformatFront}{
		\setcounter{secnumdepth}{-1}
		\titleformat{\chapter}[display]
		{\large\bfseries\centering}{\MakeUppercase{\chaptertitlename\ \thechapter}}{0pt}{\large\MakeUppercase}
		\titlespacing{\chapter}
		{0pt}{-25pt}{10pt}
		
		\titlecontents{chapter}% <section-type>
			[0pt]% <left>
			{\addvspace{0pt}}% <above-code>
			{}% <numbered-entry-format>
			{\bfseries\normalsize\uppercase}% <numberless-entry-format>
			{\titlerule*[8pt]{.}\contentspage}[\addvspace{0pt}]
	}
}

% FORMATO NUMERADO FIN -----------------------
% --------------------------------------------

% ---------------------------------------------------------------------

%\let\svsubsubsection\subsubsection %separacion del bloque de texto por seccion
%\def\subsubsection{\leftskip 1.25cm\svsubsubsection} 

%\BeforeBeginEnvironment{equation}{\vspace{7pt}}
%\AfterEndEnvironment{equation}{\leavevmode}

\BeforeBeginEnvironment{align}{\vspace{-10pt}}
\AfterEndEnvironment{align}{\vspace{-30pt}\leavevmode}

\newenvironment{longlisting}{\captionsetup{type=listing}}{}
\tcbuselibrary{breakable}
\BeforeBeginEnvironment{minted}{\begin{tcolorbox}[colback=colorcodes,breakable]}
\AfterEndEnvironment{minted}{\end{tcolorbox}}

\BeforeBeginEnvironment{longlisting}{\vspace*{9pt}}
\AfterEndEnvironment{longlisting}{\vspace*{-3pt}}

% ---------------------------------------------------------------------

\setlist[itemize]{leftmargin=1.25cm}

%Inicio del macro para el sortdlist ( ordenar un itemize ) ------------
\newcommand{\sortitem}[2][\relax]{%
	\DTLnewrow{list}% Create a new entry
	\ifx#1\relax
	\DTLnewdbentry{list}{sortlabel}{#2}% Add entry sortlabel (no optional argument)
	\else
	\DTLnewdbentry{list}{sortlabel}{#1}% Add entry sortlabel (optional argument)
	\fi%
	\DTLnewdbentry{list}{description}{#2}% Add entry description
}
\newenvironment{sortedlist}{%
	\DTLifdbexists{list}{\DTLcleardb{list}}{\DTLnewdb{list}}% Create new/discard old list
}{%
	\DTLsort{sortlabel}{list}% Sort list
	\begin{itemize}[label={}, leftmargin=0pt, labelwidth=0pt, labelsep=0pt]%
		\DTLforeach*{list}{\theDesc=description}{%
			\item \theDesc}% Print each item
	\end{itemize}%
}
% ---------------------------------------------------------------------

% para modificar el ambiente " qoute "
%\AtBeginEnvironment{quote}{\addtolength\leftmargini{1.25cm}}


% Para modificar el Y en las referencias y bibliografia
\DeclareDelimFormat*{finalnamedelim}
{\ifnum\value{liststop}>2 \fi\addspace y \addspace}

\DeclareDelimFormat[bib,biblist]{finalnamedelim}{%
	\ifthenelse{\value{listcount}>\maxprtauth}
	{}
	{\ifthenelse{\value{liststop}>2}
		{\addspace y \addspace}
		{\addspace y \addspace}}}

\DeclareDelimFormat*{finalnamedelim:apa:family-given}{%
	\ifthenelse{\value{listcount}>\maxprtauth}
	{}
	{\addspace y \addspace}}

\DeclareCaptionLabelSeparator*{spaced}{.\\}
\DeclareCaptionLabelFormat{numero}{\textbf{#1 #2}}

% Con encabezado y pie de pagina
\newcommand{\Addheadfoot}{
\newgeometry{right=3cm, left=4cm, top=3cm, bottom=3cm, includeheadfoot, headheight=15pt, heightrounded}
	\newcommand{\MainStyle}{
		\fancypagestyle{Main}{
			\fancyhf{}% clear all fields
			\renewcommand{\headrulewidth}{0.4pt}%
			\renewcommand{\footrulewidth}{0.4pt}
			\rhead{\rightmark}
			\lhead{\leftmark}
			\rfoot{Autor: Kleiver Carrasco}%
			\cfoot{\thepage}
		}
	
		\fancypagestyle{plain}{% for the chapter start pages
			\fancyhf{}% clear all fields
			\renewcommand{\headrulewidth}{0pt}%
			\renewcommand{\footrulewidth}{0.4pt}
			\rfoot{Autor: Kleiver Carrasco}%
			\cfoot{\thepage}
		}
	
		\pagestyle{Main}
	}
	
	\newcommand{\FrontBackStyle}{
		\fancypagestyle{plain}{% for the chapter start pages
			\fancyhf{}% clear all fields
			\renewcommand{\headrulewidth}{0pt}%
			\renewcommand{\footrulewidth}{0pt}
			\fancyfoot[C]{\thepage}%
		}
	
		\pagestyle{plain}
	}
}

% sin encabezado y sin pie de pagina
\newcommand{\noAddheadfoot}{
	
	\newcommand{\FrontBackStyle}{
		\fancypagestyle{plain}{% for the chapter start pages
			\fancyhf{}% clear all fields
			\renewcommand{\headrulewidth}{0pt}%
			\renewcommand{\footrulewidth}{0pt}
			\fancyfoot[C]{\thepage}%
		}
		
		\pagestyle{plain}}
	
	\newcommand{\MainStyle}{
		\fancypagestyle{Main}{% for the chapter start pages
			\fancyhf{}% clear all fields
			\renewcommand{\headrulewidth}{0pt}%
			\renewcommand{\footrulewidth}{0pt}
			\fancyfoot[C]{\thepage}%
		}
		
		\pagestyle{Main}}
}

% Solo encabezado
\newcommand{\AddheadOnly}{
	\newgeometry{right=3cm, left=4cm, top=3cm, bottom=3cm, includeheadfoot, heightrounded}
	\newcommand{\MainStyle}{
		\fancypagestyle{Main}{
			\fancyhf{}% clear all fields
			\renewcommand{\headrulewidth}{0.4pt}%
			\renewcommand{\footrulewidth}{0pt}
			\rhead{}
			\lhead{\leftmark}
			\cfoot{\thepage}
		}
		
		\fancypagestyle{plain}{% for the chapter start pages
			\fancyhf{}% clear all fields
			\renewcommand{\headrulewidth}{0pt}%
			\renewcommand{\footrulewidth}{0pt}
			\cfoot{\thepage}
		}
		
		\pagestyle{Main}
	}
	
	\newcommand{\FrontBackStyle}{
		\fancypagestyle{plain}{% for the chapter start pages
			\fancyhf{}% clear all fields
			\renewcommand{\headrulewidth}{0pt}%
			\renewcommand{\footrulewidth}{0pt}
			\fancyfoot[C]{\thepage}%
		}
		
		\pagestyle{plain}
	}
}

% \addtocontents{toc}{~\hfill\par}
\addtocontents{toc}{~\hfill Pág.\par}
\addtocontents{toc}{~\hfill\par}
\addtocontents{toc}{\protect\afterpage{~\hfill Pág.\par\bigskip}}

% \addtocontents{lof}{~\hfill\par}
\addtocontents{lof}{\hspace{-1.5cm}FIGURA~\hfill Pág.\par}
\addtocontents{lof}{~\hfill\par}
\addtocontents{lof}{\protect\afterpage{\hspace{-1.27cm}FIGURA~\hfill Pág.\par\bigskip}}

% \addtocontents{lot}{~\hfill\par}
\addtocontents{lot}{\hspace{-1.5cm}TABLA~\hfill Pág.\par}
\addtocontents{lot}{~\hfill\par}
%\addtocontents{lot}{\protect\afterpage{\hspace{-1.27cm}TABLA~\hfill Pág.\par\medskip}}

% \addtocontents{lol}{~\hfill\par}
\addtocontents{lol}{\hspace{-1.5cm}CÓDIGO~\hfill Pág.\par}
\addtocontents{lol}{~\hfill\par}
%\addtocontents{lol}{\protect\afterpage{\hspace{-1.27cm}CÓDIGO~\hfill Pág.\par\medskip}}


\newcommand*{\noaddvspace}{\renewcommand*{\addvspace}[1]{}}
\addtocontents{lof}{\protect\noaddvspace}
\addtocontents{lot}{\protect\noaddvspace}
\addtocontents{lol}{\protect\noaddvspace}


% Para arreglar el spacing en el ambiente quoting
\makeatletter
\renewcommand*\singlespacing{%
	\par    % ensure vertical mode
	\null   % add fake line with previous leading still in force
	\setstretch {\setspace@singlespace}% change leading
	\nobreak
	\vskip -\baselineskip   % compensate for the fake line we added, but with 
	% the new leading
	\vskip \z@skip  % tell "\addvspace" and "\addpenalty" _not_ to remove the 
	% above correction
}
\makeatother

% Para la separacion de ecuaciones
\makeatletter
\g@addto@macro\normalsize{\setlength\abovedisplayshortskip{0pt}}
\g@addto@macro\normalsize{\setlength\belowdisplayshortskip{0pt}}
\g@addto@macro\normalsize{\setlength\abovedisplayskip{17pt}}
\g@addto@macro\normalsize{\setlength\belowdisplayskip{12pt}}
\makeatother

% Para mostrar correctamente el showframe con landscape
\makeatletter
\newcommand*{\gmshow@textheight}{\textheight}
\newdimen\gmshow@@textheight
\g@addto@macro\landscape{%
  \gmshow@@textheight=\hsize
  \renewcommand*{\gmshow@textheight}{\gmshow@@textheight}%
}
\def\Gm@vrule{%
  \vrule width 0.2pt height\gmshow@textheight depth\z@
}%
\makeatother

% Ambiente para el resumen (abstract)
\newcommand\abstractname{Abstract}  %%% here
\makeatletter
% \if@titlepage
%   \newenvironment{abstract}{%
%       \titlepage
%       \null\vfil
%       \@beginparpenalty\@lowpenalty
%       \begin{center}%
%         \bfseries \abstractname
%         \@endparpenalty\@M
%       \end{center}}%
%      {\par\vfil\null\endtitlepage}
% \else
  \newenvironment{abstract}{%
      \if@twocolumn
        \section*{\abstractname}%
      \else
        \small
        \begin{center}%
          {\bfseries {\large RESUMEN}\vspace{10pt}}%
        \end{center}%
		\begin{spacing}{1}
			\parskip=0pt plus 1pt
      \fi}
		{\if@twocolumn\else\end{spacing}\fi}
% \fi
\makeatother

% Keywords command
\providecommand{\keywords}[1]
{	
	\vspace{10pt}
  	{\small	
  	\noindent\textbf{\textit{Descriptores: }} #1
  	}
}

% ACTIVACION DEL HEADER O FOTTER -----------------------------------------
%\Addheadfoot
%\AddheadOnly
\noAddheadfoot