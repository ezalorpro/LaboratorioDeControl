% \vspace{15pt}

El presente trabajo plantea la creación de un software enfocado en el área de los sistemas de control. Así como MATLAB es un laboratorio de matrices y SciLab es un laboratorio científico, el Laboratorio Virtual de sistemas de control clásicos y difusos apunta a algo similar con el propósito de utilizarse en el área de control por parte de ingenieros que puedan estar buscando una alternativa simple y gratuita para abarcar los problemas del control de procesos.

Los sistemas de control han incrementado su importancia en el tiempo, lo cual se debe a la necesidad de automatizar los procesos de producción de bienes y servicios para ofrecer productos de mejor calidad a un mejor precio. Los procesos a controlar para la producción son, con frecuencia, complejos y no lineales, por tanto, es común el uso de controladores diferentes al clásico PID, una de estas alternativas se basa en emplear controladores con base en la lógica difusa para contrarrestar los efectos no lineales del proceso a controlar.

Así mismo, la problemática a resolver es la necesidad existente de software para el análisis, diseño y simulación de sistemas de control, el cual, existe de manera reducida de forma gratuita y libre. Opciones como SciLab y Octave no pueden competir con entornos cerrados y de pago como MATLAB en cuanto a opciones y desempeño, adicionalmente, la posibilidad de implementar controladores difusos es tratado como una función opcional y de poca prioridad que no viene por defecto en las alternativas de software mencionados.

\looseness=-1000
Con el objetivo de solventar esta problemática se plantea la creación de un Laboratorio Virtual utilizando el lenguaje de programación libre y gratuito Python, el cual posee librerías externas aptas para la tarea expuesta, todo lo anterior, con la finalidad de otorgar la posibilidad de analizar, diseñar y simular sistemas de control clásicos y difusos por medio de una interfaz gráfica sin la necesidad de instalar elementos externos o escribir alguna línea de código de programación a la vez que se realiza de forma rápida y sencilla.

A continuación, se describe de forma más especifica la problemática que atiende esta investigación. Pasado este capítulo, se exponen los conceptos teóricos necesarios para poder implementar un software con las características ya descritas que logre competir con otras herramientas de corte similar, luego, se continua con la metodología empleada para llevar a cabo con éxito la realización del Laboratorio Virtual.  Así mismo, se presenta la estructura final del software realizado y los resultados que se obtuvieron en forma de una comparación numérica y grafica entre el Laboratorio Virtual, MATLAB y SciLab. Finalmente, se presentan las conclusiones obtenidas y las recomendaciones que se consideraron de prioridad para mejorar el Laboratorio Virtual a futuro.