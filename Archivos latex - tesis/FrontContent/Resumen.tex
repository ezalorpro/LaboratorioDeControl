\begin{center}
	\begin{spacing}{1}
		UNIVERSIDAD NACIONAL EXPERIMENTAL DEL TÁCHIRA
		
		VICERRECTORADO ACADÉMICO
		
		DECANATO DE DOCENCIA
		
		CARRERA DE INGENIERÍA ELECTRÓNICA
	
		\vspace{35pt}

		{\large \textbf{Laboratorio virtual de sistemas de control clásicos y difusos utilizando software libre}\par}
	
	\end{spacing}
\end{center}

\vspace{25pt}

\begin{flushright}
	\begin{spacing}{1}
		\parskip=0pt plus 1pt

		Autor: Kleiver J. Carrasco M.

		Tutor: MSc. Ing. Juan R. Vizcaya R.

		Fecha: Marzo, 2020
		
	\end{spacing}	
\end{flushright}

\vspace{15pt}

\begin{abstract}
	La investigación realizada se llevó a cabo con la finalidad de crear un software enfocado en el área de los sistemas de control. El Laboratorio Virtual de sistemas de control clásicos y difusos apunta a utilizarse de forma similar a MATLAB y SciLab, particularmente, con el propósito de hacer uso del mismo en el área del control de procesos haciendo uso de software libre, específicamente, el lenguaje de programación libre Python. Con la finalidad de cumplir con los objetivos planteados se hizo uso de librerías para Python externas de control de procesos, lógica difusa, cálculo numérico, entre otras, además, se realizó la implementación de algoritmos de solución de ecuaciones diferenciales para la simulación de los sistemas de control con controladores difusos. El resultado de la investigación fue una aplicación confiable y práctica para el análisis, diseño y simulación de controladores clásicos y difusos manteniendo un enfoque simple y de uso rápido que logra competir en cierta medida con otras herramientas de corte similar, y a su vez, dejando abierta la posibilidad de implementar mejoras a futuro o la creación de nuevas funciones.
\end{abstract}

\keywords{Sistemas de control, logica difusa, Python, Laboratorio Virtual}
